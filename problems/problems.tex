\documentclass[12pt]{article}
\usepackage[utf8]{inputenc}
\usepackage[russian]{babel}
\usepackage{geometry}
\usepackage{listings}
\usepackage{xcolor}
\usepackage{titlesec}
\usepackage{amssymb}

\geometry{left=2cm, right=2cm, top=2cm, bottom=2cm}

% Настройка listings для Java
\lstset{
    language=Java,
    basicstyle=\ttfamily\small,
    keywordstyle=\color{blue},
    commentstyle=\color{green!60!black},
    stringstyle=\color{red},
    showstringspaces=false,
    numbers=left,
    numberstyle=\tiny\color{gray},
    frame=single,
    breaklines=true,
    breakatwhitespace=false,
    tabsize=4,
    captionpos=b
}

\titleformat{\section}{\large\bfseries\centering}{\thesection}{1em}{}
\titleformat{\subsection}{\normalsize\bfseries}{\thesubsection}{1em}{}

\title{Задачник по Java \\ Занятие 1: Управляющие конструкции}
\author{}
\date{}

\begin{document}

\maketitle

\section*{Введение}
На этом занятии вы познакомитесь с основными управляющими конструкциями языка Java: условными операторами (\texttt{if}, \texttt{switch}) и циклами (\texttt{for}, \texttt{while}, \texttt{do\_while}). Каждая тема включает краткое пояснение, пример и набор задач для самостоятельного решения.

% Сюда будем добавлять разделы по мере ваших указаний

\section{Условный оператор \texttt{if}}

\subsection*{Задачи}

Решите следующие задачи, используя условные операторы \texttt{if}. Обеспечьте полную проверку ввода данных и корректную обработку всех математических случаев (деление на ноль, вырожденные интервалы, отрицательные значения и т.п.).

\begin{enumerate}
    \item Напишите программу, которая решает неравенство $(x - a)(x - b) > 0$, где $a$ и $b$ — вещественные числа, вводимые пользователем.
    \item Напишите программу, которая решает неравенство $(x - a)(x - b) < 0$, где $a$ и $b$ — вещественные числа, вводимые пользователем.
    \item Напишите программу, которая решает неравенство $(x - a)(x - b) \geqslant 0$, где $a$ и $b$ — вещественные числа, вводимые пользователем.
    \item Напишите программу, которая решает неравенство $(x - a)(x - b) \leqslant 0$, где $a$ и $b$ — вещественные числа, вводимые пользователем.
    \item Напишите программу, которая решает неравенство $(x + a)(x + b) > 0$, где $a$ и $b$ — вещественные числа, вводимые пользователем.
    \item Напишите программу, которая решает неравенство $(x + a)(x + b) < 0$, где $a$ и $b$ — вещественные числа, вводимые пользователем.
    \item Напишите программу, которая решает неравенство $(a - x)(b - x) > 0$, где $a$ и $b$ — вещественные числа, вводимые пользователем.
    \item Напишите программу, которая решает неравенство $(a - x)(b - x) < 0$, где $a$ и $b$ — вещественные числа, вводимые пользователем.
    \item Напишите программу, которая решает неравенство $(x - a)^2 > 0$, где $a$ — вещественное число, вводимое пользователем.
    \item Напишите программу, которая решает неравенство $(x - a)^2 \geqslant 0$, где $a$ — вещественное число, вводимое пользователем.
    \item Напишите программу, которая решает неравенство $\frac{x - a}{x - b} > 0$, где $a$ и $b$ — вещественные числа, вводимые пользователем.
    \item Напишите программу, которая решает неравенство $\frac{x - a}{x - b} < 0$, где $a$ и $b$ — вещественные числа, вводимые пользователем.
    \item Напишите программу, которая решает неравенство $\frac{x - a}{x - b} \geqslant 0$, где $a$ и $b$ — вещественные числа, вводимые пользователем.
    \item Напишите программу, которая решает неравенство $\frac{x - a}{x - b} \leqslant 0$, где $a$ и $b$ — вещественные числа, вводимые пользователем.
    \item Напишите программу, которая решает неравенство $|x - a| > b$, где $a$ и $b$ — вещественные числа, вводимые пользователем.
    \item Напишите программу, которая решает неравенство $|x - a| < b$, где $a$ и $b$ — вещественные числа, вводимые пользователем.
    \item Напишите программу, которая решает неравенство $|x - a| \geqslant b$, где $a$ и $b$ — вещественные числа, вводимые пользователем.
    \item Напишите программу, которая решает неравенство $|x - a| \leqslant b$, где $a$ и $b$ — вещественные числа, вводимые пользователем.
    \item Напишите программу, которая решает неравенство $(x - a)(x - b)(x - c) > 0$, где $a$, $b$, $c$ — вещественные числа, вводимые пользователем.
    \item Напишите программу, которая решает неравенство $(x - a)(x - b)(x - c) < 0$, где $a$, $b$, $c$ — вещественные числа, вводимые пользователем.
    \item Напишите программу, которая определяет, принадлежит ли точка $x$ интервалу $(a; b)$, где $a$, $b$, $x$ — вещественные числа, вводимые пользователем.
    \item Напишите программу, которая определяет, принадлежит ли точка $x$ отрезку $[a; b]$, где $a$, $b$, $x$ — вещественные числа, вводимые пользователем.
    \item Напишите программу, которая определяет, лежит ли число $x$ вне отрезка $[a; b]$, где $a$, $b$, $x$ — вещественные числа, вводимые пользователем.
    \item Напишите программу, которая решает систему неравенств $x > a$ и $x < b$, где $a$ и $b$ — вещественные числа, вводимые пользователем.
    \item Напишите программу, которая решает совокупность неравенств $x < a$ или $x > b$, где $a$ и $b$ — вещественные числа, вводимые пользователем.
\end{enumerate}


\section{Оператор \texttt{switch}}

\subsection*{Задачи}

Решите следующие задачи, используя оператор \texttt{switch}. Запрещено использовать стандартные классы для работы с датами — дата задаётся тремя целыми числами: день, месяц, год. Обеспечьте полную проверку корректности ввода (существование даты, високосный год, допустимые диапазоны).

\begin{enumerate}
    \item По введённой дате определите дату следующего дня. Выведите её и проверьте, совпадает ли количество дней в месяце исходной даты с количеством дней в месяце полученной даты.
    \item По введённой дате определите дату предыдущего дня. Выведите её и проверьте, совпадает ли количество дней в месяце исходной даты с количеством дней в месяце полученной даты.
    \item По введённой дате определите дату, которая наступит ровно через месяц (прибавить 1 к месяцу, при необходимости корректируя год). Если в следующем месяце нет дня с таким же числом (например, 31 апреля), то возьмите последний день следующего месяца. Выведите полученную дату и проверьте, является ли она последним днём месяца.
    \item По введённой дате определите дату, которая была ровно месяц назад (вычесть 1 из месяца, при необходимости корректируя год). Если в предыдущем месяце нет дня с таким же числом, возьмите последний день предыдущего месяца. Выведите полученную дату и проверьте, является ли она первым днём месяца.
    \item По введённой дате определите дату, которая наступит через 3 месяца (прибавить 3 к месяцу, корректируя год). Корректировка дня, как в предыдущих задачах. Выведите полученную дату и проверьте, находится ли она в том же квартале года, что и исходная дата. (Кварталы: 1-3, 4-6, 7-9, 10-12)
    \item По введённой дате определите дату, которая была 3 месяца назад. Выведите полученную дату и проверьте, находится ли она в том же году, что и исходная дата.
    \item По введённой дате определите дату, которая наступит через 1 год (прибавить 1 к году). Учтите високосность года для февраля. Если исходная дата - 29 февраля, то в следующем невисокосном году возьмите 28 февраля. Выведите полученную дату и проверьте, является ли она високосным днём (29 февраля).
    \item По введённой дате определите дату, которая была 1 год назад. Выведите полученную дату и проверьте, была ли исходная дата високосным днём (29 февраля), а полученная - нет.
    \item По введённой дате определите дату, которая наступит через 100 дней. Выведите её и проверьте, является ли полученная дата последним днём месяца.
    \item По введённой дате определите дату, которая была 100 дней назад. Выведите её и проверьте, является ли полученная дата первым днём месяца.
    \item По введённой дате определите дату, которая наступит через 1 неделю (7 дней). Выведите её и проверьте, находится ли полученная дата в том же месяце, что и исходная.
    \item По введённой дате определите дату, которая была 1 неделю назад. Выведите её и проверьте, находится ли полученная дата в том же году, что и исходная.
    \item По введённой дате определите дату, которая наступит через 2 месяца. Выведите её и проверьте, является ли день полученной даты последним днём месяца.
    \item По введённой дате определите дату, которая была 2 месяца назад. Выведите её и проверьте, является ли день полученной даты первым днём месяца.
    \item По введённой дате определите дату, которая наступит через 6 месяцев. Выведите её и проверьте, находится ли полученная дата во второй половине года (месяц с июля по декабрь).
    \item По введённой дате определите дату, которая была 6 месяцев назад. Выведите её и проверьте, находится ли полученная дата в первом полугодии (месяц с января по июнь).
    \item По введённой дате определите дату, которая наступит через 1 месяц и 1 день (сначала прибавить месяц, затем день). Корректировка дня, как в задаче 3. Выведите полученную дату и проверьте, является ли она первым днём месяца.
    \item По введённой дате определите дату, которая была 1 месяц и 1 день назад (сначала вычесть месяц, затем день). Выведите полученную дату и проверьте, является ли она последним днём месяца.
    \item По введённой дате определите дату, которая наступит через 2 года. Выведите её и проверьте, является ли год полученной даты високосным.
    \item По введённой дате определите дату, которая была 2 года назад. Выведите её и проверьте, был ли год полученной даты високосным.
    \item По введённой дате определите дату, которая наступит через 1 квартал (3 месяца). Выведите её и проверьте, является ли полученная дата последним днём квартала (31 марта, 30 июня, 30 сентября, 31 декабря).
    \item По введённой дате определите дату, которая была 1 квартал назад. Выведите её и проверьте, является ли полученная дата первым днём квартала (1 января, 1 апреля, 1 июля, 1 октября).
    \item По введённой дате определите дату, которая наступит через 1 год и 1 месяц. Выведите её и проверьте, является ли день полученной даты первым числом месяца.
    \item По введённой дате определите дату, которая была 1 год и 1 месяц назад. Выведите её и проверьте, является ли день полученной даты последним числом месяца.
    \item По введённой дате определите дату, которая наступит через 366 дней (чтобы перепрыгнуть через год). Выведите её и проверьте, является ли полученная дата високосным днём (29 февраля).
\end{enumerate}

\section{Оператор \texttt{do...while}}

\subsection*{Задачи}

Решите следующие задачи, используя цикл \texttt{do...while}. Все задачи предполагают последовательный ввод чисел, оканчивающийся нулём. Нулевое значение является признаком окончания ввода и в вычислениях \textbf{не участвует}. Обеспечьте корректную обработку граничных случаев: пустая последовательность (только 0), отсутствие подходящих чисел, деление на ноль, извлечение корня из отрицательного числа и т.п. При необходимости выводите сообщения об ошибках.

\textbf{Указание}. Для целочисленных операций:
\begin{enumerate}
\item Остаток при делении $a$ на $b$: \verb|a \% b|.
    \item Целая часть частного: \verb|a / b| (при целочисленном делении).
    \item Последняя цифра числа $n$: \verb|n \% 10|.
    \item Предпоследняя цифра: \verb|(n / 10) \% 10|.
\end{enumerate}

\begin{enumerate}
    \item Последовательно вводятся вещественные числа, оканчивающиеся нулём. Выведите максимальное число и количество чисел, больших 5 (кроме завершающего нуля).

    \item Последовательно вводятся целые числа, оканчивающиеся нулём. Выведите минимальное число и количество чисел, у которых последняя цифра равна 0 (кроме завершающего нуля).

    \item Последовательно вводятся вещественные числа, оканчивающиеся нулём. Выведите сумму синусов всех чисел и третье число последовательности (если чисел меньше трёх — вывести сообщение об ошибке).

    \item Последовательно вводятся целые числа, оканчивающиеся нулём. Выведите сумму всех нечётных чисел и количество чисел, делящихся на 3 (кроме завершающего нуля).

    \item Последовательно вводятся целые числа, оканчивающиеся нулём. Выведите количество двузначных натуральных чисел и минимальную последнюю цифру среди всех введённых чисел (кроме завершающего нуля).

    \item Последовательно вводятся натуральные числа, оканчивающиеся нулём. Выведите количество трёхзначных палиндромов (чисел, которые читаются одинаково слева направо и справа налево, например, 121, 343) (кроме завершающего нуля).

    \item Последовательно вводятся целые числа, оканчивающиеся нулём. Выведите сумму всех чисел и предпоследнее число последовательности (если чисел меньше двух — вывести сообщение об ошибке).

    \item Последовательно вводятся целые числа, оканчивающиеся нулём. Выведите произведение всех чисел (кроме завершающего нуля) и второе число последовательности (если чисел меньше двух — вывести сообщение об ошибке).

    \item Последовательно вводятся целые числа, оканчивающиеся нулём. Выведите среднее арифметическое всех чисел и максимум модуля введённых чисел (кроме завершающего нуля).

    \item Последовательно вводятся вещественные числа, оканчивающиеся нулём. Выведите среднее геометрическое всех чисел (кроме завершающего нуля) и минимум модуля введённых чисел. \\
    \textbf{Примечание}: среднее геометрическое определено только для положительных чисел. Если есть неположительные — вывести сообщение об ошибке. \\
    Формула: $\left(a_1 a_2 \dots a_n\right)^{1/n}$.

    \item Последовательно вводятся вещественные числа, оканчивающиеся нулём. Выведите среднее квадратическое всех чисел (кроме завершающего нуля) и минимум квадрата введённых чисел. \\
    Формула: $\sqrt{\frac{a_1^2 + a_2^2 + \dots + a_n^2}{n}}$.

    \item Последовательно вводятся вещественные числа, оканчивающиеся нулём. Выведите среднее гармоническое всех чисел (кроме завершающего нуля) и максимум квадрата введённых чисел. \\
    \textbf{Примечание}: среднее гармоническое не определено, если есть нули или числа разных знаков. Проверяйте знаменатель. \\
    Формула: $\frac{n}{\frac{1}{a_1} + \frac{1}{a_2} + \dots + \frac{1}{a_n}}$.

    \item Последовательно вводятся вещественные числа, оканчивающиеся нулём. Выведите среднее арифметическое модулей всех чисел и максимум синусов введённых чисел (кроме завершающего нуля).

    \item Последовательно вводятся вещественные числа, оканчивающиеся нулём. Выведите среднее гармоническое модулей всех чисел (кроме завершающего нуля) и минимум синусов введённых чисел. \\
    \textbf{Примечание}: модули положительны — среднее гармоническое определено, если только не все числа нулевые.

    \item Последовательно вводятся вещественные числа, оканчивающиеся нулём. Выведите среднее квадратическое модулей всех чисел (кроме завершающего нуля) и минимум косинусов введённых чисел.

    \item Последовательно вводятся вещественные числа, оканчивающиеся нулём. Выведите среднее геометрическое модулей всех чисел (кроме завершающего нуля) и максимум косинусов введённых чисел. \\
    \textbf{Примечание}: модули неотрицательны — если есть ноль, среднее геометрическое = 0.

    \item Последовательно вводятся натуральные числа, оканчивающиеся нулём. Выведите среднее арифметическое квадратов всех чисел (кроме завершающего нуля) и максимальную последнюю цифру среди всех чисел.

    \item Последовательно вводятся натуральные числа, оканчивающиеся нулём. Выведите среднее геометрическое квадратов всех чисел (кроме завершающего нуля) и минимальную последнюю цифру среди всех чисел.

    \item Последовательно вводятся натуральные числа, оканчивающиеся нулём. Выведите среднее квадратическое квадратов всех чисел (кроме завершающего нуля) и максимальную предпоследнюю цифру среди всех чисел.

    \item Последовательно вводятся натуральные числа, оканчивающиеся нулём. Выведите среднее гармоническое квадратов всех чисел (кроме завершающего нуля) и минимальную предпоследнюю цифру среди всех чисел.

    \item Последовательно вводятся натуральные числа от 1 до 999, оканчивающиеся нулём. Выведите максимальную сумму цифр в числах и среднее арифметическое сумм цифр (кроме завершающего нуля).

    \item Последовательно вводятся натуральные числа от 1 до 999, оканчивающиеся нулём. Выведите минимальную сумму цифр в числах и среднее гармоническое сумм цифр (кроме завершающего нуля).

    \item Последовательно вводятся натуральные числа от 1 до 999, оканчивающиеся нулём. Выведите минимальную сумму количества сотен и единиц в числах и среднее геометрическое сумм цифр (кроме завершающего нуля). \\
    \textbf{Пример}: для числа 347: сотни = 3, единицы = 7, сумма = 10.

    \item Последовательно вводятся натуральные числа от 1 до 999, оканчивающиеся нулём. Выведите максимальную сумму количества сотен и единиц в числах и среднее квадратическое сумм цифр (кроме завершающего нуля).

    \item Последовательно вводятся целые числа, оканчивающиеся нулём. Выведите среднее геометрическое всех чётных чисел (кроме завершающего нуля) и максимум среди нечётных чисел. \\
    \textbf{Примечание}: если чётных чисел нет — вывести сообщение об ошибке. Учтите, что среднее геометрическое требует положительных значений.
\end{enumerate}

\section{Цикл \texttt{for}}

\subsection*{Задачи}

Решите следующие задачи, используя цикл \texttt{for}. Все задачи должны использовать именно \texttt{for} (не \texttt{while} или \texttt{do...while}). Обеспечьте корректную обработку граничных случаев: деление на ноль, отрицательные числа, пустые диапазоны и т.п.

\begin{enumerate}
    \item Найдите количество трёхзначных чисел в диапазоне $[100; 999]$, в которых вторая цифра равна сумме первой и третьей цифры.

    \item Найдите количество трёхзначных чисел в диапазоне $[100; 999]$, в которых сумма первых двух цифр равна третьей цифре.

    \item Найдите количество трёхзначных чисел в диапазоне $[100; 999]$, в которых сумма последних двух цифр равна первой цифре.

    \item Найдите все натуральные числа в диапазоне $[m; n]$ ($1 \leqslant m \leqslant n \leqslant 999$), которые равны сумме квадратов своих цифр. \\
    \textbf{Пример}: $1^2 + 3^2 + 0^2 = 10$ — не подходит; $1^2 + 6^2 + 3^2 = 46$ — не подходит.

    \item Найдите все натуральные числа в диапазоне $[m; n]$ ($1 \leqslant m \leqslant n \leqslant 999$), которые равны сумме кубов своих цифр. \\
    \textbf{Пример}: $153 = 1^3 + 5^3 + 3^3$ — подходит.

    \item Найдите все натуральные числа в диапазоне $[m; n]$ ($1 \leqslant m \leqslant n \leqslant 999$), которые равны сумме своих цифр. \\
    \textbf{Пример}: $18 = 1 + 8 = 9$ — не подходит; $1 = 1$ — подходит.

    \item Найдите все натуральные делители числа $n \in \mathbb{N}$ ($n > 0$). Выведите их в порядке возрастания.

    \item Определите, является ли число $n \in \mathbb{N}$ ($n > 1$) простым. Выведите «Да» или «Нет».

    \item Найдите все натуральные числа в диапазоне $[m; n]$ ($1 \leqslant m \leqslant n \leqslant 999$), которые делятся на свою последнюю цифру. \\
    \textbf{Примечание}: если последняя цифра — 0, число не учитывается (деление на ноль).

    \item Напечатайте таблицу перевода двоичных чисел от $1_2$ до $11111_2$ (т.е. от 1 до 31 в десятичной) в десятичную систему счисления.

    \item Напечатайте таблицу перевода восьмеричных чисел от $1_8$ до $777_8$ (т.е. от 1 до 511 в десятичной) в десятичную систему счисления.

    \item Напечатайте таблицу умножения (от 1×1 до 10×10).

    \item Напечатайте первые 20 чисел Фибоначчи ($f_1 = 1$, $f_2 = 1$, $f_{n} = f_{n-1} + f_{n-2}$ для $n > 2$).

    \item Найдите все трёхзначные числа в диапазоне $[100; 999]$, которые при зачёркивании средней цифры уменьшаются в 7 раз. \\
    \textbf{Пример}: число 357 → зачёркиваем 5 → получаем 37; $357 / 37 = 9.648$ — не подходит.

    \item Найдите сумму всех натуральных делителей числа $n \in \mathbb{N}$ ($n > 0$).

    \item Вычислите $a^n$, где $a \in \mathbb{R}$, $n \in \mathbb{Z}$, $n \geqslant 0$. \\
    \textbf{Примечание}: если $n < 0$, вывести сообщение об ошибке. Используйте только умножение (не \verb|Math.pow|).

    \item Найдите сумму всех нечётных натуральных чисел в диапазоне $[m; n]$ ($1 \leqslant m \leqslant n \leqslant 1000$).

    \item Найдите сумму всех чётных натуральных чисел в диапазоне $[m; n]$ ($1 \leqslant m \leqslant n \leqslant 1000$).

    \item Найдите все общие делители натуральных чисел $n$ и $m$ ($n > 0$, $m > 0$). Выведите их в порядке возрастания.

    \item Найдите все натуральные числа в диапазоне $[m; n]$ ($10 \leqslant m \leqslant n \leqslant 999$), которые делятся на свою предпоследнюю цифру. \\
    \textbf{Примечание}: если предпоследняя цифра — 0, число не учитывается.

    \item Напечатайте все трёхзначные палиндромы (числа, которые читаются одинаково слева направо и справа налево, например, 121, 343) в диапазоне $[100; 999]$.

    \item Найдите все трёхзначные числа в диапазоне $[100; 999]$, которые пропорциональны числу, составленному из второй и третьей цифр. \\
    \textbf{Пример}: число 135 → вторая и третья цифры = 35; $135 / 35 = 3.857$ — не целое → не подходит. \\
    \textbf{Уточнение}: пропорциональны = делятся без остатка.

    \item Найдите все четырёхзначные числа в диапазоне $[1000; 9999]$, в которых сумма первых двух цифр равна сумме последних двух цифр.

    \item Найдите все четырёхзначные числа в диапазоне $[1000; 9999]$, в которых сумма крайних цифр равна сумме средних цифр.

    \item Найдите все четырёхзначные числа в диапазоне $[1000; 9999]$, в которых сумма первой и третьей цифр равна сумме второй и четвёртой цифр.
\end{enumerate}

\section{Цикл \texttt{while}}

\subsection*{Задачи}

Решите следующие задачи, используя цикл \texttt{while}. Использование \texttt{for} или \texttt{do...while} не допускается. Все задачи предполагают, что количество итераций заранее неизвестно и определяется в процессе выполнения. Обеспечьте обработку граничных случаев: нули, единицы, отрицательные числа, переполнения.

\begin{enumerate}
    \item Дано натуральное число $n$. Найдите сумму его цифр, используя \texttt{while}.

    \item Дано натуральное число $n$. Найдите количество его цифр, используя \texttt{while}.

    \item Дано натуральное число $n$. Найдите произведение его цифр, используя \texttt{while}.

    \item Дано натуральное число $n$. Определите, является ли оно палиндромом (читается одинаково слева направо и справа налево), используя \texttt{while}. \\
    \textbf{Указание}: постройте зеркальное число и сравните.

    \item Дано натуральное число $n$. Удалите из него все чётные цифры и выведите результат (если получилось пустое число — вывести 0). Используйте \texttt{while}.

    \item Дано натуральное число $n$. Проверьте, является ли оно факториалом какого-либо натурального числа. Если да — выведите это число, иначе — сообщение «Не является факториалом». \\
    \textbf{Пример}: $120 = 5!$ → вывести 5.

    \item Дано натуральное число $n$. Найдите наименьшее $k$, такое что $k! \geqslant n$. Используйте \texttt{while}.

    \item Дано натуральное число $n$. Разложите его на простые множители и выведите их в порядке возрастания (с повторениями). Используйте \texttt{while}.

    \item Даны два натуральных числа $a$ и $b$. Найдите их наибольший общий делитель (НОД) с помощью алгоритма Евклида, используя \texttt{while}.

    \item Даны два натуральных числа $a$ и $b$. Найдите их наименьшее общее кратное (НОК), используя \texttt{while} и НОД.

    \item Дано натуральное число $n$. Переведите его в двоичную систему счисления, используя \texttt{while}. Выведите результат как число (не строку).

    \item Дано натуральное число $n$. Переведите его в восьмеричную систему счисления, используя \texttt{while}. Выведите результат как число.

    \item Дано натуральное число $n$. Определите, сколько раз в нём встречается цифра 7, используя \texttt{while}.

    \item Дано натуральное число $n$. Найдите максимальную цифру в числе, используя \texttt{while}.

    \item Дано натуральное число $n$. Найдите минимальную цифру в числе, используя \texttt{while}.

    \item Дано натуральное число $n$. Определите, содержит ли оно хотя бы одну цифру, равную 0, используя \texttt{while}.

    \item Дано натуральное число $n$. Определите, все ли его цифры нечётные, используя \texttt{while}.

    \item Дано натуральное число $n$. Найдите число, составленное из его цифр в обратном порядке (зеркальное отражение), используя \texttt{while}.

    \item Дано натуральное число $n$. Определите, является ли оно степенью двойки (т.е. $n = 2^k$ для некоторого $k \geqslant 0$), используя \texttt{while}.

    \item Дано натуральное число $n$. Определите, является ли оно степенью тройки, используя \texttt{while}.

    \item Дано натуральное число $n$. Найдите сумму всех его делителей, используя \texttt{while}.

    \item Дано натуральное число $n$. Определите, является ли оно совершенным (т.е. сумма его собственных делителей равна самому числу), используя \texttt{while}.

    \item Дано натуральное число $n$. Найдите количество нулей в его двоичном представлении, используя \texttt{while}.

    \item Дано натуральное число $n$. Найдите количество единиц в его двоичном представлении, используя \texttt{while}.

    \item Дано натуральное число $n$. Определите, можно ли его представить в виде суммы двух квадратов натуральных чисел, используя \texttt{while}. \\
    \textbf{Пример}: $25 = 3^2 + 4^2$ → можно.
\end{enumerate}

\end{document}