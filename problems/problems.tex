\documentclass[12pt]{article}
\usepackage[utf8]{inputenc}
\usepackage[russian]{babel}
\usepackage{geometry}
\usepackage{listings}
\usepackage{xcolor}
\usepackage{titlesec}
\usepackage{amssymb}

\geometry{left=2cm, right=2cm, top=2cm, bottom=2cm}

% Настройка listings для Java
\lstset{
    language=Java,
    basicstyle=\ttfamily\small,
    keywordstyle=\color{blue},
    commentstyle=\color{green!60!black},
    stringstyle=\color{red},
    showstringspaces=false,
    numbers=left,
    numberstyle=\tiny\color{gray},
    frame=single,
    breaklines=true,
    breakatwhitespace=false,
    tabsize=4,
    captionpos=b
}

\titleformat{\section}{\large\bfseries\centering}{\thesection}{1em}{}
\titleformat{\subsection}{\normalsize\bfseries}{\thesubsection}{1em}{}

\title{Задачник по Java \\ Занятие 1: Управляющие конструкции}
\author{}
\date{}

\begin{document}

\maketitle

\section*{Введение}
На этом занятии вы познакомитесь с основными управляющими конструкциями языка Java: условными операторами (\texttt{if}, \texttt{switch}) и циклами (\texttt{for}, \texttt{while}, \texttt{do\_while}). Каждая тема включает краткое пояснение, пример и набор задач для самостоятельного решения.

% Сюда будем добавлять разделы по мере ваших указаний

\section{Условный оператор \texttt{if}}

\subsection*{Задачи}

Решите следующие задачи, используя условные операторы \texttt{if}. Обеспечьте полную проверку ввода данных и корректную обработку всех математических случаев (деление на ноль, вырожденные интервалы, отрицательные значения и т.п.).

\begin{enumerate}
    \item Напишите программу, которая решает неравенство $(x - a)(x - b) > 0$, где $a$ и $b$ — вещественные числа, вводимые пользователем.
    \item Напишите программу, которая решает неравенство $(x - a)(x - b) < 0$, где $a$ и $b$ — вещественные числа, вводимые пользователем.
    \item Напишите программу, которая решает неравенство $(x - a)(x - b) \geqslant 0$, где $a$ и $b$ — вещественные числа, вводимые пользователем.
    \item Напишите программу, которая решает неравенство $(x - a)(x - b) \leqslant 0$, где $a$ и $b$ — вещественные числа, вводимые пользователем.
    \item Напишите программу, которая решает неравенство $(x + a)(x + b) > 0$, где $a$ и $b$ — вещественные числа, вводимые пользователем.
    \item Напишите программу, которая решает неравенство $(x + a)(x + b) < 0$, где $a$ и $b$ — вещественные числа, вводимые пользователем.
    \item Напишите программу, которая решает неравенство $(a - x)(b - x) > 0$, где $a$ и $b$ — вещественные числа, вводимые пользователем.
    \item Напишите программу, которая решает неравенство $(a - x)(b - x) < 0$, где $a$ и $b$ — вещественные числа, вводимые пользователем.
    \item Напишите программу, которая решает неравенство $(x - a)^2 > 0$, где $a$ — вещественное число, вводимое пользователем.
    \item Напишите программу, которая решает неравенство $(x - a)^2 \geqslant 0$, где $a$ — вещественное число, вводимое пользователем.
    \item Напишите программу, которая решает неравенство $\frac{x - a}{x - b} > 0$, где $a$ и $b$ — вещественные числа, вводимые пользователем.
    \item Напишите программу, которая решает неравенство $\frac{x - a}{x - b} < 0$, где $a$ и $b$ — вещественные числа, вводимые пользователем.
    \item Напишите программу, которая решает неравенство $\frac{x - a}{x - b} \geqslant 0$, где $a$ и $b$ — вещественные числа, вводимые пользователем.
    \item Напишите программу, которая решает неравенство $\frac{x - a}{x - b} \leqslant 0$, где $a$ и $b$ — вещественные числа, вводимые пользователем.
    \item Напишите программу, которая решает неравенство $|x - a| > b$, где $a$ и $b$ — вещественные числа, вводимые пользователем.
    \item Напишите программу, которая решает неравенство $|x - a| < b$, где $a$ и $b$ — вещественные числа, вводимые пользователем.
    \item Напишите программу, которая решает неравенство $|x - a| \geqslant b$, где $a$ и $b$ — вещественные числа, вводимые пользователем.
    \item Напишите программу, которая решает неравенство $|x - a| \leqslant b$, где $a$ и $b$ — вещественные числа, вводимые пользователем.
    \item Напишите программу, которая решает неравенство $(x - a)(x - b)(x - c) > 0$, где $a$, $b$, $c$ — вещественные числа, вводимые пользователем.
    \item Напишите программу, которая решает неравенство $(x - a)(x - b)(x - c) < 0$, где $a$, $b$, $c$ — вещественные числа, вводимые пользователем.
    \item Напишите программу, которая определяет, принадлежит ли точка $x$ интервалу $(a; b)$, где $a$, $b$, $x$ — вещественные числа, вводимые пользователем.
    \item Напишите программу, которая определяет, принадлежит ли точка $x$ отрезку $[a; b]$, где $a$, $b$, $x$ — вещественные числа, вводимые пользователем.
    \item Напишите программу, которая определяет, лежит ли число $x$ вне отрезка $[a; b]$, где $a$, $b$, $x$ — вещественные числа, вводимые пользователем.
    \item Напишите программу, которая решает систему неравенств $x > a$ и $x < b$, где $a$ и $b$ — вещественные числа, вводимые пользователем.
    \item Напишите программу, которая решает совокупность неравенств $x < a$ или $x > b$, где $a$ и $b$ — вещественные числа, вводимые пользователем.
\end{enumerate}


\section{Оператор \texttt{switch}}

\subsection*{Задачи}

Решите следующие задачи, используя оператор \texttt{switch}. Запрещено использовать стандартные классы для работы с датами — дата задаётся тремя целыми числами: день, месяц, год. Обеспечьте полную проверку корректности ввода (существование даты, високосный год, допустимые диапазоны).

\begin{enumerate}
    \item По введённой дате определите дату следующего дня. Выведите её и проверьте, совпадает ли количество дней в месяце исходной даты с количеством дней в месяце полученной даты.
    \item По введённой дате определите дату предыдущего дня. Выведите её и проверьте, совпадает ли количество дней в месяце исходной даты с количеством дней в месяце полученной даты.
    \item По введённой дате определите дату, которая наступит ровно через месяц (прибавить 1 к месяцу, при необходимости корректируя год). Если в следующем месяце нет дня с таким же числом (например, 31 апреля), то возьмите последний день следующего месяца. Выведите полученную дату и проверьте, является ли она последним днём месяца.
    \item По введённой дате определите дату, которая была ровно месяц назад (вычесть 1 из месяца, при необходимости корректируя год). Если в предыдущем месяце нет дня с таким же числом, возьмите последний день предыдущего месяца. Выведите полученную дату и проверьте, является ли она первым днём месяца.
    \item По введённой дате определите дату, которая наступит через 2 месяца (прибавить 2 к месяцу, корректируя год). Корректировка дня, как в предыдущих задачах. Выведите полученную дату и проверьте, находится ли она в том же квартале года, что и исходная дата. (Кварталы: 1-3, 4-6, 7-9, 10-12)
    \item По введённой дате определите дату, которая была 3 месяца назад. Выведите полученную дату и проверьте, находится ли она в том же году, что и исходная дата.
    \item По введённой дате определите дату, которая наступит через 1 год (прибавить 1 к году). Учтите високосность года для февраля. Если исходная дата - 29 февраля, то в следующем невисокосном году возьмите 28 февраля. Выведите полученную дату и проверьте, является ли она високосным днём (29 февраля).
    \item По введённой дате определите дату, которая была 1 год назад. Выведите полученную дату и проверьте, была ли исходная дата високосным днём (29 февраля), а полученная - нет.
    \item По введённой дате определите дату, которая наступит через 100 дней. Выведите её и проверьте, является ли полученная дата последним днём месяца.
    \item По введённой дате определите дату, которая была 100 дней назад. Выведите её и проверьте, является ли полученная дата первым днём месяца.
    \item По введённой дате определите дату, которая наступит через 1 неделю (7 дней). Выведите её и проверьте, находится ли полученная дата в том же месяце, что и исходная.
    \item По введённой дате определите дату, которая была 1 неделю назад. Выведите её и проверьте, находится ли полученная дата в том же году, что и исходная.
    \item По введённой дате определите дату, которая наступит через 2 месяца. Выведите её и проверьте, является ли день полученной даты последним днём месяца.
    \item По введённой дате определите дату, которая была 2 месяца назад. Выведите её и проверьте, является ли день полученной даты первым днём месяца.
    \item По введённой дате определите дату, которая наступит через 6 месяцев. Выведите её и проверьте, находится ли полученная дата во второй половине года (месяц с июля по декабрь).
    \item По введённой дате определите дату, которая была 6 месяцев назад. Выведите её и проверьте, находится ли полученная дата в первом полугодии (месяц с января по июнь).
    \item По введённой дате определите дату, которая наступит через 1 месяц и 1 день (сначала прибавить месяц, затем день). Корректировка дня, как в задаче 3. Выведите полученную дату и проверьте, является ли она первым днём месяца.
    \item По введённой дате определите дату, которая была 1 месяц и 1 день назад (сначала вычесть месяц, затем день). Выведите полученную дату и проверьте, является ли она последним днём месяца.
    \item По введённой дате определите дату, которая наступит через 2 года. Выведите её и проверьте, является ли год полученной даты високосным.
    \item По введённой дате определите дату, которая была 2 года назад. Выведите её и проверьте, был ли год полученной даты високосным.
    \item По введённой дате определите дату, которая наступит через 1 квартал (3 месяца). Выведите её и проверьте, является ли полученная дата последним днём квартала (31 марта, 30 июня, 30 сентября, 31 декабря).
    \item По введённой дате определите дату, которая была 1 квартал назад. Выведите её и проверьте, является ли полученная дата первым днём квартала (1 января, 1 апреля, 1 июля, 1 октября).
    \item По введённой дате определите дату, которая наступит через 1 год и 1 месяц. Выведите её и проверьте, является ли день полученной даты первым числом месяца.
    \item По введённой дате определите дату, которая была 1 год и 1 месяц назад. Выведите её и проверьте, является ли день полученной даты последним числом месяца.
    \item По введённой дате определите дату, которая наступит через 366 дней (чтобы перепрыгнуть через год). Выведите её и проверьте, является ли полученная дата високосным днём (29 февраля).
\end{enumerate}

\section{Оператор \texttt{do...while}}

\subsection*{Задачи}

Решите следующие задачи, используя цикл \texttt{do...while}. Все задачи предполагают последовательный ввод чисел, оканчивающийся нулём. Нулевое значение является признаком окончания ввода и в вычислениях \textbf{не участвует}. Обеспечьте корректную обработку граничных случаев: пустая последовательность (только 0), отсутствие подходящих чисел, деление на ноль, извлечение корня из отрицательного числа и т.п. При необходимости выводите сообщения об ошибках.

\textbf{Указание}. Для целочисленных операций:
\begin{enumerate}
\item Остаток при делении $a$ на $b$: \verb|a \% b|.
    \item Целая часть частного: \verb|a / b| (при целочисленном делении).
    \item Последняя цифра числа $n$: \verb|n \% 10|.
    \item Предпоследняя цифра: \verb|(n / 10) \% 10|.
\end{enumerate}

\begin{enumerate}
    \item Последовательно вводятся вещественные числа, оканчивающиеся нулём. Выведите максимальное число и количество чисел, больших 5 (кроме завершающего нуля).

    \item Последовательно вводятся целые числа, оканчивающиеся нулём. Выведите минимальное число и количество чисел, у которых последняя цифра равна 0 (кроме завершающего нуля).

    \item Последовательно вводятся вещественные числа, оканчивающиеся нулём. Выведите сумму синусов всех чисел и третье число последовательности (если чисел меньше трёх — вывести сообщение об ошибке).

    \item Последовательно вводятся целые числа, оканчивающиеся нулём. Выведите сумму всех нечётных чисел и количество чисел, делящихся на 3 (кроме завершающего нуля).

    \item Последовательно вводятся целые числа, оканчивающиеся нулём. Выведите количество двузначных натуральных чисел и минимальную последнюю цифру среди всех введённых чисел (кроме завершающего нуля).

    \item Последовательно вводятся натуральные числа, оканчивающиеся нулём. Выведите количество трёхзначных палиндромов (чисел, которые читаются одинаково слева направо и справа налево, например, 121, 343) (кроме завершающего нуля).

    \item Последовательно вводятся целые числа, оканчивающиеся нулём. Выведите сумму всех чисел и предпоследнее число последовательности (если чисел меньше двух — вывести сообщение об ошибке).

    \item Последовательно вводятся целые числа, оканчивающиеся нулём. Выведите произведение всех чисел (кроме завершающего нуля) и второе число последовательности (если чисел меньше двух — вывести сообщение об ошибке).

    \item Последовательно вводятся целые числа, оканчивающиеся нулём. Выведите среднее арифметическое всех чисел и максимум модуля введённых чисел (кроме завершающего нуля).

    \item Последовательно вводятся вещественные числа, оканчивающиеся нулём. Выведите среднее геометрическое всех чисел (кроме завершающего нуля) и минимум модуля введённых чисел. \\
    \textbf{Примечание}: среднее геометрическое определено только для положительных чисел. Если есть неположительные — вывести сообщение об ошибке. \\
    Формула: $\left(a_1 a_2 \dots a_n\right)^{1/n}$.

    \item Последовательно вводятся вещественные числа, оканчивающиеся нулём. Выведите среднее квадратическое всех чисел (кроме завершающего нуля) и минимум квадрата введённых чисел. \\
    Формула: $\sqrt{\frac{a_1^2 + a_2^2 + \dots + a_n^2}{n}}$.

    \item Последовательно вводятся вещественные числа, оканчивающиеся нулём. Выведите среднее гармоническое всех чисел (кроме завершающего нуля) и максимум квадрата введённых чисел. \\
    \textbf{Примечание}: среднее гармоническое не определено, если есть нули или числа разных знаков. Проверяйте знаменатель. \\
    Формула: $\frac{n}{\frac{1}{a_1} + \frac{1}{a_2} + \dots + \frac{1}{a_n}}$.

    \item Последовательно вводятся вещественные числа, оканчивающиеся нулём. Выведите среднее арифметическое модулей всех чисел и максимум синусов введённых чисел (кроме завершающего нуля).

    \item Последовательно вводятся вещественные числа, оканчивающиеся нулём. Выведите среднее гармоническое модулей всех чисел (кроме завершающего нуля) и минимум синусов введённых чисел. \\
    \textbf{Примечание}: модули положительны — среднее гармоническое определено, если только не все числа нулевые.

    \item Последовательно вводятся вещественные числа, оканчивающиеся нулём. Выведите среднее квадратическое модулей всех чисел (кроме завершающего нуля) и минимум косинусов введённых чисел.

    \item Последовательно вводятся вещественные числа, оканчивающиеся нулём. Выведите среднее геометрическое модулей всех чисел (кроме завершающего нуля) и максимум косинусов введённых чисел. \\
    \textbf{Примечание}: модули неотрицательны — если есть ноль, среднее геометрическое = 0.

    \item Последовательно вводятся натуральные числа, оканчивающиеся нулём. Выведите среднее арифметическое квадратов всех чисел (кроме завершающего нуля) и максимальную последнюю цифру среди всех чисел.

    \item Последовательно вводятся натуральные числа, оканчивающиеся нулём. Выведите среднее геометрическое квадратов всех чисел (кроме завершающего нуля) и минимальную последнюю цифру среди всех чисел.

    \item Последовательно вводятся натуральные числа, оканчивающиеся нулём. Выведите среднее квадратическое квадратов всех чисел (кроме завершающего нуля) и максимальную предпоследнюю цифру среди всех чисел.

    \item Последовательно вводятся натуральные числа, оканчивающиеся нулём. Выведите среднее гармоническое квадратов всех чисел (кроме завершающего нуля) и минимальную предпоследнюю цифру среди всех чисел.

    \item Последовательно вводятся натуральные числа от 1 до 999, оканчивающиеся нулём. Выведите максимальную сумму цифр в числах и среднее арифметическое сумм цифр (кроме завершающего нуля).

    \item Последовательно вводятся натуральные числа от 1 до 999, оканчивающиеся нулём. Выведите минимальную сумму цифр в числах и среднее гармоническое сумм цифр (кроме завершающего нуля).

    \item Последовательно вводятся натуральные числа от 1 до 999, оканчивающиеся нулём. Выведите минимальную сумму количества сотен и единиц в числах и среднее геометрическое сумм цифр (кроме завершающего нуля). \\
    \textbf{Пример}: для числа 347: сотни = 3, единицы = 7, сумма = 10.

    \item Последовательно вводятся натуральные числа от 1 до 999, оканчивающиеся нулём. Выведите максимальную сумму количества сотен и единиц в числах и среднее квадратическое сумм цифр (кроме завершающего нуля).

    \item Последовательно вводятся целые числа, оканчивающиеся нулём. Выведите среднее геометрическое всех чётных чисел (кроме завершающего нуля) и максимум среди нечётных чисел. \\
    \textbf{Примечание}: если чётных чисел нет — вывести сообщение об ошибке. Учтите, что среднее геометрическое требует положительных значений.
\end{enumerate}

\section{Цикл \texttt{for}}

\subsection*{Задачи}

Решите следующие задачи, используя цикл \texttt{for}. Все задачи должны использовать именно \texttt{for} (не \texttt{while} или \texttt{do...while}). Обеспечьте корректную обработку граничных случаев: деление на ноль, отрицательные числа, пустые диапазоны и т.п.

\begin{enumerate}
    \item Найдите количество трёхзначных чисел в диапазоне $[100; 999]$, в которых вторая цифра равна сумме первой и третьей цифры.

    \item Найдите количество трёхзначных чисел в диапазоне $[100; 999]$, в которых сумма первых двух цифр равна третьей цифре.

    \item Найдите количество трёхзначных чисел в диапазоне $[100; 999]$, в которых сумма последних двух цифр равна первой цифре.

    \item Найдите все натуральные числа в диапазоне $[m; n]$ ($1 \leqslant m \leqslant n \leqslant 999$), которые равны сумме квадратов своих цифр. \\
    \textbf{Пример}: $1^2 + 3^2 + 0^2 = 10$ — не подходит; $1^2 + 6^2 + 3^2 = 46$ — не подходит.

    \item Найдите все натуральные числа в диапазоне $[m; n]$ ($1 \leqslant m \leqslant n \leqslant 999$), которые равны сумме кубов своих цифр. \\
    \textbf{Пример}: $153 = 1^3 + 5^3 + 3^3$ — подходит.

    \item Найдите все натуральные числа в диапазоне $[m; n]$ ($1 \leqslant m \leqslant n \leqslant 999$), которые равны сумме своих цифр. \\
    \textbf{Пример}: $18 = 1 + 8 = 9$ — не подходит; $1 = 1$ — подходит.

    \item Найдите все натуральные делители числа $n \in \mathbb{N}$ ($n > 0$). Выведите их в порядке возрастания.

    \item Определите, является ли число $n \in \mathbb{N}$ ($n > 1$) простым. Выведите «Да» или «Нет».

    \item Найдите все натуральные числа в диапазоне $[m; n]$ ($1 \leqslant m \leqslant n \leqslant 999$), которые делятся на свою последнюю цифру. \\
    \textbf{Примечание}: если последняя цифра — 0, число не учитывается (деление на ноль).

    \item Напечатайте таблицу перевода двоичных чисел от $1_2$ до $11111_2$ (т.е. от 1 до 31 в десятичной) в десятичную систему счисления.

    \item Напечатайте таблицу перевода восьмеричных чисел от $1_8$ до $777_8$ (т.е. от 1 до 511 в десятичной) в десятичную систему счисления.

    \item Напечатайте таблицу умножения (от 1×1 до 10×10).

    \item Напечатайте первые 20 чисел Фибоначчи ($f_1 = 1$, $f_2 = 1$, $f_{n} = f_{n-1} + f_{n-2}$ для $n > 2$).

    \item Найдите все трёхзначные числа в диапазоне $[100; 999]$, которые при зачёркивании средней цифры уменьшаются в 7 раз. \\
    \textbf{Пример}: число 357 → зачёркиваем 5 → получаем 37; $357 / 37 = 9.648$ — не подходит.

    \item Найдите сумму всех натуральных делителей числа $n \in \mathbb{N}$ ($n > 0$).

    \item Вычислите $a^n$, где $a \in \mathbb{R}$, $n \in \mathbb{Z}$, $n \geqslant 0$. \\
    \textbf{Примечание}: если $n < 0$, вывести сообщение об ошибке. Используйте только умножение (не \verb|Math.pow|).

    \item Найдите сумму всех нечётных натуральных чисел в диапазоне $[m; n]$ ($1 \leqslant m \leqslant n \leqslant 1000$).

    \item Найдите сумму всех чётных натуральных чисел в диапазоне $[m; n]$ ($1 \leqslant m \leqslant n \leqslant 1000$).

    \item Найдите все общие делители натуральных чисел $n$ и $m$ ($n > 0$, $m > 0$). Выведите их в порядке возрастания.

    \item Найдите все натуральные числа в диапазоне $[m; n]$ ($10 \leqslant m \leqslant n \leqslant 999$), которые делятся на свою предпоследнюю цифру. \\
    \textbf{Примечание}: если предпоследняя цифра — 0, число не учитывается.

    \item Напечатайте все трёхзначные палиндромы (числа, которые читаются одинаково слева направо и справа налево, например, 121, 343) в диапазоне $[100; 999]$.

    \item Найдите все трёхзначные числа в диапазоне $[100; 999]$, которые пропорциональны числу, составленному из второй и третьей цифр. \\
    \textbf{Пример}: число 135 → вторая и третья цифры = 35; $135 / 35 = 3.857$ — не целое → не подходит. \\
    \textbf{Уточнение}: пропорциональны = делятся без остатка.

    \item Найдите все четырёхзначные числа в диапазоне $[1000; 9999]$, в которых сумма первых двух цифр равна сумме последних двух цифр.

    \item Найдите все четырёхзначные числа в диапазоне $[1000; 9999]$, в которых сумма крайних цифр равна сумме средних цифр.

    \item Найдите все четырёхзначные числа в диапазоне $[1000; 9999]$, в которых сумма первой и третьей цифр равна сумме второй и четвёртой цифр.
\end{enumerate}

\section{Цикл \texttt{while}}

\subsection*{Задачи}

Решите следующие задачи, используя цикл \texttt{while}. Использование \texttt{for} или \texttt{do...while} не допускается. Все задачи предполагают, что количество итераций заранее неизвестно и определяется в процессе выполнения. Обеспечьте обработку граничных случаев: нули, единицы, отрицательные числа, переполнения.

\begin{enumerate}
    \item Дано натуральное число $n$. Найдите сумму его цифр, используя \texttt{while}.

    \item Дано натуральное число $n$. Найдите количество его цифр, используя \texttt{while}.

    \item Дано натуральное число $n$. Найдите произведение его цифр, используя \texttt{while}.

    \item Дано натуральное число $n$. Определите, является ли оно палиндромом (читается одинаково слева направо и справа налево), используя \texttt{while}. \\
    \textbf{Указание}: постройте зеркальное число и сравните.

    \item Дано натуральное число $n$. Удалите из него все чётные цифры и выведите результат (если получилось пустое число — вывести 0). Используйте \texttt{while}.

    \item Дано натуральное число $n$. Проверьте, является ли оно факториалом какого-либо натурального числа. Если да — выведите это число, иначе — сообщение «Не является факториалом». \\
    \textbf{Пример}: $120 = 5!$ → вывести 5.

    \item Дано натуральное число $n$. Найдите наименьшее $k$, такое что $k! \geqslant n$. Используйте \texttt{while}.

    \item Дано натуральное число $n$. Разложите его на простые множители и выведите их в порядке возрастания (с повторениями). Используйте \texttt{while}.

    \item Даны два натуральных числа $a$ и $b$. Найдите их наибольший общий делитель (НОД) с помощью алгоритма Евклида, используя \texttt{while}.

    \item Даны два натуральных числа $a$ и $b$. Найдите их наименьшее общее кратное (НОК), используя \texttt{while} и НОД.

    \item Дано натуральное число $n$. Переведите его в двоичную систему счисления, используя \texttt{while}. Выведите результат как число (не строку).

    \item Дано натуральное число $n$. Переведите его в восьмеричную систему счисления, используя \texttt{while}. Выведите результат как число.

    \item Дано натуральное число $n$. Определите, сколько раз в нём встречается цифра 7, используя \texttt{while}.

    \item Дано натуральное число $n$. Найдите максимальную цифру в числе, используя \texttt{while}.

    \item Дано натуральное число $n$. Найдите минимальную цифру в числе, используя \texttt{while}.

    \item Дано натуральное число $n$. Определите, содержит ли оно хотя бы одну цифру, равную 0, используя \texttt{while}.

    \item Дано натуральное число $n$. Определите, все ли его цифры нечётные, используя \texttt{while}.

    \item Дано натуральное число $n$. Найдите число, составленное из его цифр в обратном порядке (зеркальное отражение), используя \texttt{while}.

    \item Дано натуральное число $n$. Определите, является ли оно степенью двойки (т.е. $n = 2^k$ для некоторого $k \geqslant 0$), используя \texttt{while}.

    \item Дано натуральное число $n$. Определите, является ли оно степенью тройки, используя \texttt{while}.

    \item Дано натуральное число $n$. Найдите сумму всех его делителей, используя \texttt{while}.

    \item Дано натуральное число $n$. Определите, является ли оно совершенным (т.е. сумма его собственных делителей равна самому числу), используя \texttt{while}.

    \item Дано натуральное число $n$. Найдите количество нулей в его двоичном представлении, используя \texttt{while}.

    \item Дано натуральное число $n$. Найдите количество единиц в его двоичном представлении, используя \texttt{while}.

    \item Дано натуральное число $n$. Определите, можно ли его представить в виде суммы двух квадратов натуральных чисел, используя \texttt{while}. \\
    \textbf{Пример}: $25 = 3^2 + 4^2$ → можно.
\end{enumerate}

\section{Семинар 2}

\section*{Задание 1: Манипуляции со строками с помощью StringBuilder}
  \textbf{Описание:} Напишите программу, которая читает с клавиатуры целое число $n$ ($1 \leq n \leq 1000$), затем $n$ строк (каждая длиной до 100 символов). Используя только StringBuilder для сборки результата, обработайте каждую строку согласно условию варианта (например, инвертируйте, удалите символы и т.д.), затем добавьте её в результат, если она удовлетворяет фильтру варианта (например, длина, наличие символов). Между добавленными строками вставьте фиксированный разделитель "---". В конце добавьте общее количество символов в результате. Выводите на экран. Запрещено использовать String methods вроде reverse, replaceAll, format, join; все операции вручную через циклы и append/insert/delete.

  \begin{enumerate}
    \item Обработка: инвертировать строку; Фильтр: длина > 5
    \item Обработка: удалить все гласные; Фильтр: начинается с согласной
    \item Обработка: удвоить каждый символ; Фильтр: содержит цифру
    \item Обработка: перевести в верхний регистр; Фильтр: заканчивается на 'a'
    \item Обработка: удалить пробелы; Фильтр: длина четная
    \item Обработка: добавить '!' в конец; Фильтр: не содержит 'e'
    \item Обработка: заменить 'a' на '@'; Фильтр: больше 3 гласных
    \item Обработка: удалить дубликаты символов; Фильтр: все символы уникальны
    \item Обработка: отсортировать символы по алфавиту (вручную); Фильтр: длина < 10
    \item Обработка: добавить индекс в начало; Фильтр: содержит специальный символ
    \item Обработка: обернуть в скобки; Фильтр: начинается с цифры
    \item Обработка: удалить последние 2 символа; Фильтр: длина >= 3
    \item Обработка: повторить строку twice; Фильтр: не пустая
    \item Обработка: заменить пробелы на '\_'; Фильтр: содержит пробел
    \item Обработка: удалить все цифры; Фильтр: была хотя бы одна цифра
    \item Обработка: инвертировать регистр; Фильтр: смешанный регистр
    \item Обработка: добавить длину в конец; Фильтр: длина делится на 3
    \item Обработка: удалить гласные в начале; Фильтр: заканчивается гласной
    \item Обработка: удвоить гласные; Фильтр: только гласные
    \item Обработка: заменить согласные на '*'; Фильтр: больше согласных
    \item Обработка: добавить 'prefix-'; Фильтр: не начинается с 'p'
    \item Обработка: удалить середину (если длина >2); Фильтр: длина нечетная
    \item Обработка: циклический сдвиг влево; Фильтр: длина >1
    \item Обработка: циклический сдвиг вправо; Фильтр: содержит 'z'
    \item Обработка: удалить все кроме букв; Фильтр: была не-буква
  \end{enumerate}


\section*{Задание 2: Операции с HashMap без упрощений}
  \textbf{Описание:} Напишите программу, которая читает целое число $m$ ($1 \leq m \leq 500$), затем $m$ пар: строка-ключ (до 50 символов) и целое значение (-1000..1000). Используя HashMap<String, Integer>, сохраните (при дубликатах ключей суммируйте значения). Затем прочитайте $k$ ($1 \leq k \leq 100$) запросов, каждый — строка. Для каждого запроса выполните операцию варианта над значениями, чьи ключи соответствуют условию варианта (например, начинаются с запроса, содержат и т.д.), и выведите результат. Если ничего не найдено, выведите 0. Запрещено использовать streams, Collectors, computeIf; все через циклы по keySet или entrySet, contains, get, put.

  \begin{enumerate}
    \item Операция: сумма значений; Условие: ключ начинается с запроса
    \item Операция: максимум значения; Условие: ключ заканчивается запросом
    \item Операция: количество ключей; Условие: ключ содержит запрос
    \item Операция: минимум значения; Условие: ключ равен запросу (equals)
    \item Операция: среднее значение (int); Условие: длина ключа = длине запроса
    \item Операция: произведение значений; Условие: ключ лексикографически > запрос
    \item Операция: сумма квадратов; Условие: ключ имеет подстроку запрос реверс
    \item Операция: количество положительных; Условие: ключ в нижнем регистре содержит запрос
    \item Операция: максимум по модулю; Условие: ключ без гласных содержит запрос
    \item Операция: сумма только четных; Условие: ключ с цифрами содержит запрос
    \item Операция: количество уникальных значений; Условие: ключ короче запроса
    \item Операция: минимум среди отрицательных; Условие: ключ длиннее запроса
    \item Операция: сумма абсолютных; Условие: ключ стартует с реверса запроса
    \item Операция: произведение нечетных; Условие: ключ в верхнем регистре = запрос
    \item Операция: количество нулей; Условие: ключ содержит запрос дважды
    \item Операция: максимум среди четных; Условие: ключ без пробелов = запрос
    \item Операция: сумма делимых на 3; Условие: ключ с удаленными цифрами содержит запрос
    \item Операция: минимум по квадрату; Условие: ключ инвертированный содержит запрос
    \item Операция: количество > среднего; Условие: ключ с удвоенными символами содержит запрос
    \item Операция: произведение положительных; Условие: ключ без последних 2 символов = запрос
    \item Операция: сумма первых цифр значений; Условие: ключ с префиксом "a" содержит запрос
    \item Операция: максимум разницы с min; Условие: ключ циклически сдвинутый содержит запрос
    \item Операция: количество пар значений; Условие: ключ с заменой 'a' на 'b' содержит запрос
    \item Операция: сумма факториалов (малых); Условие: ключ только буквы содержит запрос
    \item Операция: минимум среди делимых на 5; Условие: ключ с добавленным суффиксом содержит запрос
  \end{enumerate}


\section*{Задание 3: Множества с HashSet и ручными операциями}
  \textbf{Описание:} Напишите программу, которая читает $p$ ($1 \leq p \leq 800$), затем $p$ целых чисел (1..10000). Используя HashSet<Integer>, сохраните уникальные. Затем прочитайте $q$ ($1 \leq q \leq 200$) запросов, каждый — целое число. Для каждого выполните действие варианта: например, если есть, удалите и добавьте трансформацию (квадрат, удвоение и т.д.), если трансформация уже есть, пропустите. В конце выведите элементы в порядке возрастания (сортируйте вручную в массив, без TreeSet или sorted). Запрещено использовать containsAll, addAll; все через add, remove, contains, iterator.

  \begin{enumerate}
    \item Действие: удалить и добавить квадрат
    \item Действие: удалить и добавить удвоенное
    \item Действие: удалить и добавить +1
    \item Действие: удалить и добавить факториал (малый)
    \item Действие: удалить и добавить корень (int)
    \item Действие: удалить и добавить обратное (1/x если !=0)
    \item Действие: удалить и добавить модуль
    \item Действие: удалить и добавить сумму цифр
    \item Действие: удалить и добавить произведение цифр
    \item Действие: удалить и добавить реверс цифр
    \item Действие: удалить и добавить +100
    \item Действие: удалить и добавить -50
    \item Действие: удалить и добавить куб
    \item Действие: удалить и добавить лог2 (int)
    \item Действие: удалить и добавить fib next (простой fib)
    \item Действие: удалить и добавить prime next
    \item Действие: удалить и добавить делимое на 3
    \item Действие: удалить и добавить битовый сдвиг
    \item Действие: удалить и добавить XOR 42
    \item Действие: удалить и добавить AND 255
    \item Действие: удалить и добавить кол-во бит 1
    \item Действие: удалить и добавить pow 2
    \item Действие: удалить и добавить div 2
    \item Действие: удалить и добавить mul 3
    \item Действие: удалить и добавить mod 100
  \end{enumerate}

\section*{Задание 4: Списки с ArrayList и ручными манипуляциями}
  \textbf{Описание:} Напишите программу, которая читает $r$ ($1 \leq r \leq 600$), затем $r$ целых (-5000..5000). Используя ArrayList<Integer>, сохраните. Затем прочитайте $s$ ($1 \leq s \leq 150$) операций, каждая в формате строки (парсите вручную без split упрощений). Операция варианта: например, "add X Y" - добавить X в Y, но с условием; "remove Z" - удалить Z если условие; "swap A B" - swap если разница >k. После операций выведите список по условию варианта (реверс, только четные и т.д.) через цикл, без Collections.reverse/sort. Запрещено использовать subList, sort, reverse; все get/set/add/remove вручную.

  \begin{enumerate}
    \item Операция: add если X >0, в позицию Y mod size
    \item Операция: remove если Z четный
    \item Операция: swap если A+B even
    \item Операция: add X в начало если X odd
    \item Операция: remove последний если >0
    \item Операция: swap первый и последний если size>1
    \item Операция: add X в конец если not contains
    \item Операция: remove по значению если exists
    \item Операция: swap если |A-B|>10
    \item Операция: add если X prime
    \item Операция: remove если divisible 5
    \item Операция: swap random (но fixed seed)
    \item Операция: add в середину
    \item Операция: remove дубликаты (ручной)
    \item Операция: swap соседние
    \item Операция: add сумму соседей
    \item Операция: remove min
    \item Операция: swap max и min
    \item Операция: add среднее
    \item Операция: remove > average
    \item Операция: swap если both positive
    \item Операция: add квадрат last
    \item Операция: remove first negative
    \item Операция: swap every other
    \item Операция: add 0 в позиции multiples 3
  \end{enumerate}


\section{Семинар 3 (простейшее ООП)}

Напишите программу в соответствии с заданием, 
используя объектно-ориентированный стиль программирования. 
В программе должны быть отражены свойства объектно-ориентированного 
программирования: 
инкапсуляция, наследование и полиморфизм (наряду с этим в некоторых
вариантах нужно реализовывать абстрактные классы и методы). 

Обращаем внимание, что каждый класс следует поместить в отдельный файл.

\begin{enumerate}
\item Программа работы со списком работников. Каждый работник определяется фамилией, именем и отчеством, должностью 
(преподаватель и лаборант). Для преподавателя указывается количество часов в год, а для лаборанта -- количество ставок.
Дополнительно в программу вводится стоимость одного часа и стоимость ставки (за год). После ввода необходимо вывести на экран
список работников в порядке возрастания оплаты за год, при этом в списке должны быть указаны ФИО, должность, количество часов/
количество ставок и <<стоимость>> работника.
\item Программа работы со списком учебных заведений (школ и ВУЗов). Школа определяется номером, количеством
учащихся и специализацией (физ-мат, гуманитарный); ВУЗ -- названием, количеством студентов, наличием магистратуры, наличием
аспирантуры. Программа должна предоставлять возможность ввести информацию о ВУЗах и школах, после чего
вывести информацию о школах/ВУЗах в порядке убывания количества учащихся (в независимости от типа учебного заведения).
В списке должна выводиться вся информация, что была введена.
\item Программа суммирования последовательностей двух типов: $\frac{n}{1!}+\frac{n+1}{2!}+\dots+\frac{n+m}{(m+1)!}$ и
$\frac{n}{2^1}+\dots+\frac{n+m}{2^{m+1}}$. $n$ и $m$ вводятся с клавиатуры.
\item Программа нахождения интеграла методом прямоугольников для функций двух видов: $ax^3+bx^2+cx+d$ и $a\sin x+ be^x$.
\item Программа решения уравнений двух видов методом дихотомии: $ax^3+bx^2+cx+d=0$ и $a\sin x+ be^x=c$.
\item Программа <<часы>>. При запуске пользователь выбирает способ вывода времени: часы:минуты или стрелки (второе -- в графическом
режиме).
\item У игрока может быть несколько принадлежностей (до 10): бластеры (с индикатором количества заряда и уровня
бластера от 1 до 5), 
витамины (с количеством оставшихся таблеток), плащи (характеризуются уровнем защиты). Написать программу, которая 
вводит с клавиатуры информацию об имеющихся игровых принадлежностях, после чего выводит информацию на экран.
Данная программа (в части вывода данных) может быть фрагментом игры.
\item Написать программу, отображающую на экране два индикатора: цифровой и в виде полоски нарастающей длины. 
При этом пользователь может выбрать: отображать два цифровых индикатора, один цифровой -- один в виде
полоски или два в виде полоски. Значения на одном индикаторе увеличиваются, на втором -- уменьшаются.
\item Написать программу, в которой пользователь выбирает тип файла, куда записывается информация 
(тестовый или типизированный), после чего вводит последовательность целых чисел, которая записывается
в указанный тип файла.
\item Напишите программу, для работы с бегущими строками: пользователь задает от 1 до 24 строк и выбирает
режим работы каждой строки: слева направо или справа налево (выбор производится для каждой строки по отдельности).
\item В текстовом режиме на экране отображаются квадратик и крестик, с помощью клавиши TAB происходит переключение
между квадратиком и крестиком, каждую фигуру пользователь имеет возможность передвигать с помощью клавиш-стрелок (независимо).
\item Пользователь задает простейший тест, состоящий из вопросов двух видов: с выбором варианта ответа и с вводом верного 
ответа; после чего компьютер тестирует (другого) пользователя по введенному тесту.
\item Создайте программу сортировки массива натуральных чисел, которая сортирует по выбору пользователя: 
1) по возрастания; 2) по убыванию; 3) по возрастанию сумм цифр в числе; 4) по убыванию сумм цифр в числе.
\item Создайте программу вывода всех элементов заданного массива по выбору пользователя: 1) в прямом порядке; 2)
в обратном порядке; 3) в случайном порядке; 4) в челночном порядке (первый-последний-второй-предпоследний и т. д.).
\item Создайте программу суммирования двух чисел, при этом по выбору пользователя либо ввод осуществляется путем выбора
числа с помощью клавиш-стрелок, либо число вводится с клавиатуры.
\item Напишите программу-игру <<чет-нечет>>. Один игрок загадывает <<чет>> или <<нечет>>, а второй угадывает. За один раунд 
идет 10 угадываний. Пользователь выбирает в начала работы программы ее режим работы: пользователь-компьютер,
компьютер-пользователь, компьютер-компьютер или пользователь-пользователь.
\item С клавиатуры задается информация о рисунке, состоящего из нескольких окружностей и прямоугольников со сторонами,
параллельными осям, после этого программа выводит на экран рисунок, сумму площадей и сумму периметров выведенных фигур.
\item Пользователь имеет несколько счетов трех видов: первый вид характеризуется тем, что за его использование с него списывается 1
рубль в месяц, второй -- тем, что количество денег на нем увеличивается на 1\% в месяц,
третий -- тем, что с вероятностью 50\% количество денег на нем за месяц не меняется, с вероятностью 50\% --
увеличивается на 2\%. Пользователь задает список своих счетов с указанием количества денег на них. После чего
программа должна вывести таблицу изменения сумм, размещенных на указанных счетах, в течении года.
\item Пользователь задает информацию о своих контактных сведениях /он может задать один или несколько
телефонов, один или несколько адресов и т. д./: телефон (код города+сам телефон), адрес (город, улица, дом,
корпус, квартира), номер ICQ, e-mail; после чего он может выводить список контактов, изменять информацию
по контактному сведению любого вида, добавлять и удалять контакт.
\item Программа нахождения производной для функций двух видов: $ax^3+bx^2+cx+d$ и $a\sin x+ be^x$.
\item Пользователь размещает на экране несколько рисунков (человечек, колобок, столбик); после чего человечек
подпрыгивает, колобок перекатывается влево-вправо, а столбик стоит на месте.
 \item Программа библиотеки, в которой хранятся книги (описываются автором, названием, количеством страниц) и
CD-диски (название CD, производитель, количество треков). Программа должна позволять добавлять в библиотеку книги и 
CD-диски, а также выводить на экран содержимое библиотеки.
\item Конфигуратор компьютеров. Пользователь выбирает конфигурацию компьютера: процессор (марка, быстродействие), 
один или несколько жестких дисков (марка, емкость), клавиатуру, мышь, принтеры (не обязательно, марка, тип). После
чего ему выводится на экран полная информацию о компьютере (включая стоимость).
\item Формирование заказа в магазине: пользователь выбирает тип товара: рубашка (указывается ее размер), 
ткань (указывается длина и ширина), нитки (выбирается цвет и длина). После чего ему выводится полная информация
о заказе (включая стоимость).
\item Пользователь магазина выбирает конфигурацию велосипеда, который будет собран для него: тип рамы 
(обычная, женская, изогнутая); колеса (размер - 24, 26 или 28); велокомпьютер (может отсутствовать, если есть, то
выбирается беспроводной он или нет); амортизатор (может отсутствовать, если есть, то -- одноподвес или двухподвес). 
После чего ему выводится на экран полная информация о получившемся велосипеде.
\end{enumerate}

\section{Семинар 4 (методы Object)}

Создайте класс в соответствии с вашим вариантом, реализуйте корректно в нём
методы \texttt{clone}, \texttt{equals}, \texttt{hashCode}, \texttt{toString}.

\begin{enumerate}
    \item Класс «Точка в 3D»: поля \texttt{x}, \texttt{y}, \texttt{z} (тип \texttt{double}).
    \item Класс «Цвет в RGB»: поля \texttt{red}, \texttt{green}, \texttt{blue} (тип \texttt{int}, значения от 0 до 255).
    \item Класс «Время суток»: поля \texttt{hour} (0–23), \texttt{minute} (0–59), \texttt{second} (0–59) (все — \texttt{int}).
    \item Класс «Геокоординаты с высотой»: поля \texttt{latitude}, \texttt{longitude} (\texttt{double}), \texttt{altitude} (\texttt{int}, в метрах).
    \item Класс «Дробь с единицей измерения»: поля \texttt{numerator}, \texttt{denominator} (\texttt{int}, знаменатель $\neq 0$), \texttt{unit} (\texttt{String}).
    \item Класс «Комплексное число с меткой»: поля \texttt{re}, \texttt{im} (\texttt{double}), \texttt{label} (\texttt{String}).
    \item Класс «Размер экрана с ориентацией»: поля \texttt{width}, \texttt{height} (\texttt{int}), \texttt{orientation} (\texttt{String}: \texttt{"portrait"} или \texttt{"landscape"}).
    \item Класс «Погодные данные»: поля \texttt{temp} (температура в °C, \texttt{double}), \texttt{humidity} (влажность в \%, \texttt{int}), \texttt{pressure} (давление в мм рт.\,ст., \texttt{int}).
    \item Класс «Скорость в 3D»: поля \texttt{vx}, \texttt{vy}, \texttt{vz} (тип \texttt{double}).
    \item Класс «Пиксель»: поля \texttt{x}, \texttt{y} (\texttt{int}), \texttt{brightness} (\texttt{int}, 0–255).
    \item Класс «Дата»: поля \texttt{day} (1–31), \texttt{month} (1–12), \texttt{year} (\texttt{int}, $\geq 1900$).
    \item Класс «Валютная пара с курсом»: поля \texttt{base}, \texttt{quote} (\texttt{String}), \texttt{rate} (\texttt{double}).
    \item Класс «Углы Эйлера»: поля \texttt{yaw}, \texttt{pitch}, \texttt{roll} (в градусах, \texttt{double}).
    \item Класс «Физическая величина»: поля \texttt{value} (\texttt{double}), \texttt{unit} (\texttt{String}), \texttt{precision} (\texttt{int}, знаков после запятой).
    \item Класс «Настройки звука»: поля \texttt{left}, \texttt{right}, \texttt{master} (\texttt{int}, 0–100).
    \item Класс «Коэффициенты плоскости»: поля \texttt{a}, \texttt{b}, \texttt{c} (\texttt{double}, уравнение $ax + by + cz = 0$).
    \item Класс «Шахматная позиция»: поля \texttt{file} (\texttt{char}, от \texttt{'a'} до \texttt{'h'}), \texttt{rank} (\texttt{int}, 1–8), \texttt{piece} (\texttt{String}, например \texttt{"king"}).
    \item Класс «Интервал с единицей измерения»: поля \texttt{start}, \texttt{end} (\texttt{double}), \texttt{unit} (\texttt{String}).
    \item Класс «Разрешение с частотой»: поля \texttt{width}, \texttt{height} (\texttt{int}), \texttt{refreshRate} (\texttt{int}, в Гц).
    \item Класс «Именованный вектор»: поля \texttt{x}, \texttt{y}, \texttt{z} (\texttt{double}), \texttt{name} (\texttt{String}).
    \item Класс «Позиция с меткой времени»: поля \texttt{x}, \texttt{y} (\texttt{double}), \texttt{timestamp} (\texttt{long}).
    \item Класс «Параметры изображения»: поля \texttt{brightness}, \texttt{contrast}, \texttt{saturation} (\texttt{int}, 0–100).
    \item Класс «Квадратный трёхчлен»: поля \texttt{a}, \texttt{b}, \texttt{c} (\texttt{double}, $a \neq 0$).
    \item Класс «GPS-точка»: поля \texttt{lat}, \texttt{lon} (\texttt{double}), \texttt{accuracy} (\texttt{float}, в метрах).
    \item Класс «Финансовая операция»: поля \texttt{amount} (\texttt{double}), \texttt{currency} (\texttt{String}, например \texttt{"RUB"}), \texttt{date} (\texttt{String} в формате \texttt{"YYYY-MM-DD"}).
\end{enumerate}

\section{Семинар 5 (списки, inner, static классы и Iterator)}

Реализуйте список (самостоятельно, без использования библиотеки) согласно
вашему варианту. В необходимых случаях используйте static и inner-классы. 

\begin{enumerate}
\item Реализуйте класс \texttt{MyLinkedList}, содержащий внутренний класс \texttt{Node} и вложенный итератор. Метод \texttt{add(int value)} добавляет элемент в конец списка. Метод \texttt{remove(int value)} удаляет первое вхождение значения. Метод \texttt{print()} выводит все элементы. Итератор последовательно возвращает все элементы.

\item Реализуйте класс \texttt{MyLinkedList} с методом \texttt{add(int value)}, добавляющим элемент в начало. Метод \texttt{remove(int index)} удаляет элемент по индексу. Метод \texttt{print()} выводит все элементы. Итератор возвращает элементы в порядке хранения.

\item Реализуйте класс \texttt{MyLinkedList}, в котором \texttt{add(int value)} вставляет элемент с сохранением неубывающего порядка. Метод \texttt{remove(int value)} удаляет все вхождения значения. Метод \texttt{print()} выводит все элементы. Итератор возвращает элементы по порядку.

\item Реализуйте класс \texttt{MyLinkedList} с методом \texttt{add(int target, int value)}, добавляющим \texttt{value} после первого вхождения \texttt{target} (если \texttt{target} не найден — в конец). Метод \texttt{remove(int value)} удаляет первое вхождение. Метод \texttt{print()} выводит все элементы. Итератор возвращает все элементы.

\item Реализуйте класс \texttt{MyLinkedList} с методом \texttt{add(int index, int value)}, добавляющим элемент по индексу (с проверкой границ). Метод \texttt{remove(int index)} удаляет элемент по индексу. Метод \texttt{print()} выводит все элементы. Итератор возвращает элементы в прямом порядке.

\item Реализуйте класс \texttt{MyLinkedList}, в котором \texttt{add(int value)} добавляет в конец. Метод \texttt{removeLast()} удаляет последний элемент. Метод \texttt{print()} выводит все элементы. Итератор возвращает все элементы.

\item Реализуйте класс \texttt{MyLinkedList}, в котором \texttt{add(int value)} добавляет только уникальные значения. Метод \texttt{remove(int value)} удаляет все вхождения. Метод \texttt{print()} выводит все элементы. Итератор возвращает элементы без дубликатов (в порядке первого вхождения).

\item Реализуйте класс \texttt{MyLinkedList} с методом \texttt{addAll(int[] values)}, добавляющим все элементы массива в конец. Метод \texttt{remove(int value)} удаляет все вхождения. Метод \texttt{print(int k)} выводит первые $k$ элементов. Итератор возвращает все элементы.

\item Реализуйте класс \texttt{MyLinkedList}, в котором \texttt{add(int value)} вставляет элемент перед первым значением, большим \texttt{value} (иначе — в конец). Метод \texttt{removeAbove(int threshold)} удаляет все элементы > \texttt{threshold}. Метод \texttt{print()} выводит все элементы. Итератор возвращает оставшиеся элементы.

\item Реализуйте класс \texttt{MyLinkedList}, в котором \texttt{add(int value)} добавляет в конец. Метод \texttt{removeFirst()} удаляет первый элемент. Метод \texttt{printEven()} выводит только чётные элементы. Итератор возвращает только чётные значения.

\item Реализуйте класс \texttt{MyLinkedList}, в котором \texttt{add(int value)} добавляет в начало. Метод \texttt{removeBelow(int threshold)} удаляет все элементы < \texttt{threshold}. Метод \texttt{printOdd()} выводит только нечётные элементы. Итератор возвращает только нечётные значения.

\item Реализуйте класс \texttt{MyLinkedList}, в котором \texttt{add(int value)} добавляет в конец. Метод \texttt{removeZeros()} удаляет все нули. Метод \texttt{printPositive()} выводит только положительные элементы. Итератор возвращает только положительные числа.

\item Реализуйте класс \texttt{MyLinkedList} с методом \texttt{add(int index, int value)}. Метод \texttt{removeEvenIndices()} удаляет элементы с чётными индексами. Метод \texttt{printOddIndices()} выводит элементы с нечётными индексами. Итератор возвращает элементы с нечётными индексами.

\item Реализуйте класс \texttt{MyLinkedList}, в котором \texttt{add(int value)} добавляет в начало. Метод \texttt{removeDivisibleBy(int n)} удаляет все элементы, делящиеся на $n$. Метод \texttt{print()} выводит оставшиеся элементы. Итератор возвращает элементы, не делящиеся на $n$.

\item Реализуйте класс \texttt{MyLinkedList}, в котором \texttt{add(int value)} добавляет в конец. Метод \texttt{removeSmall(int minAbs)} удаляет элементы с $|x| < \texttt{minAbs}$. Метод \texttt{printLarge(int minAbs)} выводит элементы с $|x| \geq \texttt{minAbs}$. Итератор возвращает такие элементы.

\item Реализуйте класс \texttt{MyLinkedList} с ограниченной ёмкостью $N$: \texttt{add(int value)} добавляет в конец, при переполнении удаляется первый элемент. Метод \texttt{remove(int value)} удаляет первое вхождение. Метод \texttt{print()} выводит все элементы. Итератор возвращает все элементы в порядке хранения.

\item Реализуйте класс \texttt{MyLinkedList}, в котором \texttt{add(int value)} вставляет с сохранением невозрастающего порядка. Метод \texttt{removeMax()} удаляет первое вхождение максимального элемента. Метод \texttt{print()} выводит все элементы. Итератор возвращает элементы по порядку.

\item Реализуйте класс \texttt{MyLinkedList}, в котором \texttt{add(int value)} добавляет в конец. Метод \texttt{removeDuplicates()} оставляет только первые вхождения. Метод \texttt{printUnique()} выводит уникальные элементы. Итератор возвращает элементы без повторений.

\item Реализуйте класс \texttt{MyLinkedList}, в котором \texttt{add(int value)} добавляет в начало. Метод \texttt{removeNegative()} удаляет все отрицательные элементы. Метод \texttt{printNonNegative()} выводит оставшиеся. Итератор возвращает только неотрицательные числа.

\item Реализуйте класс \texttt{MyLinkedList} с методом \texttt{addAfterZero(int value)}, добавляющим \texttt{value} после каждого нуля. Метод \texttt{removeZeroPairs()} удаляет все пары (0, x). Метод \texttt{print()} выводит результат. Итератор возвращает элементы, не следующие непосредственно за 0.

\item Реализуйте класс \texttt{MyLinkedList}, в котором \texttt{add(int value)} добавляет в конец. Метод \texttt{removeIndexEqualsValue()} удаляет элементы, у которых значение равно их индексу. Метод \texttt{print()} выводит все элементы. Итератор возвращает элементы, у которых значение $\neq$ индекс.

\item Реализуйте класс \texttt{MyLinkedList}, в котором \texttt{add(int value)} добавляет в начало. Метод \texttt{removeAverage()} удаляет все элементы, равные округлённому среднему арифметическому списка. Метод \texttt{print()} выводит все элементы. Итератор возвращает все элементы.

\item Реализуйте класс \texttt{MyLinkedList}, в котором \texttt{add(int value)} добавляет в конец. Метод \texttt{removeLocalMinima()} удаляет локальные минимумы (элементы, меньшие обоих соседей). Метод \texttt{print()} выводит оставшиеся элементы. Итератор возвращает все элементы в исходном порядке.

\item Реализуйте класс \texttt{MyLinkedList}, в котором \texttt{add(int value)} вставляет с сохранением неубывающего порядка. Метод \texttt{removeRange(int a, int b)} удаляет все элементы в диапазоне $[a, b]$. Метод \texttt{print()} выводит оставшиеся элементы. Итератор возвращает элементы вне диапазона.

\item Реализуйте класс \texttt{MyLinkedList}, в котором \texttt{add(int value)} добавляет в конец. Метод \texttt{removeEverySecond()} удаляет каждый второй элемент (индексы 1, 3, 5, …). Метод \texttt{print()} выводит оставшиеся элементы. Итератор возвращает элементы с чётными индексами (0, 2, 4, …).
\end{enumerate}


\end{document}