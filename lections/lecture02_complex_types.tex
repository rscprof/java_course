% !TEX program = xelatex
\documentclass[aspectratio=169]{beamer}
\usetheme{Madrid}
\usecolortheme{beaver}

% ---------- Encoding & language (xelatex-friendly) ----------
\usepackage{polyglossia}
\setdefaultlanguage{russian}
\setotherlanguages{english}

\usepackage{fontspec}

% Основные шрифты (кроссплатформенные)
\setmainfont{DejaVu Serif}
\setsansfont{DejaVu Sans}
\setmonofont{DejaVu Sans Mono}

% Для polyglossia кириллица
\newfontfamily\cyrillicfont{DejaVu Serif}
\newfontfamily\cyrillicfontsf{DejaVu Sans}
\newfontfamily\cyrillicfonttt{DejaVu Sans Mono}

\setmainlanguage{russian}

% ---------- Common packages ----------
\usepackage{graphicx}
\usepackage{booktabs}
\usepackage{tabularx}
\usepackage{hyperref}
\usepackage{multicol}
\usepackage{csquotes}
\usepackage{tikz}
\usetikzlibrary{arrows.meta,positioning,fit,shapes.misc}
\usepackage[dvipsnames]{xcolor}
\usepackage{listings}

% ---------- Listings (Java) ----------
\lstdefinelanguage{JavaLite}{
  language=Java,
  basicstyle=\footnotesize\ttfamily,
  keywordstyle=\color{NavyBlue}\bfseries,
  commentstyle=\color{gray}\ttfamily,
  stringstyle=\color{ForestGreen}\ttfamily
}
\lstset{
  extendedchars=true,
  basicstyle=\footnotesize\ttfamily,
  breaklines=true,
  showstringspaces=false,
  frame=single,
  framerule=0.2pt,
  tabsize=2,
  numbers=left,
  numberstyle=\tiny,
  numbersep=6pt
}

% ---------- Title ----------
\title{Типы данных в Java: от примитивов до коллекций}
\subtitle{Лекция 2: Примитивы, ссылки, строки, StringBuilder, базовые коллекции}
\author{Александр Глускер}
\institute{РУТ МИИТ/ВИШ}
\date{\today}

% ---------- TOC at section start ----------
\AtBeginSection[]{
  \begin{frame}{Навигация по лекции}
    \tableofcontents[currentsection, hideallsubsections]
  \end{frame}
}

\begin{document}

% ================== Title ==================
\begin{frame}
  \titlepage
\end{frame}

% ================== Agenda ==================
\begin{frame}{Содержание}
  \begin{columns}[T,onlytextwidth]
    \begin{column}{0.55\textwidth}
      \begin{enumerate}
        \item Примитивные и ссылочные типы: стек vs куча
        \item Передача аргументов по значению, семантика ссылок
        \item Автоупаковка и обёртки: мост к коллекциям
        \item String: неизменяемость, пул строк, сравнение через equals
        \item StringBuilder: эффективные мутации, внутренний буфер
        \item Коллекции: List, Set, Map — интерфейсы и реализации
        \item Практика: типичные ошибки и анти-паттерны
      \end{enumerate}
    \end{column}
    \begin{column}{0.45\textwidth}
      \begin{block}{Цели}
        \begin{itemize}
          \item Различать хранение в стеке и куче
          \item Понимать семантику передачи ссылок
          \item Использовать автоупаковку осознанно
          \item Избегать ошибок сравнения строк
          \item Выбирать StringBuilder при конкатенации в цикле
          \item Применять коллекции вместо массивов
        \end{itemize}
      \end{block}
    \end{column}
  \end{columns}
\end{frame}

% =====================================================
\section{Примитивные и ссылочные типы: стек vs куча}
% =====================================================

\begin{frame}{Примитивные типы: хранение в стеке}
  \small
  \begin{tabularx}{\linewidth}{l X}
    \toprule
    \textbf{Тип} & \textbf{Размер и диапазон} \\
    \midrule
    \texttt{byte} & 1 байт, -128 до 127 \\
    \texttt{short} & 2 байта, -32768 до 32767 \\
    \texttt{int} & 4 байта, -2³¹ до 2³¹-1 \\
    \texttt{long} & 8 байт, -2⁶³ до 2⁶³-1 \\
    \texttt{float} & 4 байта, IEEE 754 \\
    \texttt{double} & 8 байт, IEEE 754 (по умолчанию) \\
    \texttt{char} & 2 байта, UTF-16 \\
    \texttt{boolean} & не определён явно, обычно 1 байт \\
    \bottomrule
  \end{tabularx}
  \begin{itemize}
    \item Хранятся \textbf{в стеке вызовов (call stack)} — быстро, локально, автоматически освобождаются
    \item Не могут быть \texttt{null}, не являются объектами
    \item Копируются при присваивании и передаче в метод
  \end{itemize}
\end{frame}

\begin{frame}{Ссылочные типы: хранение в куче}
  \begin{itemize}
    \item Все объекты (включая массивы, строки, экземпляры классов) размещаются в \textbf{куче (heap)}
    \item Переменная-ссылка хранится в стеке и содержит \textbf{адрес} объекта в куче
    \item Объекты управляются сборщиком мусора (GC) — освобождаются, когда становятся недостижимы
    \item Ссылка может быть \texttt{null} — не указывает ни на какой объект
  \end{itemize}
  \begin{block}{Пример}
    \begin{itemize}
      \item \texttt{String s = new String("Hello");} — \texttt{s} (в стеке) → объект \texttt{String} (в куче)
      \item \texttt{int[] arr = new int[10];} — \texttt{arr} (в стеке) → массив (в куче)
    \end{itemize}
  \end{block}
\end{frame}

\begin{frame}[fragile]{Глубже: стек вызовов и кадры}
\lstset{language=JavaLite}
\begin{lstlisting}
void example() {
  int x = 10;               // x — в стеке текущего кадра
  String s = "Java";        // s — ссылка в стеке, объект — в куче
  modify(x, s);
}

void modify(int a, String t) {
  // a — копия x (в новом кадре стека)
  // t — копия ссылки s (указывает на тот же объект в куче)
  t = t + "!"; // создаётся НОВЫЙ объект в куче, t теперь ссылается на него
}
\end{lstlisting}
  \begin{block}{Ключевое}
    Java всегда передаёт аргументы \textbf{по значению}:  
    — для примитивов — копия значения,  
    — для ссылок — копия адреса (но объект в куче — один).
  \end{block}
\end{frame}

% =====================================================
\section{Автоупаковка и обёртки: мост к коллекциям}
% =====================================================

\begin{frame}[fragile]{Автоупаковка: преобразование примитив ↔ объект}
\lstset{language=JavaLite}
\begin{lstlisting}
int x = 42;
Integer obj = x;           // автоупаковка: int → Integer (new Integer(x) до Java 9, кэширование после)
int y = obj;               // автораспаковка: Integer → int

// В коллекциях — только объекты:
List<Integer> list = new ArrayList<>();
list.add(100);             // автоупаковка: 100 → Integer.valueOf(100)
int first = list.get(0);   // автораспаковка: Integer → int
\end{lstlisting}
  \begin{block}{Важные детали}
    \begin{itemize}
      \item \texttt{ArrayList<int>} — \textbf{недопустимо} на уровне компиляции
      \item \texttt{Integer} — класс-обёртка, наследник \texttt{Object}
      \item Кэширование значений от -128 до 127 (спецификация JLS)
      \item \texttt{Integer a = 127; Integer b = 127; a == b → true} (из-за кэша), но \texttt{a = 128; b = 128; a == b → false}
    \end{itemize}
  \end{block}
\end{frame}

\begin{frame}[fragile]{Опасности автораспаковки: NullPointerException}
\lstset{language=JavaLite}
\begin{lstlisting}
List<Integer> numbers = new ArrayList<>();
numbers.add(null);         // допустимо — null-ссылка

// Попытка распаковать null:
int value = numbers.get(0); // ← java.lang.NullPointerException!

// Безопасная альтернатива:
Integer boxed = numbers.get(0);
if (boxed != null) {
  int safe = boxed;
}
\end{lstlisting}
  \begin{block}{Рекомендация}
    Избегайте \texttt{null} в коллекциях примитивных обёрток. Используйте \texttt{Optional<Integer>} или явную проверку.
  \end{block}
\end{frame}

% =====================================================
\section{String: неизменяемость, пул строк, equals}
% =====================================================

\begin{frame}[fragile]{String: immutable объект в куче}
\lstset{language=JavaLite}
\begin{lstlisting}
String s1 = "Hello";           // литерал — попадает в пул строк (String Pool)
String s2 = "Hello";           // ссылается на тот же объект в пуле
String s3 = new String("Hello"); // НОВЫЙ объект в куче, вне пула
String s4 = s1.toUpperCase();  // НОВЫЙ объект — "HELLO"

System.out.println(s1 == s2);  // true — один объект в пуле
System.out.println(s1 == s3);  // false — разные объекты
System.out.println(s1.equals(s3)); // true — содержание идентично
\end{lstlisting}
  \begin{block}{Архитектурные следствия}
    \begin{itemize}
      \item Безопасность в многопоточной среде — не требует синхронизации
      \item Экономия памяти за счёт пула строк
      \item Каждая операция возвращает \textbf{новый объект} → дорого при частом изменении
    \end{itemize}
  \end{block}
\end{frame}

\begin{frame}[fragile]{Сравнение: \texttt{==} vs \texttt{.equals()}}
\lstset{language=JavaLite}
\begin{lstlisting}
String input = new Scanner(System.in).nextLine();

// ❌ Опасная ошибка:
if (input == "exit") { ... } // сравнивает адреса, почти всегда false

// ✅ Правильно:
if ("exit".equals(input)) { ... } // сравнивает содержимое, безопасно даже при null

// Альтернатива с защитой от null:
if (input != null && input.equals("exit")) { ... }
\end{lstlisting}
  \begin{block}{Правило}
    \texttt{==} — сравнение \textbf{ссылок} (адресов в памяти),  
    \texttt{.equals()} — сравнение \textbf{логического содержания} (переопределён в \texttt{String}).
  \end{block}
\end{frame}

\begin{frame}[fragile]{Полезные методы String (java.lang.String)}
\lstset{language=JavaLite}
\begin{lstlisting}
String text = "  Java 17 is Great!  ";

text.length();              // 21 — количество символов (char)
text.charAt(2);             // 'v' — символ по индексу
text.substring(2, 6);       // "va 1" — [beginIndex, endIndex)
text.indexOf("17");         // 6 — позиция подстроки
text.replace("Great", "Cool"); // "  Java 17 is Cool!  "
text.trim();                // "Java 17 is Great!" — удаляет пробелы по краям
text.split("\\s+");         // ["Java", "17", "is", "Great!"] — регулярное выражение
text.toLowerCase();         // "  java 17 is great!  "
text.startsWith("  J");     // true
text.endsWith("!  ");       // true
\end{lstlisting}
\end{frame}

% =====================================================
\section{StringBuilder: эффективные мутации, внутренний буфер}
% =====================================================

\begin{frame}[fragile]{StringBuilder: изменяемый аналог String}
\lstset{language=JavaLite}
\begin{lstlisting}
StringBuilder sb = new StringBuilder(); // начальная ёмкость — 16 символов
sb.append("Hello");
sb.append(" ");
sb.append("World");
String result = sb.toString(); // "Hello World" — неизменяемая строка

// Цепочка вызовов:
sb.append("!").append("\n").append("Java").append(21);
\end{lstlisting}
  \begin{block}{Внутреннее устройство}
    \begin{itemize}
      \item Инкапсулирует изменяемый \texttt{char[]} буфер в куче
      \item При переполнении — создаётся новый буфер с увеличенной ёмкостью (x2)
      \item Все операции — \textbf{мутирующие}, не создают промежуточные объекты
    \end{itemize}
  \end{block}
\end{frame}

\begin{frame}[fragile]{Производительность: StringBuilder vs String+}
\lstset{language=JavaLite}
\begin{lstlisting}
// ❌ Анти-паттерн: O(n²) — создаётся n временных строк
String s = "";
for (int i = 0; i < 10000; i++) {
  s = s + i; // каждый + → new String → копирование всего предыдущего
}

// ✅ Оптимально: O(n) — мутирует один буфер
StringBuilder sb = new StringBuilder();
for (int i = 0; i < 10000; i++) {
  sb.append(i);
}
String result = sb.toString();
\end{lstlisting}
  \begin{block}{Рекомендация}
    Используйте \texttt{StringBuilder} при конкатенации \textbf{в цикле} или при сборке строки из многих частей.  
    Для 2–3 конкатенаций — \texttt{+} допустим (компилятор может сам оптимизировать в \texttt{StringBuilder}).
  \end{block}
\end{frame}

% =====================================================
\section{Коллекции: List, Set, Map — интерфейсы и реализации}
% =====================================================

\begin{frame}{Иерархия коллекций: java.util.Collection}
  \begin{itemize}
    \item \texttt{Collection} — корневой интерфейс (не включает \texttt{Map})
    \item \texttt{List} — упорядоченная коллекция, допускает дубликаты, доступ по индексу
    \item \texttt{Set} — неупорядоченная, уникальные элементы, нет индексов
    \item \texttt{Map} — не наследует \texttt{Collection}, хранит пары \texttt{ключ→значение}, ключи уникальны
  \end{itemize}
  \begin{block}{Стандартные реализации}
    \begin{itemize}
      \item \texttt{List} → \texttt{ArrayList} (массив с динамическим ростом)
      \item \texttt{Set} → \texttt{HashSet} (хеш-таблица, O(1) вставка/поиск)
      \item \texttt{Map} → \texttt{HashMap} (хеш-таблица, O(1) операции)
    \end{itemize}
  \end{block}
\end{frame}

\begin{frame}[fragile]{List: ArrayList — динамический массив}
\lstset{language=JavaLite}
\begin{lstlisting}
List<String> names = new ArrayList<>(); // diamond operator (Java 7+)
names.add("Alice");    // [Alice]
names.add("Bob");      // [Alice, Bob]
names.add(0, "Eve");   // [Eve, Alice, Bob] — вставка по индексу
names.set(1, "Carol"); // [Eve, Carol, Bob] — замена
names.remove("Bob");   // [Eve, Carol]

System.out.println(names.get(0));     // "Eve"
System.out.println(names.size());     // 2
System.out.println(names.contains("Eve")); // true

// Итерация — for-each (рекомендуется):
for (String name : names) {
  System.out.println(name);
}
\end{lstlisting}
\end{frame}

\begin{frame}[fragile]{Set: HashSet — уникальность через хеш}
\lstset{language=JavaLite}
\begin{lstlisting}
Set<String> tags = new HashSet<>();
tags.add("java");      // true — добавлено
tags.add("spring");    // true — добавлено
tags.add("java");      // false — дубликат, игнорируется

System.out.println(tags.size());       // 2
System.out.println(tags.contains("java")); // true

// Итерация — порядок не гарантирован!
for (String tag : tags) {
  System.out.println(tag);
}

// Удаление:
tags.remove("spring");
\end{lstlisting}
  \begin{block}{Важно}
    Для корректной работы \texttt{HashSet} объекты должны корректно реализовывать \texttt{equals()} и \texttt{hashCode()} — пока используем \texttt{String}, где это сделано.
  \end{block}
\end{frame}

\begin{frame}[fragile]{Map: HashMap — ассоциативный массив}
\lstset{language=JavaLite}
\begin{lstlisting}
Map<String, Integer> scores = new HashMap<>();
scores.put("Alice", 95);     // добавление
scores.put("Bob", 87);
scores.put("Alice", 98);     // обновление значения по ключу

System.out.println(scores.get("Alice")); // 98
System.out.println(scores.containsKey("Bob")); // true
System.out.println(scores.size());       // 2

// Удаление:
scores.remove("Bob");

// Итерация по ключам:
for (String key : scores.keySet()) {
  System.out.println(key + " → " + scores.get(key));
}

// Итерация по записям (предпочтительно):
for (Map.Entry<String, Integer> entry : scores.entrySet()) {
  System.out.println(entry.getKey() + " → " + entry.getValue());
}
\end{lstlisting}
\end{frame}

\begin{frame}[fragile]{Фабричные методы: неизменяемые коллекции (Java 9+)}
\lstset{language=JavaLite}
\begin{lstlisting}
// Создание маленьких, неизменяемых коллекций — удобно и безопасно
List<String> colors = List.of("red", "green", "blue");
Set<Integer> primes = Set.of(2, 3, 5, 7);
Map<String, String> config = Map.of(
  "host", "localhost",
  "port", "8080"
);

// Попытка модификации → UnsupportedOperationException
colors.add("yellow"); // ← ошибка времени выполнения!

// Для изменяемых коллекций из массива:
List<String> fromArray = new ArrayList<>(Arrays.asList("A", "B"));
\end{lstlisting}
\end{frame}

% =====================================================
\section{Практика: типичные ошибки и анти-паттерны}
% =====================================================

\begin{frame}{Типичные ошибки: диагностика и профилактика}
  \begin{block}{1. NullPointerException при автораспаковке}
    \begin{itemize}
      \item \texttt{List<Integer> list = Arrays.asList(1, null, 3);}
      \item \texttt{int x = list.get(1);} → \texttt{NullPointerException}
      \item \textbf{Решение}: избегайте \texttt{null} в коллекциях, проверяйте перед распаковкой
    \end{itemize}
  \end{block}
  \begin{block}{2. Сравнение строк через ==}
    \begin{itemize}
      \item \texttt{String cmd = scanner.nextLine();}
      \item \texttt{if (cmd == "quit") \{...\}} → почти всегда \texttt{false}
      \item \textbf{Решение}: всегда \texttt{.equals()}, лучше \texttt{"quit".equals(cmd)} для защиты от \texttt{null}
    \end{itemize}
  \end{block}
  \begin{block}{3. Конкатенация строк в цикле}
    \begin{itemize}
      \item \texttt{String result = "";}
      \item \texttt{for (...) \{ result += item; \}} → квадратичная сложность
      \item \textbf{Решение}: \texttt{StringBuilder}
    \end{itemize}
  \end{block}
\end{frame}

\begin{frame}{Что нужно запомнить}
  \begin{itemize}
    \item Примитивы — в стеке, быстрые, не null; ссылки — адреса объектов в куче, могут быть null
    \item Передача аргументов — всегда по значению (копия примитива или копия ссылки)
    \item Автоупаковка — мост к коллекциям, но опасна при null
    \item String — immutable, пул строк, сравнение ТОЛЬКО через \texttt{.equals()}
    \item StringBuilder — mutable, эффективен при множественных изменениях
    \item Коллекции: \texttt{ArrayList} (индексы, дубли), \texttt{HashSet} (уникальность), \texttt{HashMap} (ключи-значения)
    \item Используйте фабричные методы (\texttt{List.of()}) для неизменяемых коллекций
  \end{itemize}
\end{frame}

\end{document}