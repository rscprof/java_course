\documentclass[12pt,a4paper]{article}
\usepackage{fontspec}
\usepackage{polyglossia}
\usepackage{xcolor}
\usepackage{listings}
\usepackage{amsmath}
\usepackage{geometry}
\usepackage{enumitem}
\usepackage{longtable}

% Основные шрифты (кроссплатформенные)
\setmainfont{DejaVu Serif}
\setsansfont{DejaVu Sans}
\setmonofont{DejaVu Sans Mono}

% Для polyglossia кириллица
\newfontfamily\cyrillicfont{DejaVu Serif}
\newfontfamily\cyrillicfontsf{DejaVu Sans}
\newfontfamily\cyrillicfonttt{DejaVu Sans Mono}

\setmainlanguage{russian}

\geometry{margin=2cm}

\lstset{
  language=Java,
  basicstyle=\ttfamily\small,
  keywordstyle=\color{blue},
  commentstyle=\color{green},
  stringstyle=\color{red},
  breaklines=true,
  frame=single
}

\title{Конспект первой пары по JavaFX: Базовые основы FXML и простое приложение}
\author{Подсказка лектору}
\date{Октябрь 2025}

\begin{document}

\maketitle

\section{Общие принципы урока}

\textbf{Цель:} Ввести в основы JavaFX через FXML: настроить проект в IntelliJ IDEA, создать простое FXML-приложение (калькулятор суммы двух чисел), изучить базовые контролы и layouts. Фокус на практике: 30--40\% теории, 60--70\% демо в Scene Builder + IntelliJ.

\textbf{Аудитория:} Новички в JavaFX (знают базовый Java).

\textbf{Формат:} Теория кратко, демонстрации в IntelliJ + Scene Builder. Каждый шаг -- показ на экране с комментариями в FXML и пояснениями. Использовать комментарии XML для ключевых понятий.

\textbf{Распределение времени:} 80 мин.

\section{Введение и настройка проекта (0--10 мин)}

\textbf{0--2 мин: Введение.}\\
FXML -- декларативный XML для UI (как HTML). Позволяет создавать GUI на Java. К концу пары студенты создают FXML-приложение для суммирования чисел. Вербально объяснить и нарисовать диаграмму: Stage $\to$ Scene $\to$ Nodes.

\textbf{2--8 мин: Настройка в IntelliJ.}\\
\begin{enumerate}
    \item File $\to$ New $\to$ Project $\to$ Import (открыть готовый Gradle-проект).
    \item В \texttt{build.gradle} уже добавлена зависимость:
\begin{lstlisting}
dependencies {
    implementation 'org.openjfx:javafx-controls:21'
}
\end{lstlisting}
    \item Скачать Scene Builder и интегрировать: Right-click на FXML $\to$ Open in Scene Builder.
\end{enumerate}

Демо: создать \texttt{sample.fxml} и main.  

\begin{lstlisting}[language=XML]
<?xml version="1.0" encoding="UTF-8"?>
<?import javafx.scene.control.Label?>
<?import javafx.scene.layout.VBox?>
<VBox xmlns="http://javafx.com/javafx/21" xmlns:fx="http://javafx.com/fxml/1" spacing="10" alignment="CENTER">
   <Label text="Hello, FXML!" />
</VBox>
\end{lstlisting}

\begin{lstlisting}[language=Java]
import javafx.application.Application;
import javafx.fxml.FXMLLoader;
import javafx.scene.Parent;
import javafx.scene.Scene;
import javafx.stage.Stage;

public class MainApp extends Application {
    @Override
    public void start(Stage stage) throws Exception {
        Parent root = FXMLLoader.load(getClass().getResource("/sample.fxml"));
        Scene scene = new Scene(root, 300, 200);
        stage.setScene(scene);
        stage.setTitle("Hello FXML");
        stage.show();
    }
}
\end{lstlisting}

\section{База JavaFX: Stage, Scene, граф и FXML (10--30 мин)}

\textbf{10--15 мин: Stage $\to$ Scene $\to$ граф.}\\
JavaFX строится как дерево: Stage содержит Scene, Scene -- root-узел (layout или control). Node -- элемент UI (текст, кнопка), родитель управляет положением и размером детей.

\textbf{15--25 мин: FXML.}\\
FXML -- декларативный XML для UI. Используется в Scene Builder. Плюсы: разделение UI/логики, легкое редактирование. Элементы: \texttt{<?import?>}, \texttt{xmlns}, \texttt{fx:controller}.

\section{3. Основные Layouts в FXML (30--50 мин)}

\textbf{30--40 мин: HBox и VBox.}\\
Контейнеры для размещения контролов:

\begin{itemize}
    \item \texttt{alignment="BOTTOM\_LEFT"} -- выравнивание содержимого (Pos: TOP\_LEFT, CENTER и др.)
    \item \texttt{HBox.hgrow="ALWAYS"} -- растяжка по ширине
    \item \texttt{VBox.vgrow="ALWAYS"} -- растяжка по высоте
    \item \texttt{spacing="10"} -- расстояние между детьми
    \item \texttt{padding="10"} -- внутренние отступы
    \item \texttt{VBox.margin="10"} / \texttt{HBox.margin="10"} -- внешние отступы для детей
\end{itemize}

Пример: 
\texttt{<VBox spacing="10" alignment="CENTER" padding="10"><Label text="Пример"/></VBox>}  

\textbf{Константы Pos для alignment:}

\begin{longtable}{|p{3cm}|p{10cm}|}
\hline
\textbf{Константа} & \textbf{Описание} \\
\hline
BASELINE\_CENTER & Позиционирование по базовой линии вертикально и по центру горизонтально. \\
BASELINE\_LEFT & По базовой линии вертикально и слева горизонтально. \\
BASELINE\_RIGHT & По базовой линии вертикально и справа горизонтально. \\
BOTTOM\_CENTER & Снизу вертикально и по центру горизонтально. \\
BOTTOM\_LEFT & Снизу вертикально и слева горизонтально. \\
BOTTOM\_RIGHT & Снизу вертикально и справа горизонтально. \\
CENTER & По центру вертикально и горизонтально. \\
CENTER\_LEFT & По центру вертикально и слева горизонтально. \\
CENTER\_RIGHT & По центру вертикально и справа горизонтально. \\
TOP\_CENTER & Сверху вертикально и по центру горизонтально. \\
TOP\_LEFT & Сверху вертикально и слева горизонтально. \\
TOP\_RIGHT & Сверху вертикально и справа горизонтально. \\
\hline
\end{longtable}

\textbf{40--50 мин: Pane, GridPane и AnchorPane.}\\
\begin{itemize}
    \item Pane -- базовый, ручное расположение (layoutX/Y)
    \item GridPane -- таблицы (rowIndex, columnIndex, hgap, vgap, padding)
    \item AnchorPane -- якоря к краям (topAnchor, leftAnchor)
\end{itemize}

Примеры в FXML показаны выше.

\section{Базовые Controls в FXML (50--60 мин)}

\textbf{50--55 мин: Label, Button, TextField.}\\
\begin{itemize}
    \item Label: \texttt{<Label fx:id="myLabel" text="Пример" font="Arial 14"/>}
    \item Button: \texttt{<Button fx:id="myButton" text="Клик" onAction="\#handleClick"/>}
    \item TextField: \texttt{<TextField fx:id="myField" promptText="Ввод" prefWidth="100"/>}
\end{itemize}

\textbf{55--60 мин: CheckBox, RadioButton, Tooltip.}\\
\begin{itemize}
    \item CheckBox: \texttt{<CheckBox fx:id="myCheck" text="Согласен"/>}
    \item RadioButton с ToggleGroup:
\begin{lstlisting}[language=Java]
<HBox xmlns:fx="http://javafx.com/fxml/1">
    <children>
        <fx:define>
            <ToggleGroup fx:id="group"/>
        </fx:define>
        <RadioButton text="Да" toggleGroup="$group"/>
        <RadioButton text="Нет" toggleGroup="$group"/>
    </children>
</HBox>
\end{lstlisting}
    \item Tooltip: использовать \texttt{Tooltip.install(myButton, new Tooltip("Текст"));}
\end{itemize}

\section{Events и создание приложения в FXML (60--80 мин)}

\textbf{60--65 мин: Events.}\\
Для Button: \texttt{onAction="\#handleSum"} -- ссылка на метод в контроллере.

\textbf{65--80 мин: Полное демо-приложение.}\\
FXML (\texttt{sample.fxml}):
\begin{lstlisting}[language=XML]
<?xml version="1.0" encoding="UTF-8"?>
<?import javafx.scene.control.*?>
<?import javafx.scene.layout.*?>
<VBox xmlns="http://javafx.com/javafx/21" xmlns:fx="http://javafx.com/fxml/1" fx:controller="com.example.SumController" spacing="10" alignment="CENTER">
   <Label text="Калькулятор суммы"/>
   <HBox spacing="5">
      <TextField fx:id="num1" promptText="Первое число" prefWidth="100"/>
      <Label text="+"/>
      <TextField fx:id="num2" promptText="Второе число" prefWidth="100"/>
   </HBox>
   <Button fx:id="sumButton" text="Сумма" onAction="#handleSum"/>
   <Label fx:id="resultLabel" text="Результат: 0"/>
</VBox>
\end{lstlisting}

Контроллер (\texttt{SumController.java}):
\begin{lstlisting}[language=Java]
import javafx.event.ActionEvent;
import javafx.fxml.FXML;
import javafx.scene.control.Label;
import javafx.scene.control.TextField;

public class SumController {
    @FXML private TextField num1;
    @FXML private TextField num2;
    @FXML private Label resultLabel;

    @FXML
    private void handleSum(ActionEvent event) {
        try {
            double a = Double.parseDouble(num1.getText());
            double b = Double.parseDouble(num2.getText());
            resultLabel.setText("Результат: " + (a + b));
        } catch (NumberFormatException e) {
            resultLabel.setText("Ошибка: введите числа!");
        }
    }
}
\end{lstlisting}

Main -- как выше. Запуск: ввод 5 + 3 $\to$ ``Результат: 8.0''.

\end{document}
