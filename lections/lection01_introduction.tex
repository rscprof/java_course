% !TEX program = xelatex
\documentclass[aspectratio=169]{beamer}
\usetheme{Madrid}
\usecolortheme{beaver}

% ---------- Encoding & language (xelatex-friendly) ----------
\usepackage{polyglossia}
\setdefaultlanguage{russian}
\setotherlanguages{english}

\usepackage{fontspec}
\setmainfont{CMU Serif}
\setsansfont{CMU Sans Serif}
\setmonofont{CMU Typewriter Text}

% ---------- Common packages ----------
\usepackage{graphicx}
\usepackage{booktabs}
\usepackage{tabularx}
\usepackage{hyperref}
\usepackage{multicol}
\usepackage{csquotes}
\usepackage{tikz}
\usetikzlibrary{arrows.meta,positioning,fit,shapes.misc}
\usepackage[dvipsnames]{xcolor}
\usepackage{listings}

% ---------- Listings (Java) ----------
\lstdefinelanguage{JavaLite}{
  language=Java,
  basicstyle=\footnotesize\ttfamily,
  keywordstyle=\color{NavyBlue}\bfseries,
  commentstyle=\color{gray}\ttfamily,
  stringstyle=\color{ForestGreen}\ttfamily
}
\lstset{
  extendedchars=true,
  basicstyle=\footnotesize\ttfamily,
  breaklines=true,
  showstringspaces=false,
  frame=single,
  framerule=0.2pt,
  tabsize=2,
  numbers=left,
  numberstyle=\tiny,
  numbersep=6pt
}

% ---------- Title ----------
\title{Введение в программирование на Java}
\subtitle{Лекция 1: Обзор языков высокого уровня, синтаксис Java и инструменты (IntelliJ IDEA)}
\author{Александр Глускер}
\institute{РУТ МИИТ/ВИШ}
\date{\today}

% ---------- TOC at section start ----------
\AtBeginSection[]{
  \begin{frame}{Навигация по лекции}
    \tableofcontents[currentsection, hideallsubsections]
  \end{frame}
}

\begin{document}

% ================== Title ==================
\begin{frame}
  \titlepage
\end{frame}

% ================== Agenda ==================
\begin{frame}{Содержание}
  \begin{columns}[T,onlytextwidth]
    \begin{column}{0.55\textwidth}
      \begin{enumerate}
        \item Обзор языков высокого уровня
        \item Java сегодня: роль языка и JVM
        \item Инструменты: JDK, IntelliJ IDEA, CLI, JShell
        \item Синтаксис Java: базовые конструкции
        \item Ввод/вывод и обработка ошибок
        \item Массивы и методы (без ООП)
        \item Итоги и мини-практика
      \end{enumerate}
    \end{column}
    \begin{column}{0.45\textwidth}
      \begin{block}{Цели}
        \begin{itemize}
          \item Понять место Java и философию JVM
          \item Установить JDK и настроить IntelliJ IDEA
          \item Научиться писать и запускать простые программы
          \item Подготовить базу к занятию по ООП
        \end{itemize}
      \end{block}
    \end{column}
  \end{columns}
\end{frame}

% =====================================================
\section{Обзор языков высокого уровня}
% =====================================================

\begin{frame}{Высокоуровневые языки: карта местности}
  \begin{itemize}
    \item \textbf{Типизация:} статическая (Java, C\#), динамическая (Python, JS)
    \item \textbf{Компиляция:} в машинный код (Go, Rust, Kotlin), 
    в байткод под VM (Java, Kotlin, Scala), JIT/интерпретация (Python/JS)
    \item \textbf{Управление памятью:} сборка мусора (Java, C\#), 
    владение/заимствование (Rust), ручная (C)
    \item \textbf{Экосистемы:} JVM (Java/Kotlin/Scala), 
    .NET, нативные (C/C++/Rust), веб (JS/TS)
  \end{itemize}
  \begin{block}{Почему нам Java}
    \begin{itemize}
      \item Сильная статическая типизация, крупная экосистема, стабильные LTS-релизы
      \item Универсальность: бэкенд (но...), инструменты  (но...), Android (исторически), 
      desktop/CLI  (но...)
    \end{itemize}
  \end{block}
\end{frame}

% =====================================================
\section{Java сегодня: роль языка и JVM}
% =====================================================

\begin{frame}{Java в 2025: трезвый взгляд}
  \begin{itemize}
    \item \textbf{Сервер/Enterprise:} де-факто стандарт (Spring, Jakarta EE)
    \item \textbf{Android:} \underline{Kotlin-first} для нового кода; 
    Java остаётся в существующих проектах и библиотеках
    \item \textbf{JVM-платформа:} множество языков (Kotlin, Scala и др.) 
    на одном рантайме
  \end{itemize}
  \begin{block}{Версии языка}
    \begin{itemize}
      \item LTS: 17 (2021), 21 (2023), \emph{25 (план — сентябрь 2025)}
    \end{itemize}
  \end{block}
  \begin{block}{Философия JVM}
    \begin{itemize}
      \item \emph{Write once, run anywhere:} компиляция в байткод, исполнение на JVM
      \item HotSpot JIT оптимизирует «на лету», есть современные GC
      \item OpenJDK — \emph{референсная реализация} Java SE
    \end{itemize}
  \end{block}
\end{frame}

\begin{frame}{Экосистема Java кратко}
  \small
  \begin{tabularx}{\linewidth}{l X}
    \toprule
    \textbf{Компонент} & \textbf{Что это} \\
    \midrule
    JDK & Компилятор \texttt{javac}, рантайм, инструменты (\texttt{jar}, 
    \texttt{javadoc}, \texttt{jshell}) \\
    JVM (HotSpot) & Виртуальная машина: загрузка классов, JIT, GC \\
    OpenJDK & Открытая референсная реализация Java SE \\
    LTS & Версии с длительной поддержкой (17/21/25) \\
    Билд-системы & Maven/Gradle (зависимости, сборка, тесты) \\
    IDE & IntelliJ IDEA (Community/Ultimate) \\
    \bottomrule
  \end{tabularx}
\end{frame}

% =====================================================
\section{Инструменты: JDK, IntelliJ IDEA, CLI, JShell}
% =====================================================

\begin{frame}{Установка и настройка}
  \begin{enumerate}
    \item Установите \textbf{IntelliJ IDEA Community} (хватает для чистой Java, Maven/Gradle)
    \item В IDEA: \texttt{New Project} $\rightarrow$ \texttt{Java} $\rightarrow$, укажите название проекта (English), остальное по умолчанию
    \item Запуск: зелёный треугольник слева от \texttt{main}
  \end{enumerate}
\end{frame}

\begin{frame}[fragile]{JShell: REPL для быстрых экспериментов}
\lstset{language=JavaLite}
\begin{lstlisting}
$ jshell
jshell> int x = 2 + 2;
x ==> 4
jshell> "Hello, " + "Java"
$2 ==> "Hello, Java"
\end{lstlisting}
  \begin{itemize}
    \item Полезно для проверки выражений, без создания проекта
  \end{itemize}
\end{frame}

% =====================================================
\section{Синтаксис Java: база до ООП}
% =====================================================

\begin{frame}[fragile]{Структура минимальной программы}
\lstset{language=JavaLite}
\begin{lstlisting}
// Файл Main.java
public class Main {
  public static void main(String[] args) {
    System.out.println("Hello, Java!");
  }
}
\end{lstlisting}
\begin{itemize}
  \item Пока игнорируем ООП: мыслите о \texttt{class Main} как о контейнере для \texttt{main}
\end{itemize}
\end{frame}

\begin{frame}[fragile]{Комментарии, идентификаторы, литералы}
\lstset{language=JavaLite}
\begin{lstlisting}
// Однострочный комментарий
/* Многострочный */
/**
 * Док-комментарий
 */
int answer = 42;
double pi = 3.1415;
char letter = 'A';
String s = "Строка";
\end{lstlisting}
\end{frame}

\begin{frame}[fragile]{Типы и переменные}
\lstset{language=JavaLite}
\begin{lstlisting}
int i = 10; long L = 10L;
double d = 2.5; float f = 2.5f;
boolean ok = true;
char c = 'Ж';
String text = "Привет";
var sum = i + 5; // локальное выведение типа (Java 10+)
\end{lstlisting}
\begin{block}{Замечания}
  \begin{itemize}
    \item \texttt{var} — только для локальных переменных; тип всё равно статический
    \item Преобразования: явные касты, \texttt{(int) d}, и методы вроде \texttt{Integer.parseInt}
  \end{itemize}
\end{block}
\end{frame}

\begin{frame}[fragile]{Операторы}
\lstset{language=JavaLite}
\begin{lstlisting}
int a = 7, b = 3;
int sum = a + b;     // + - * / %
int mod = a % b;
boolean gt = a > b;  // > < >= <= == !=
boolean both = (a > 0) && (b > 0); // && || !
a++; --b; // инкремент/декремент
\end{lstlisting}
\end{frame}

\begin{frame}[fragile]{Условия: if/else и тернарный оператор}
\lstset{language=JavaLite}
\begin{lstlisting}
int x = -3;
if (x >= 0) {
  System.out.println("Неотрицательное");
} else {
  System.out.println("Отрицательное");
}
String sign = (x >= 0) ? "non-negative" : "negative";
\end{lstlisting}
\end{frame}

\begin{frame}[fragile]{switch: классический и современный (Java 14+)}
\lstset{language=JavaLite}
\begin{lstlisting}
// Классический
int day = 2;
switch (day) {
  case 1: System.out.println("Пн"); break;
  case 2: System.out.println("Вт"); break;
  default: System.out.println("Другое");
}

// Выражение switch (Java 14+)
int n = 3;
String name = switch (n) {
  case 1 -> "Один";
  case 2, 3 -> "Два или три";
  default -> "Иное";
};
\end{lstlisting}
\end{frame}

\begin{frame}[fragile]{Циклы: while, do-while, for, for-each}
\lstset{language=JavaLite}
\begin{lstlisting}
int i = 0;
while (i < 3) { System.out.println(i); i++; }

int j = 0;
do { System.out.println(j); j++; } while (j < 3);

for (int k = 0; k < 3; k++) { System.out.println(k); }

int[] arr = {1, 2, 3};
for (int v : arr) { System.out.println(v); } // for-each
\end{lstlisting}
\end{frame}

% =====================================================
\section{Ввод/вывод и обработка ошибок}
% =====================================================

\begin{frame}[fragile]{Стандартный ввод: Scanner}
\lstset{language=JavaLite}
\begin{lstlisting}
import java.util.*;

public class Main {
  public static void main(String[] args) {
    try (Scanner sc = new Scanner(System.in)) {
      System.out.print("Введите целое число: ");
      String line = sc.nextLine();        // читаем строку полностью
      int x = Integer.parseInt(line.trim()); // парсим явно
      System.out.println("Квадрат: " + Math.multiplyExact(x, x));
    } catch (NumberFormatException e) {
      System.out.println("Ошибка: введите целое число");
    } catch (ArithmeticException e) {
      System.out.println("Переполнение при вычислениях");
    }
  }
}
\end{lstlisting}
\begin{block}{Почему так}
  \begin{itemize}
    \item Явный разбор строки избегает «ловушки» \texttt{nextInt()} + перевод строки
    \item \texttt{Math.multiplyExact/addExact} корректно детектирует переполнение
    \item С JDK 18 по умолчанию UTF-8 (файлы/строки) — меньше сюрпризов при I/O
  \end{itemize}
\end{block}
\end{frame}

\begin{frame}[fragile]{Файлы: чтение/запись в одну строку (NIO)}
\lstset{language=JavaLite}
\begin{lstlisting}
import java.nio.file.*; import java.io.IOException;
import java.util.*;

public class Main {
  public static void main(String[] args) throws IOException {
    Path p = Path.of("data.txt");
    Files.writeString(p, "Hello\n", java.nio.charset.StandardCharsets.UTF_8);
    String content = Files.readString(p);
    System.out.println(content);
  }
}
\end{lstlisting}
\end{frame}

% =====================================================
\section{Массивы и методы (минимум до ООП)}
% =====================================================

\begin{frame}[fragile]{Массивы: создание и базовые операции}
\lstset{language=JavaLite}
\begin{lstlisting}
int[] a = {3, 1, 4};
int[] b = new int[5];        // заполнен нулями
System.out.println(a.length);
Arrays.sort(a);              // [1,3,4]
int pos = Arrays.binarySearch(a, 3); // позиция 3
\end{lstlisting}
\end{frame}

\begin{frame}[fragile]{Методы (статические функции)}
\lstset{language=JavaLite}
\begin{lstlisting}
public class Main {
  static int sum(int x, int y) {
    return Math.addExact(x, y);
  }
  public static void main(String[] args) {
    System.out.println(sum(2, 3));
  }
}
\end{lstlisting}
\begin{block}{Важно}
  \begin{itemize}
    \item Пока обходимся \texttt{static} методами; принципы ООП — на следующем занятии
    \item Перегрузка методов разрешена (одинаковое имя, разные параметры)
  \end{itemize}
\end{block}
\end{frame}

% =====================================================
\section{Итоги}
% =====================================================

\begin{frame}{Что нужно запомнить}
  \begin{itemize}
    \item Java = язык + платформа JVM; OpenJDK — референсная реализация
    \item Актуальные LTS: 17, 21, (план) 25; для курса — JDK 21
    \item IntelliJ IDEA Community достаточно для старта; JShell для экспресс-проверок
    \item База синтаксиса: типы, операторы, if/switch, циклы, массивы, методы
    \item Ввод/вывод: \texttt{Scanner} + явный парсинг; \texttt{Files.readString/writeString}
  \end{itemize}
\end{frame}


\end{document}
