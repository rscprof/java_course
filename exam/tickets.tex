\documentclass[a4paper]{article}
\usepackage{xltxtra}
\usepackage{amsmath}
\usepackage{amssymb}
\usepackage{hyperref}
\usepackage{polyglossia}
\setmainlanguage{russian}

%\setkeys{russian}{babelshorthands=true}

\setmainfont{Times New Roman}
\setromanfont{Times New Roman} 
\setsansfont{Arial} 
\setmonofont{Courier New} 

\newfontfamily{\cyrillicfont}{Times New Roman} 
\newfontfamily{\cyrillicfontrm}{Times New Roman}
\newfontfamily{\cyrillicfonttt}{Courier New}
\newfontfamily{\cyrillicfontsf}{Arial}

\usepackage{array}
\usepackage{verbatim}
%\usepackage[utf8]{inputenc} % Кодировка
%\usepackage[russian]{babel} % Поддержка русского языка
%\usepackage{fontspec} % Для указания шрифтов (только XeLaTeX или LuaLaTeX)
%\setmainfont{Times New Roman} % Установка Times New Roman как основного шрифта
\usepackage{geometry} % Настройка полей страницы
\geometry{top=2cm, bottom=2cm, left=2.5cm, right=2.5cm}
%\renewcommand{\baselinestretch}{1.5} % Межстрочный интервал
\pagestyle{empty}
\begin{document}

\section*{Список вопросов по курсу <<Программирование на Java>>}

В ходе практической части экзамена будет доступна документация по Java (с помощью zeal), 
по Spring Boot (с помощью zeal), 
JUnit (с помощью пользовательского контента для zeal: 
\url
\url{https://github.com/Kapeli/Dash-User-Contributions/tree/master/docsets/JUnit5},
JavaFX (самостоятельно будет импортировано преподавателем)).

Кроме того, практическое задание можно выполнять с собственноручным конспектом.

Для Spring проектов будет предоставлен результат запуска \url{https://start.spring.io/}

@content@

\end{document}
