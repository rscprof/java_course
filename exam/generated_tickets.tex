\documentclass[a4paper]{article}
\usepackage{xltxtra}
\usepackage{amsmath}
\usepackage{amssymb}
\usepackage{hyperref}
\usepackage{polyglossia}
\setmainlanguage{russian}

%\setkeys{russian}{babelshorthands=true}

\setmainfont{Times New Roman}
\setromanfont{Times New Roman} 
\setsansfont{Arial} 
\setmonofont{Courier New} 

\newfontfamily{\cyrillicfont}{Times New Roman} 
\newfontfamily{\cyrillicfontrm}{Times New Roman}
\newfontfamily{\cyrillicfonttt}{Courier New}
\newfontfamily{\cyrillicfontsf}{Arial}

\usepackage{array}
\usepackage{verbatim}
%\usepackage[utf8]{inputenc} % Кодировка
%\usepackage[russian]{babel} % Поддержка русского языка
%\usepackage{fontspec} % Для указания шрифтов (только XeLaTeX или LuaLaTeX)
%\setmainfont{Times New Roman} % Установка Times New Roman как основного шрифта
\usepackage{geometry} % Настройка полей страницы
\geometry{top=2cm, bottom=2cm, left=2.5cm, right=2.5cm}
%\renewcommand{\baselinestretch}{1.5} % Межстрочный интервал
\pagestyle{empty}
\begin{document}


\begin{tabular}{!{\vrule width 2pt}p{5cm}|p{6cm}|p{4cm}!{\vrule width 2pt}}
    \noalign{\hrule height 2pt}

    {\centering 
    \fontsize{14pt}{16pt}\selectfont
    РУТ(МИИТ)

\vspace{14pt}

Академия «Высшая инженерная школа»

\vspace{14pt}

2025/2026 учебный год

    }
&
{
    \centering
\fontsize{14pt}{16pt}\selectfont

\textbf{ЭКЗАМЕНАЦИОННЫЙ
БИЛЕТ №1}


по дисциплине 

«Программирование на Java» 
\fontsize{12pt}{14pt}\selectfont
для студентов образовательной программы «Цифровая инженерия транспортных процессов»

}
&
{
\centering
\fontsize{14pt}{16pt}\selectfont

УТВЕРЖДАЮ
Руководитель образовательной программы

\vspace{1cm}

\fontsize{12pt}{14pt}\selectfont
\underline{\hspace{3cm}}

к.т.н., \underline{Проневич О.Б.}

}
\\
\hline
\multicolumn{3}{!{\vrule width 2pt}p{16cm}!{\vrule width 2pt}}{
\begin{minipage}{16cm}
    \vspace{0.2cm}

\fontsize{14pt}{16pt}\selectfont\itshape
\begin{enumerate}
    \item Java: отличия List и Set, принципы работы с множествами
    \item Java: синхронизация потоков, ключевое слово synchronized
    \item Напишите JavaFX программу, которая по введённой дате определяет дату, 
которая наступит через месяц. Если в следующем месяце 
нет дня с таким же числом, возьмите последний день следующего месяца. 
Запрещено использовать стандартные классы для работы с датами.
Проверяйте корректность входных данных. Разработайте UNIT-тесты для модели. 
\end{enumerate}

\vspace{0.2cm}
    
\end{minipage}
}
\\
\noalign{\hrule height 2pt}
\end{tabular}

\newpage


\begin{tabular}{!{\vrule width 2pt}p{5cm}|p{6cm}|p{4cm}!{\vrule width 2pt}}
    \noalign{\hrule height 2pt}

    {\centering 
    \fontsize{14pt}{16pt}\selectfont
    РУТ(МИИТ)

\vspace{14pt}

Академия «Высшая инженерная школа»

\vspace{14pt}

2025/2026 учебный год

    }
&
{
    \centering
\fontsize{14pt}{16pt}\selectfont

\textbf{ЭКЗАМЕНАЦИОННЫЙ
БИЛЕТ №2}


по дисциплине 

«Программирование на Java» 
\fontsize{12pt}{14pt}\selectfont
для студентов образовательной программы «Цифровая инженерия транспортных процессов»

}
&
{
\centering
\fontsize{14pt}{16pt}\selectfont

УТВЕРЖДАЮ
Руководитель образовательной программы

\vspace{1cm}

\fontsize{12pt}{14pt}\selectfont
\underline{\hspace{3cm}}

к.т.н., \underline{Проневич О.Б.}

}
\\
\hline
\multicolumn{3}{!{\vrule width 2pt}p{16cm}!{\vrule width 2pt}}{
\begin{minipage}{16cm}
    \vspace{0.2cm}

\fontsize{14pt}{16pt}\selectfont\itshape
\begin{enumerate}
    \item Java: основные методы класса String
    \item Java: взаимная блокировка (deadlock), причины и методы предотвращения
    \item Реализуйте программу с графическим интерфейсом JavaFX, которая читает файл 
с описанием Label'ов и интервалов времени. После загрузки на форме 
появляются Label'ы с начальными значениями. Для каждого Label 
создаётся поток, который с указанным интервалом обновляет значение (увеличивает на 1). 
\end{enumerate}

\vspace{0.2cm}
    
\end{minipage}
}
\\
\noalign{\hrule height 2pt}
\end{tabular}

\newpage


\begin{tabular}{!{\vrule width 2pt}p{5cm}|p{6cm}|p{4cm}!{\vrule width 2pt}}
    \noalign{\hrule height 2pt}

    {\centering 
    \fontsize{14pt}{16pt}\selectfont
    РУТ(МИИТ)

\vspace{14pt}

Академия «Высшая инженерная школа»

\vspace{14pt}

2025/2026 учебный год

    }
&
{
    \centering
\fontsize{14pt}{16pt}\selectfont

\textbf{ЭКЗАМЕНАЦИОННЫЙ
БИЛЕТ №3}


по дисциплине 

«Программирование на Java» 
\fontsize{12pt}{14pt}\selectfont
для студентов образовательной программы «Цифровая инженерия транспортных процессов»

}
&
{
\centering
\fontsize{14pt}{16pt}\selectfont

УТВЕРЖДАЮ
Руководитель образовательной программы

\vspace{1cm}

\fontsize{12pt}{14pt}\selectfont
\underline{\hspace{3cm}}

к.т.н., \underline{Проневич О.Б.}

}
\\
\hline
\multicolumn{3}{!{\vrule width 2pt}p{16cm}!{\vrule width 2pt}}{
\begin{minipage}{16cm}
    \vspace{0.2cm}

\fontsize{14pt}{16pt}\selectfont\itshape
\begin{enumerate}
    \item Java: абстрактные классы и интерфейсы, отличия и применение
    \item JavaFX: элементы управления  - TextField, Label, Button, CheckBox
    \item Напишите программу, которая читает целое число $m$ ($1 \leq m \leq 100$), затем $m$ пар: 
строка-ключ и целое значение. Используя HashMap<String, Integer>, сохраните данные. 
Затем для каждого запроса выведите количество ключей, которые содержат 
запрос как подстроку. 
\end{enumerate}

\vspace{0.2cm}
    
\end{minipage}
}
\\
\noalign{\hrule height 2pt}
\end{tabular}

\newpage


\begin{tabular}{!{\vrule width 2pt}p{5cm}|p{6cm}|p{4cm}!{\vrule width 2pt}}
    \noalign{\hrule height 2pt}

    {\centering 
    \fontsize{14pt}{16pt}\selectfont
    РУТ(МИИТ)

\vspace{14pt}

Академия «Высшая инженерная школа»

\vspace{14pt}

2025/2026 учебный год

    }
&
{
    \centering
\fontsize{14pt}{16pt}\selectfont

\textbf{ЭКЗАМЕНАЦИОННЫЙ
БИЛЕТ №4}


по дисциплине 

«Программирование на Java» 
\fontsize{12pt}{14pt}\selectfont
для студентов образовательной программы «Цифровая инженерия транспортных процессов»

}
&
{
\centering
\fontsize{14pt}{16pt}\selectfont

УТВЕРЖДАЮ
Руководитель образовательной программы

\vspace{1cm}

\fontsize{12pt}{14pt}\selectfont
\underline{\hspace{3cm}}

к.т.н., \underline{Проневич О.Б.}

}
\\
\hline
\multicolumn{3}{!{\vrule width 2pt}p{16cm}!{\vrule width 2pt}}{
\begin{minipage}{16cm}
    \vspace{0.2cm}

\fontsize{14pt}{16pt}\selectfont\itshape
\begin{enumerate}
    \item Java: архитектура выполнения - JVM, JDK, компиляция и интерпретация
    \item JavaFX: компоновка - HBox, VBox, GridPane, AnchorPane, Pane, Canvas, TabPane
    \item Напишите программу, которая читает $r$ целых чисел ($1 \leq r \leq 100$). 
Используя ArrayList<Integer>, сохраните их. Затем обработайте операции: 
"add X to begin" - добавить число X в начало списка, только если X нечётное. 
После всех операций выведите список в обратном порядке. 
\end{enumerate}

\vspace{0.2cm}
    
\end{minipage}
}
\\
\noalign{\hrule height 2pt}
\end{tabular}

\newpage


\begin{tabular}{!{\vrule width 2pt}p{5cm}|p{6cm}|p{4cm}!{\vrule width 2pt}}
    \noalign{\hrule height 2pt}

    {\centering 
    \fontsize{14pt}{16pt}\selectfont
    РУТ(МИИТ)

\vspace{14pt}

Академия «Высшая инженерная школа»

\vspace{14pt}

2025/2026 учебный год

    }
&
{
    \centering
\fontsize{14pt}{16pt}\selectfont

\textbf{ЭКЗАМЕНАЦИОННЫЙ
БИЛЕТ №5}


по дисциплине 

«Программирование на Java» 
\fontsize{12pt}{14pt}\selectfont
для студентов образовательной программы «Цифровая инженерия транспортных процессов»

}
&
{
\centering
\fontsize{14pt}{16pt}\selectfont

УТВЕРЖДАЮ
Руководитель образовательной программы

\vspace{1cm}

\fontsize{12pt}{14pt}\selectfont
\underline{\hspace{3cm}}

к.т.н., \underline{Проневич О.Б.}

}
\\
\hline
\multicolumn{3}{!{\vrule width 2pt}p{16cm}!{\vrule width 2pt}}{
\begin{minipage}{16cm}
    \vspace{0.2cm}

\fontsize{14pt}{16pt}\selectfont\itshape
\begin{enumerate}
    \item Java: примитивные и ссылочные типы данных, автоупаковка и распаковка
    \item JDBC: подключение SQLite
    \item Напишите программу, которая читает целое число $m$ ($1 \leq m \leq 100$), затем $m$ пар: 
строка-ключ и целое значение. Используя HashMap<String, Integer>, сохраните данные. 
Затем для каждого запроса выведите минимальное значение среди тех,
 чьи ключи точно равны запросу. 
\end{enumerate}

\vspace{0.2cm}
    
\end{minipage}
}
\\
\noalign{\hrule height 2pt}
\end{tabular}

\newpage


\begin{tabular}{!{\vrule width 2pt}p{5cm}|p{6cm}|p{4cm}!{\vrule width 2pt}}
    \noalign{\hrule height 2pt}

    {\centering 
    \fontsize{14pt}{16pt}\selectfont
    РУТ(МИИТ)

\vspace{14pt}

Академия «Высшая инженерная школа»

\vspace{14pt}

2025/2026 учебный год

    }
&
{
    \centering
\fontsize{14pt}{16pt}\selectfont

\textbf{ЭКЗАМЕНАЦИОННЫЙ
БИЛЕТ №6}


по дисциплине 

«Программирование на Java» 
\fontsize{12pt}{14pt}\selectfont
для студентов образовательной программы «Цифровая инженерия транспортных процессов»

}
&
{
\centering
\fontsize{14pt}{16pt}\selectfont

УТВЕРЖДАЮ
Руководитель образовательной программы

\vspace{1cm}

\fontsize{12pt}{14pt}\selectfont
\underline{\hspace{3cm}}

к.т.н., \underline{Проневич О.Б.}

}
\\
\hline
\multicolumn{3}{!{\vrule width 2pt}p{16cm}!{\vrule width 2pt}}{
\begin{minipage}{16cm}
    \vspace{0.2cm}

\fontsize{14pt}{16pt}\selectfont\itshape
\begin{enumerate}
    \item Java: лямбда-выражения в Java, синтаксис и применение
    \item Java: функциональные интерфейсы Predicate и Function
    \item Создайте класс «Геокоординаты с высотой» с полями 
latitude, longitude (double), altitude (int, в метрах). 
Реализуйте методы clone, equals, hashCode и toString. При некорректных данных 
для инициализации вызывайте исключение. latitude может быть от -90 до +90, longitude 
от -180 до 180. 
\end{enumerate}

\vspace{0.2cm}
    
\end{minipage}
}
\\
\noalign{\hrule height 2pt}
\end{tabular}

\newpage


\begin{tabular}{!{\vrule width 2pt}p{5cm}|p{6cm}|p{4cm}!{\vrule width 2pt}}
    \noalign{\hrule height 2pt}

    {\centering 
    \fontsize{14pt}{16pt}\selectfont
    РУТ(МИИТ)

\vspace{14pt}

Академия «Высшая инженерная школа»

\vspace{14pt}

2025/2026 учебный год

    }
&
{
    \centering
\fontsize{14pt}{16pt}\selectfont

\textbf{ЭКЗАМЕНАЦИОННЫЙ
БИЛЕТ №7}


по дисциплине 

«Программирование на Java» 
\fontsize{12pt}{14pt}\selectfont
для студентов образовательной программы «Цифровая инженерия транспортных процессов»

}
&
{
\centering
\fontsize{14pt}{16pt}\selectfont

УТВЕРЖДАЮ
Руководитель образовательной программы

\vspace{1cm}

\fontsize{12pt}{14pt}\selectfont
\underline{\hspace{3cm}}

к.т.н., \underline{Проневич О.Б.}

}
\\
\hline
\multicolumn{3}{!{\vrule width 2pt}p{16cm}!{\vrule width 2pt}}{
\begin{minipage}{16cm}
    \vspace{0.2cm}

\fontsize{14pt}{16pt}\selectfont\itshape
\begin{enumerate}
    \item Объектно-ориентированное программирование: принципы и их описание
    \item Java: многопоточность в Java, понятие потока и процесса
    \item Разработайте REST API сервис на Spring Boot для хранения информации о самолетах 
авиакомпании (марка самолета и количество мест). Предоставьте эндпоинты для CRUD-операций. 
Используйте реляционную базу данных для хранения данных. 
Реализуйте валидацию входных данных. 
\end{enumerate}

\vspace{0.2cm}
    
\end{minipage}
}
\\
\noalign{\hrule height 2pt}
\end{tabular}

\newpage


\begin{tabular}{!{\vrule width 2pt}p{5cm}|p{6cm}|p{4cm}!{\vrule width 2pt}}
    \noalign{\hrule height 2pt}

    {\centering 
    \fontsize{14pt}{16pt}\selectfont
    РУТ(МИИТ)

\vspace{14pt}

Академия «Высшая инженерная школа»

\vspace{14pt}

2025/2026 учебный год

    }
&
{
    \centering
\fontsize{14pt}{16pt}\selectfont

\textbf{ЭКЗАМЕНАЦИОННЫЙ
БИЛЕТ №8}


по дисциплине 

«Программирование на Java» 
\fontsize{12pt}{14pt}\selectfont
для студентов образовательной программы «Цифровая инженерия транспортных процессов»

}
&
{
\centering
\fontsize{14pt}{16pt}\selectfont

УТВЕРЖДАЮ
Руководитель образовательной программы

\vspace{1cm}

\fontsize{12pt}{14pt}\selectfont
\underline{\hspace{3cm}}

к.т.н., \underline{Проневич О.Б.}

}
\\
\hline
\multicolumn{3}{!{\vrule width 2pt}p{16cm}!{\vrule width 2pt}}{
\begin{minipage}{16cm}
    \vspace{0.2cm}

\fontsize{14pt}{16pt}\selectfont\itshape
\begin{enumerate}
    \item Объектно-ориентированное программирование: итераторы (Iterator), назначение и принцип работы, самописная реализация
    \item Rest API: тестирование REST API
    \item Реализуйте программу с графическим интерфейсом JavaFX для имитации 
движения мяча на бильярдном столе. Учтите упругие отскоки от 
границ стола. Пользователь задаёт начальную скорость и направление движения, лунки не учитывайте. 
\end{enumerate}

\vspace{0.2cm}
    
\end{minipage}
}
\\
\noalign{\hrule height 2pt}
\end{tabular}

\newpage


\begin{tabular}{!{\vrule width 2pt}p{5cm}|p{6cm}|p{4cm}!{\vrule width 2pt}}
    \noalign{\hrule height 2pt}

    {\centering 
    \fontsize{14pt}{16pt}\selectfont
    РУТ(МИИТ)

\vspace{14pt}

Академия «Высшая инженерная школа»

\vspace{14pt}

2025/2026 учебный год

    }
&
{
    \centering
\fontsize{14pt}{16pt}\selectfont

\textbf{ЭКЗАМЕНАЦИОННЫЙ
БИЛЕТ №9}


по дисциплине 

«Программирование на Java» 
\fontsize{12pt}{14pt}\selectfont
для студентов образовательной программы «Цифровая инженерия транспортных процессов»

}
&
{
\centering
\fontsize{14pt}{16pt}\selectfont

УТВЕРЖДАЮ
Руководитель образовательной программы

\vspace{1cm}

\fontsize{12pt}{14pt}\selectfont
\underline{\hspace{3cm}}

к.т.н., \underline{Проневич О.Б.}

}
\\
\hline
\multicolumn{3}{!{\vrule width 2pt}p{16cm}!{\vrule width 2pt}}{
\begin{minipage}{16cm}
    \vspace{0.2cm}

\fontsize{14pt}{16pt}\selectfont\itshape
\begin{enumerate}
    \item Java: перечисления (enum), синтаксис и практическое применение
    \item Java: создание потоков - Thread, Runnable, ExecutorService
    \item Напишите консольную программу, которая по введённой дате 
определяет дату предыдущего дня. Запрещено использовать 
стандартные классы для работы с датами. Учтите високосные годы 
и корректное количество дней в месяцах. Проверяйте корректность входных данных.
Реализуйте UNIT-тестирование модели. 
\end{enumerate}

\vspace{0.2cm}
    
\end{minipage}
}
\\
\noalign{\hrule height 2pt}
\end{tabular}

\newpage


\begin{tabular}{!{\vrule width 2pt}p{5cm}|p{6cm}|p{4cm}!{\vrule width 2pt}}
    \noalign{\hrule height 2pt}

    {\centering 
    \fontsize{14pt}{16pt}\selectfont
    РУТ(МИИТ)

\vspace{14pt}

Академия «Высшая инженерная школа»

\vspace{14pt}

2025/2026 учебный год

    }
&
{
    \centering
\fontsize{14pt}{16pt}\selectfont

\textbf{ЭКЗАМЕНАЦИОННЫЙ
БИЛЕТ №10}


по дисциплине 

«Программирование на Java» 
\fontsize{12pt}{14pt}\selectfont
для студентов образовательной программы «Цифровая инженерия транспортных процессов»

}
&
{
\centering
\fontsize{14pt}{16pt}\selectfont

УТВЕРЖДАЮ
Руководитель образовательной программы

\vspace{1cm}

\fontsize{12pt}{14pt}\selectfont
\underline{\hspace{3cm}}

к.т.н., \underline{Проневич О.Б.}

}
\\
\hline
\multicolumn{3}{!{\vrule width 2pt}p{16cm}!{\vrule width 2pt}}{
\begin{minipage}{16cm}
    \vspace{0.2cm}

\fontsize{14pt}{16pt}\selectfont\itshape
\begin{enumerate}
    \item Java: record классы, назначение и преимущества для DTO
    \item JavaFX: использование TableView
    \item Реализуйте потокобезопасный класс SafeWallet с 
методами addMoney(int amount), spendMoney(int amount) (возвращает true, если хватает средств) 
и getBalance(). В main инициализируйте кошелёк с 2000 единицами,
 запустите 10 потоков: 5 потоков по 80 раз добавляют по 15, 5 потоков по 80 раз тратят по 15.
  После завершения выведите баланс. 
\end{enumerate}

\vspace{0.2cm}
    
\end{minipage}
}
\\
\noalign{\hrule height 2pt}
\end{tabular}

\newpage


\begin{tabular}{!{\vrule width 2pt}p{5cm}|p{6cm}|p{4cm}!{\vrule width 2pt}}
    \noalign{\hrule height 2pt}

    {\centering 
    \fontsize{14pt}{16pt}\selectfont
    РУТ(МИИТ)

\vspace{14pt}

Академия «Высшая инженерная школа»

\vspace{14pt}

2025/2026 учебный год

    }
&
{
    \centering
\fontsize{14pt}{16pt}\selectfont

\textbf{ЭКЗАМЕНАЦИОННЫЙ
БИЛЕТ №11}


по дисциплине 

«Программирование на Java» 
\fontsize{12pt}{14pt}\selectfont
для студентов образовательной программы «Цифровая инженерия транспортных процессов»

}
&
{
\centering
\fontsize{14pt}{16pt}\selectfont

УТВЕРЖДАЮ
Руководитель образовательной программы

\vspace{1cm}

\fontsize{12pt}{14pt}\selectfont
\underline{\hspace{3cm}}

к.т.н., \underline{Проневич О.Б.}

}
\\
\hline
\multicolumn{3}{!{\vrule width 2pt}p{16cm}!{\vrule width 2pt}}{
\begin{minipage}{16cm}
    \vspace{0.2cm}

\fontsize{14pt}{16pt}\selectfont\itshape
\begin{enumerate}
    \item Java: запечатанные классы (sealed classes), синтаксис и ограничения наследования
    \item JavaFX: выбор значений - ComboBox, RadioButton, DatePicker, ColorPicker
    \item Реализуйте программу с графическим интерфейсом JavaFX, которая читает файл с 
описанием расположения TextField'ов (координаты). После выбора файла на 
форме динамически создаются TextField'ы согласно данным из файла. 
При изменении любого числа в TextField'ах, в Label автоматически обновляется 
сумма всех введённых чисел. 
\end{enumerate}

\vspace{0.2cm}
    
\end{minipage}
}
\\
\noalign{\hrule height 2pt}
\end{tabular}

\newpage


\begin{tabular}{!{\vrule width 2pt}p{5cm}|p{6cm}|p{4cm}!{\vrule width 2pt}}
    \noalign{\hrule height 2pt}

    {\centering 
    \fontsize{14pt}{16pt}\selectfont
    РУТ(МИИТ)

\vspace{14pt}

Академия «Высшая инженерная школа»

\vspace{14pt}

2025/2026 учебный год

    }
&
{
    \centering
\fontsize{14pt}{16pt}\selectfont

\textbf{ЭКЗАМЕНАЦИОННЫЙ
БИЛЕТ №12}


по дисциплине 

«Программирование на Java» 
\fontsize{12pt}{14pt}\selectfont
для студентов образовательной программы «Цифровая инженерия транспортных процессов»

}
&
{
\centering
\fontsize{14pt}{16pt}\selectfont

УТВЕРЖДАЮ
Руководитель образовательной программы

\vspace{1cm}

\fontsize{12pt}{14pt}\selectfont
\underline{\hspace{3cm}}

к.т.н., \underline{Проневич О.Б.}

}
\\
\hline
\multicolumn{3}{!{\vrule width 2pt}p{16cm}!{\vrule width 2pt}}{
\begin{minipage}{16cm}
    \vspace{0.2cm}

\fontsize{14pt}{16pt}\selectfont\itshape
\begin{enumerate}
    \item Структуры данных: односвязные списки, реализация методов добавления и удаления в Java
    \item Rest API: описание, HTTP методы и статусы
    \item Разработайте REST API сервис на Spring Boot для управления заметками (поля -- заголовок и содержание). 
Предоставьте эндпоинты для CRUD-операций. Используйте реляционную базу данных 
для хранения данных. 
\end{enumerate}

\vspace{0.2cm}
    
\end{minipage}
}
\\
\noalign{\hrule height 2pt}
\end{tabular}

\newpage


\begin{tabular}{!{\vrule width 2pt}p{5cm}|p{6cm}|p{4cm}!{\vrule width 2pt}}
    \noalign{\hrule height 2pt}

    {\centering 
    \fontsize{14pt}{16pt}\selectfont
    РУТ(МИИТ)

\vspace{14pt}

Академия «Высшая инженерная школа»

\vspace{14pt}

2025/2026 учебный год

    }
&
{
    \centering
\fontsize{14pt}{16pt}\selectfont

\textbf{ЭКЗАМЕНАЦИОННЫЙ
БИЛЕТ №13}


по дисциплине 

«Программирование на Java» 
\fontsize{12pt}{14pt}\selectfont
для студентов образовательной программы «Цифровая инженерия транспортных процессов»

}
&
{
\centering
\fontsize{14pt}{16pt}\selectfont

УТВЕРЖДАЮ
Руководитель образовательной программы

\vspace{1cm}

\fontsize{12pt}{14pt}\selectfont
\underline{\hspace{3cm}}

к.т.н., \underline{Проневич О.Б.}

}
\\
\hline
\multicolumn{3}{!{\vrule width 2pt}p{16cm}!{\vrule width 2pt}}{
\begin{minipage}{16cm}
    \vspace{0.2cm}

\fontsize{14pt}{16pt}\selectfont\itshape
\begin{enumerate}
    \item Обзор языков высокого уровня: особенности статической и динамической типизации, плюсы и минусы, основные парадигмы программирования и их поддержка в Java
    \item Java: потокобезопасные коллекции
    \item Создайте приложение на JavaFX для работы с базой данных самолетов авиакомпании. 
Поля: марка самолета, количество мест. Используйте SQLite для хранения данных. 
Реализуйте возможность добавления, чтения, фильтрации записей по первому полю. 
\end{enumerate}

\vspace{0.2cm}
    
\end{minipage}
}
\\
\noalign{\hrule height 2pt}
\end{tabular}

\newpage


\begin{tabular}{!{\vrule width 2pt}p{5cm}|p{6cm}|p{4cm}!{\vrule width 2pt}}
    \noalign{\hrule height 2pt}

    {\centering 
    \fontsize{14pt}{16pt}\selectfont
    РУТ(МИИТ)

\vspace{14pt}

Академия «Высшая инженерная школа»

\vspace{14pt}

2025/2026 учебный год

    }
&
{
    \centering
\fontsize{14pt}{16pt}\selectfont

\textbf{ЭКЗАМЕНАЦИОННЫЙ
БИЛЕТ №14}


по дисциплине 

«Программирование на Java» 
\fontsize{12pt}{14pt}\selectfont
для студентов образовательной программы «Цифровая инженерия транспортных процессов»

}
&
{
\centering
\fontsize{14pt}{16pt}\selectfont

УТВЕРЖДАЮ
Руководитель образовательной программы

\vspace{1cm}

\fontsize{12pt}{14pt}\selectfont
\underline{\hspace{3cm}}

к.т.н., \underline{Проневич О.Б.}

}
\\
\hline
\multicolumn{3}{!{\vrule width 2pt}p{16cm}!{\vrule width 2pt}}{
\begin{minipage}{16cm}
    \vspace{0.2cm}

\fontsize{14pt}{16pt}\selectfont\itshape
\begin{enumerate}
    \item Java: особенности логических операторов \&\& и ||, тернарного оператора, префиксных и постфиксных инкрементов/декрементов
    \item Spring Boot: JPA, аннотации @Entity, @Id, CRUD репозитории
    \item Создайте приложение на JavaFX для работы с базой данных самолетов авиакомпании. 
Поля: марка самолета, количество мест. Используйте SQLite для хранения данных. 
Реализуйте возможность добавления, чтения, удаления записей. 
\end{enumerate}

\vspace{0.2cm}
    
\end{minipage}
}
\\
\noalign{\hrule height 2pt}
\end{tabular}

\newpage


\begin{tabular}{!{\vrule width 2pt}p{5cm}|p{6cm}|p{4cm}!{\vrule width 2pt}}
    \noalign{\hrule height 2pt}

    {\centering 
    \fontsize{14pt}{16pt}\selectfont
    РУТ(МИИТ)

\vspace{14pt}

Академия «Высшая инженерная школа»

\vspace{14pt}

2025/2026 учебный год

    }
&
{
    \centering
\fontsize{14pt}{16pt}\selectfont

\textbf{ЭКЗАМЕНАЦИОННЫЙ
БИЛЕТ №15}


по дисциплине 

«Программирование на Java» 
\fontsize{12pt}{14pt}\selectfont
для студентов образовательной программы «Цифровая инженерия транспортных процессов»

}
&
{
\centering
\fontsize{14pt}{16pt}\selectfont

УТВЕРЖДАЮ
Руководитель образовательной программы

\vspace{1cm}

\fontsize{12pt}{14pt}\selectfont
\underline{\hspace{3cm}}

к.т.н., \underline{Проневич О.Б.}

}
\\
\hline
\multicolumn{3}{!{\vrule width 2pt}p{16cm}!{\vrule width 2pt}}{
\begin{minipage}{16cm}
    \vspace{0.2cm}

\fontsize{14pt}{16pt}\selectfont\itshape
\begin{enumerate}
    \item Java: принципы создания unit-тестов, требования к качеству тестов, средства Java для создания автотестов
    \item JDBC: выполнение SQL запросов в Java, PreparedStatement, ResultSet
    \item Напишите JavaFX программу, которая по введённой дате определяет дату, которая была месяц назад. 
Если в предыдущем месяце нет дня с таким же числом, возьмите последний день 
предыдущего месяца. Запрещено использовать стандартные классы для работы с датами.
Проверяйте корректность исходных данных. Разработайте UNIT-тесты 
\end{enumerate}

\vspace{0.2cm}
    
\end{minipage}
}
\\
\noalign{\hrule height 2pt}
\end{tabular}

\newpage


\begin{tabular}{!{\vrule width 2pt}p{5cm}|p{6cm}|p{4cm}!{\vrule width 2pt}}
    \noalign{\hrule height 2pt}

    {\centering 
    \fontsize{14pt}{16pt}\selectfont
    РУТ(МИИТ)

\vspace{14pt}

Академия «Высшая инженерная школа»

\vspace{14pt}

2025/2026 учебный год

    }
&
{
    \centering
\fontsize{14pt}{16pt}\selectfont

\textbf{ЭКЗАМЕНАЦИОННЫЙ
БИЛЕТ №16}


по дисциплине 

«Программирование на Java» 
\fontsize{12pt}{14pt}\selectfont
для студентов образовательной программы «Цифровая инженерия транспортных процессов»

}
&
{
\centering
\fontsize{14pt}{16pt}\selectfont

УТВЕРЖДАЮ
Руководитель образовательной программы

\vspace{1cm}

\fontsize{12pt}{14pt}\selectfont
\underline{\hspace{3cm}}

к.т.н., \underline{Проневич О.Б.}

}
\\
\hline
\multicolumn{3}{!{\vrule width 2pt}p{16cm}!{\vrule width 2pt}}{
\begin{minipage}{16cm}
    \vspace{0.2cm}

\fontsize{14pt}{16pt}\selectfont\itshape
\begin{enumerate}
    \item Java: сравнение массивов и коллекций List, особенности использования
    \item Java: гонки данных (race conditions), потокобезопасность
    \item Реализуйте класс LinearEquationSolver с методом static LinearResult 
solve(double a, double b), решающим уравнение $ax + b = 0$. LinearResult 
должен быть sealed-классом с подклассами NoSolution, InfiniteSolutions и 
UniqueSolution. Напишите unit-тесты для всех случаев. 
\end{enumerate}

\vspace{0.2cm}
    
\end{minipage}
}
\\
\noalign{\hrule height 2pt}
\end{tabular}

\newpage


\begin{tabular}{!{\vrule width 2pt}p{5cm}|p{6cm}|p{4cm}!{\vrule width 2pt}}
    \noalign{\hrule height 2pt}

    {\centering 
    \fontsize{14pt}{16pt}\selectfont
    РУТ(МИИТ)

\vspace{14pt}

Академия «Высшая инженерная школа»

\vspace{14pt}

2025/2026 учебный год

    }
&
{
    \centering
\fontsize{14pt}{16pt}\selectfont

\textbf{ЭКЗАМЕНАЦИОННЫЙ
БИЛЕТ №17}


по дисциплине 

«Программирование на Java» 
\fontsize{12pt}{14pt}\selectfont
для студентов образовательной программы «Цифровая инженерия транспортных процессов»

}
&
{
\centering
\fontsize{14pt}{16pt}\selectfont

УТВЕРЖДАЮ
Руководитель образовательной программы

\vspace{1cm}

\fontsize{12pt}{14pt}\selectfont
\underline{\hspace{3cm}}

к.т.н., \underline{Проневич О.Б.}

}
\\
\hline
\multicolumn{3}{!{\vrule width 2pt}p{16cm}!{\vrule width 2pt}}{
\begin{minipage}{16cm}
    \vspace{0.2cm}

\fontsize{14pt}{16pt}\selectfont\itshape
\begin{enumerate}
    \item Java: String vs StringBuilder - производительность и применение
    \item Spring Boot: архитектура, аннотации @SpringBootApplication, @RestController
    \item Напишите программу, которая читает $r$ целых чисел ($1 \leq r \leq 100$). Используя 
ArrayList<Integer>, сохраните их. Затем обработайте операции: "remove последний" - 
удалить последний элемент списка, только если он положительный. 
После всех операций выведите список. 
\end{enumerate}

\vspace{0.2cm}
    
\end{minipage}
}
\\
\noalign{\hrule height 2pt}
\end{tabular}

\newpage


\begin{tabular}{!{\vrule width 2pt}p{5cm}|p{6cm}|p{4cm}!{\vrule width 2pt}}
    \noalign{\hrule height 2pt}

    {\centering 
    \fontsize{14pt}{16pt}\selectfont
    РУТ(МИИТ)

\vspace{14pt}

Академия «Высшая инженерная школа»

\vspace{14pt}

2025/2026 учебный год

    }
&
{
    \centering
\fontsize{14pt}{16pt}\selectfont

\textbf{ЭКЗАМЕНАЦИОННЫЙ
БИЛЕТ №18}


по дисциплине 

«Программирование на Java» 
\fontsize{12pt}{14pt}\selectfont
для студентов образовательной программы «Цифровая инженерия транспортных процессов»

}
&
{
\centering
\fontsize{14pt}{16pt}\selectfont

УТВЕРЖДАЮ
Руководитель образовательной программы

\vspace{1cm}

\fontsize{12pt}{14pt}\selectfont
\underline{\hspace{3cm}}

к.т.н., \underline{Проневич О.Б.}

}
\\
\hline
\multicolumn{3}{!{\vrule width 2pt}p{16cm}!{\vrule width 2pt}}{
\begin{minipage}{16cm}
    \vspace{0.2cm}

\fontsize{14pt}{16pt}\selectfont\itshape
\begin{enumerate}
    \item Java: статические методы и поля, особенности использования static
    \item JavaFX: работа с файлами
    \item Создайте приложение на JavaFX для работы с базой данных автобусных маршрутов. 
Поля: номер маршрута, номер парка. Используйте SQLite для хранения данных. 
Реализуйте возможность добавления, чтения, удаления записей. 
\end{enumerate}

\vspace{0.2cm}
    
\end{minipage}
}
\\
\noalign{\hrule height 2pt}
\end{tabular}

\newpage


\begin{tabular}{!{\vrule width 2pt}p{5cm}|p{6cm}|p{4cm}!{\vrule width 2pt}}
    \noalign{\hrule height 2pt}

    {\centering 
    \fontsize{14pt}{16pt}\selectfont
    РУТ(МИИТ)

\vspace{14pt}

Академия «Высшая инженерная школа»

\vspace{14pt}

2025/2026 учебный год

    }
&
{
    \centering
\fontsize{14pt}{16pt}\selectfont

\textbf{ЭКЗАМЕНАЦИОННЫЙ
БИЛЕТ №19}


по дисциплине 

«Программирование на Java» 
\fontsize{12pt}{14pt}\selectfont
для студентов образовательной программы «Цифровая инженерия транспортных процессов»

}
&
{
\centering
\fontsize{14pt}{16pt}\selectfont

УТВЕРЖДАЮ
Руководитель образовательной программы

\vspace{1cm}

\fontsize{12pt}{14pt}\selectfont
\underline{\hspace{3cm}}

к.т.н., \underline{Проневич О.Б.}

}
\\
\hline
\multicolumn{3}{!{\vrule width 2pt}p{16cm}!{\vrule width 2pt}}{
\begin{minipage}{16cm}
    \vspace{0.2cm}

\fontsize{14pt}{16pt}\selectfont\itshape
\begin{enumerate}
    \item Java: работа с классом Scanner для ввода данных
    \item Spring Boot: валидация данных, аннотации @Valid, @NotNull
    \item Реализуйте класс SystemSolver с методом static SystemResult solve(double a1, double b1, double c1, double a2, double b2, double c2) для решения системы 
двух линейных уравнений. SystemResult должен быть sealed-классом с подклассами 
NoSolution, InfiniteSolutions и UniqueSolution. Напишите unit-тесты. 
\end{enumerate}

\vspace{0.2cm}
    
\end{minipage}
}
\\
\noalign{\hrule height 2pt}
\end{tabular}

\newpage


\begin{tabular}{!{\vrule width 2pt}p{5cm}|p{6cm}|p{4cm}!{\vrule width 2pt}}
    \noalign{\hrule height 2pt}

    {\centering 
    \fontsize{14pt}{16pt}\selectfont
    РУТ(МИИТ)

\vspace{14pt}

Академия «Высшая инженерная школа»

\vspace{14pt}

2025/2026 учебный год

    }
&
{
    \centering
\fontsize{14pt}{16pt}\selectfont

\textbf{ЭКЗАМЕНАЦИОННЫЙ
БИЛЕТ №20}


по дисциплине 

«Программирование на Java» 
\fontsize{12pt}{14pt}\selectfont
для студентов образовательной программы «Цифровая инженерия транспортных процессов»

}
&
{
\centering
\fontsize{14pt}{16pt}\selectfont

УТВЕРЖДАЮ
Руководитель образовательной программы

\vspace{1cm}

\fontsize{12pt}{14pt}\selectfont
\underline{\hspace{3cm}}

к.т.н., \underline{Проневич О.Б.}

}
\\
\hline
\multicolumn{3}{!{\vrule width 2pt}p{16cm}!{\vrule width 2pt}}{
\begin{minipage}{16cm}
    \vspace{0.2cm}

\fontsize{14pt}{16pt}\selectfont\itshape
\begin{enumerate}
    \item Java: вложенные классы, отличия статических и нестатических внутренних классов
    \item Spring Data: работа с базой данных H2, конфигурация
    \item Создайте приложение на JavaFX для работы с базой данных самолетов авиакомпании. 
Поля: марка самолета, количество мест. Используйте SQLite для хранения данных. 
Реализуйте возможность добавления, чтения, изменения записей. 
\end{enumerate}

\vspace{0.2cm}
    
\end{minipage}
}
\\
\noalign{\hrule height 2pt}
\end{tabular}

\newpage


\begin{tabular}{!{\vrule width 2pt}p{5cm}|p{6cm}|p{4cm}!{\vrule width 2pt}}
    \noalign{\hrule height 2pt}

    {\centering 
    \fontsize{14pt}{16pt}\selectfont
    РУТ(МИИТ)

\vspace{14pt}

Академия «Высшая инженерная школа»

\vspace{14pt}

2025/2026 учебный год

    }
&
{
    \centering
\fontsize{14pt}{16pt}\selectfont

\textbf{ЭКЗАМЕНАЦИОННЫЙ
БИЛЕТ №21}


по дисциплине 

«Программирование на Java» 
\fontsize{12pt}{14pt}\selectfont
для студентов образовательной программы «Цифровая инженерия транспортных процессов»

}
&
{
\centering
\fontsize{14pt}{16pt}\selectfont

УТВЕРЖДАЮ
Руководитель образовательной программы

\vspace{1cm}

\fontsize{12pt}{14pt}\selectfont
\underline{\hspace{3cm}}

к.т.н., \underline{Проневич О.Б.}

}
\\
\hline
\multicolumn{3}{!{\vrule width 2pt}p{16cm}!{\vrule width 2pt}}{
\begin{minipage}{16cm}
    \vspace{0.2cm}

\fontsize{14pt}{16pt}\selectfont\itshape
\begin{enumerate}
    \item Java: Map интерфейс, основные реализации и сценарии использования
    \item Spring Boot: : конфигурация, application.properties, встроенный Tomcat
    \item Создайте класс «Дробь с единицей измерения» с полями numerator, denominator 
(int, знаменатель $\neq$ 0), unit (String). Реализуйте методы 
clone, equals, hashCode и toString. При некорректных данных для 
инициализации вызывайте исключение. 
\end{enumerate}

\vspace{0.2cm}
    
\end{minipage}
}
\\
\noalign{\hrule height 2pt}
\end{tabular}

\newpage


\begin{tabular}{!{\vrule width 2pt}p{5cm}|p{6cm}|p{4cm}!{\vrule width 2pt}}
    \noalign{\hrule height 2pt}

    {\centering 
    \fontsize{14pt}{16pt}\selectfont
    РУТ(МИИТ)

\vspace{14pt}

Академия «Высшая инженерная школа»

\vspace{14pt}

2025/2026 учебный год

    }
&
{
    \centering
\fontsize{14pt}{16pt}\selectfont

\textbf{ЭКЗАМЕНАЦИОННЫЙ
БИЛЕТ №22}


по дисциплине 

«Программирование на Java» 
\fontsize{12pt}{14pt}\selectfont
для студентов образовательной программы «Цифровая инженерия транспортных процессов»

}
&
{
\centering
\fontsize{14pt}{16pt}\selectfont

УТВЕРЖДАЮ
Руководитель образовательной программы

\vspace{1cm}

\fontsize{12pt}{14pt}\selectfont
\underline{\hspace{3cm}}

к.т.н., \underline{Проневич О.Б.}

}
\\
\hline
\multicolumn{3}{!{\vrule width 2pt}p{16cm}!{\vrule width 2pt}}{
\begin{minipage}{16cm}
    \vspace{0.2cm}

\fontsize{14pt}{16pt}\selectfont\itshape
\begin{enumerate}
    \item Java: модификатор abstract, назначение и примеры использования
    \item Java: атомарные переменные (AtomicInteger, AtomicLong), назначение
    \item Реализуйте функционал регистрации и аутентификации пользователей 
в REST API сервисе с использованием JWT. Реализуйте middleware 
для проверки JWT-токенов. Реализуйте JavaFX-клиент, который осуществляет регистрацию
и аутентификацию пользователей с помощью сервиса. 
\end{enumerate}

\vspace{0.2cm}
    
\end{minipage}
}
\\
\noalign{\hrule height 2pt}
\end{tabular}

\newpage


\begin{tabular}{!{\vrule width 2pt}p{5cm}|p{6cm}|p{4cm}!{\vrule width 2pt}}
    \noalign{\hrule height 2pt}

    {\centering 
    \fontsize{14pt}{16pt}\selectfont
    РУТ(МИИТ)

\vspace{14pt}

Академия «Высшая инженерная школа»

\vspace{14pt}

2025/2026 учебный год

    }
&
{
    \centering
\fontsize{14pt}{16pt}\selectfont

\textbf{ЭКЗАМЕНАЦИОННЫЙ
БИЛЕТ №23}


по дисциплине 

«Программирование на Java» 
\fontsize{12pt}{14pt}\selectfont
для студентов образовательной программы «Цифровая инженерия транспортных процессов»

}
&
{
\centering
\fontsize{14pt}{16pt}\selectfont

УТВЕРЖДАЮ
Руководитель образовательной программы

\vspace{1cm}

\fontsize{12pt}{14pt}\selectfont
\underline{\hspace{3cm}}

к.т.н., \underline{Проневич О.Б.}

}
\\
\hline
\multicolumn{3}{!{\vrule width 2pt}p{16cm}!{\vrule width 2pt}}{
\begin{minipage}{16cm}
    \vspace{0.2cm}

\fontsize{14pt}{16pt}\selectfont\itshape
\begin{enumerate}
    \item Java: основные методы класса Object, контракты
    \item JavaFX: архитектура - Stage, Scene, контейнеры
    \item Реализуйте иерархию классов для документов: базовый класс Документ с полями 
номер, дата, производные классы Счёт (сумма, покупатель) и Приказ (текст приказа, исполнитель). 
Реализуйте метод вывода информации о документе. 
Выведите список документов в хронологическом порядке (с полной информацией о них). 
\end{enumerate}

\vspace{0.2cm}
    
\end{minipage}
}
\\
\noalign{\hrule height 2pt}
\end{tabular}

\newpage


\begin{tabular}{!{\vrule width 2pt}p{5cm}|p{6cm}|p{4cm}!{\vrule width 2pt}}
    \noalign{\hrule height 2pt}

    {\centering 
    \fontsize{14pt}{16pt}\selectfont
    РУТ(МИИТ)

\vspace{14pt}

Академия «Высшая инженерная школа»

\vspace{14pt}

2025/2026 учебный год

    }
&
{
    \centering
\fontsize{14pt}{16pt}\selectfont

\textbf{ЭКЗАМЕНАЦИОННЫЙ
БИЛЕТ №24}


по дисциплине 

«Программирование на Java» 
\fontsize{12pt}{14pt}\selectfont
для студентов образовательной программы «Цифровая инженерия транспортных процессов»

}
&
{
\centering
\fontsize{14pt}{16pt}\selectfont

УТВЕРЖДАЮ
Руководитель образовательной программы

\vspace{1cm}

\fontsize{12pt}{14pt}\selectfont
\underline{\hspace{3cm}}

к.т.н., \underline{Проневич О.Б.}

}
\\
\hline
\multicolumn{3}{!{\vrule width 2pt}p{16cm}!{\vrule width 2pt}}{
\begin{minipage}{16cm}
    \vspace{0.2cm}

\fontsize{14pt}{16pt}\selectfont\itshape
\begin{enumerate}
    \item Java: реализация инкапсуляции, наследования и полиморфизма в Java
    \item Java: пул потоков, создание фиксированного и однопоточного пула
    \item Напишите программу, которая читает целое
 число $n$ ($1 \leq n \leq 100$), затем $n$ строк. 
 Используя StringBuilder, обработайте каждую строку: удалите 
 все гласные и добавьте в результат только если строка 
 начинается с согласной. Между добавленными строками вставьте разделитель "---". 
\end{enumerate}

\vspace{0.2cm}
    
\end{minipage}
}
\\
\noalign{\hrule height 2pt}
\end{tabular}

\newpage


\begin{tabular}{!{\vrule width 2pt}p{5cm}|p{6cm}|p{4cm}!{\vrule width 2pt}}
    \noalign{\hrule height 2pt}

    {\centering 
    \fontsize{14pt}{16pt}\selectfont
    РУТ(МИИТ)

\vspace{14pt}

Академия «Высшая инженерная школа»

\vspace{14pt}

2025/2026 учебный год

    }
&
{
    \centering
\fontsize{14pt}{16pt}\selectfont

\textbf{ЭКЗАМЕНАЦИОННЫЙ
БИЛЕТ №25}


по дисциплине 

«Программирование на Java» 
\fontsize{12pt}{14pt}\selectfont
для студентов образовательной программы «Цифровая инженерия транспортных процессов»

}
&
{
\centering
\fontsize{14pt}{16pt}\selectfont

УТВЕРЖДАЮ
Руководитель образовательной программы

\vspace{1cm}

\fontsize{12pt}{14pt}\selectfont
\underline{\hspace{3cm}}

к.т.н., \underline{Проневич О.Б.}

}
\\
\hline
\multicolumn{3}{!{\vrule width 2pt}p{16cm}!{\vrule width 2pt}}{
\begin{minipage}{16cm}
    \vspace{0.2cm}

\fontsize{14pt}{16pt}\selectfont\itshape
\begin{enumerate}
    \item Java: классы и объекты в Java, синтаксис определения класса
    \item Spring Data: обработка ошибок, @ControllerAdvice, @ExceptionHandler
    \item Разработайте REST API сервис на Spring Boot для управления автобусными маршрутами (номер маршрута и номер парка). 
Предоставьте эндпоинты для CRUD-операций. Используйте реляционную базу данных для хранения данных. 
Реализуйте валидацию входных данных. 
\end{enumerate}

\vspace{0.2cm}
    
\end{minipage}
}
\\
\noalign{\hrule height 2pt}
\end{tabular}

\newpage

\end{document}
