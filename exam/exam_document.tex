
\documentclass[a4paper,12pt]{article}
\usepackage[utf8]{inputenc}
\usepackage[russian]{babel}
\usepackage{titlesec}
\usepackage{hyperref}
\usepackage{enumitem}
\usepackage{amssymb}
\titleformat{\section}[block]{\normalfont\Large\bfseries}{}{0em}{}
\titleformat{\subsection}[block]{\normalfont\large\bfseries}{}{0em}{}

\title{Вопросы к экзамену по курсу <<Разработка на Java>>}
\author{}
\date{}

\begin{document}

\maketitle

В ходе практической части экзамена будет доступна документация по Java (с помощью zeal), 
по Spring Boot (с помощью zeal), 
JUnit (с помощью пользовательского контента для zeal: 
\url{https://github.com/Kapeli/Dash-User-Contributions/tree/master/docsets/JUnit5},
JavaFX (самостоятельно будет импортировано преподавателем)).

Кроме того, практическое задание можно выполнять с собственноручным конспектом.

Для Spring проектов будет предоставлен результат запуска \url{https://start.spring.io/}


\textbf{В билете должно быть два теоретических и один практический вопрос}

\section*{Теоретические вопросы}
\begin{enumerate}[left=0pt]
    \item Обзор языков высокого уровня: особенности статической и динамической типизации, плюсы и минусы, основные парадигмы программирования и их поддержка в Java
    \item Java: архитектура выполнения - JVM, JDK, компиляция и интерпретация
    \item Java: особенности логических операторов \&\& и ||, тернарного оператора, префиксных и постфиксных инкрементов/декрементов
    \item Java: примитивные и ссылочные типы данных, автоупаковка и распаковка
    \item Java: работа с классом Scanner для ввода данных
    \item Java: сравнение массивов и коллекций List, особенности использования
    \item Java: отличия List и Set, принципы работы с множествами
    \item Java: String vs StringBuilder - производительность и применение
    \item Java: Map интерфейс, основные реализации и сценарии использования
    \item Объектно-ориентированное программирование: принципы и их описание
    \item Java: реализация инкапсуляции, наследования и полиморфизма в Java
    \item Java: абстрактные классы и интерфейсы, отличия и применение
    \item Java: классы и объекты в Java, синтаксис определения класса
    \item Java: модификатор abstract, назначение и примеры использования
    \item Java: перечисления (enum), синтаксис и практическое применение
    \item Java: запечатанные классы (sealed classes), синтаксис и ограничения наследования
    \item Java: статические методы и поля, особенности использования static
    \item Java: вложенные классы, отличия статических и нестатических внутренних классов
    \item Java: record классы, назначение и преимущества для DTO
    \item Java: основные методы класса Object, контракты
    \item Java: принципы создания unit-тестов, требования к качеству тестов, средства Java для создания автотестов
    \item Java: основные методы класса String
    \item Структуры данных: односвязные списки, реализация методов добавления и удаления в Java
    \item Объектно-ориентированное программирование: итераторы (Iterator), назначение и принцип работы, самописная реализация
    \item Java: лямбда-выражения в Java, синтаксис и применение
\end{enumerate}
\begin{enumerate}[left=0pt]
    \item Java: многопоточность в Java, понятие потока и процесса
    \item Java: создание потоков - Thread, Runnable, ExecutorService
    \item Java: синхронизация потоков, ключевое слово synchronized
    \item Java: взаимная блокировка (deadlock), причины и методы предотвращения
    \item Java: гонки данных (race conditions), потокобезопасность
    \item Java: атомарные переменные (AtomicInteger, AtomicLong), назначение
    \item Java: пул потоков, создание фиксированного и однопоточного пула
    \item Java: функциональные интерфейсы Predicate и Function
    \item Java: потокобезопасные коллекции
    \item JavaFX: архитектура - Stage, Scene, контейнеры
    \item JavaFX: компоновка - HBox, VBox, GridPane, AnchorPane, Pane, Canvas, TabPane
    \item JavaFX: элементы управления  - TextField, Label, Button, CheckBox
    \item JavaFX: выбор значений - ComboBox, RadioButton, DatePicker, ColorPicker
    \item JavaFX: работа с файлами
    \item JavaFX: использование TableView
    \item JDBC: подключение SQLite
    \item JDBC: выполнение SQL запросов в Java, PreparedStatement, ResultSet
    \item Spring Boot: архитектура, аннотации @SpringBootApplication, @RestController
    \item Spring Boot: : конфигурация, application.properties, встроенный Tomcat
    \item Rest API: описание, HTTP методы и статусы
    \item Spring Boot: JPA, аннотации @Entity, @Id, CRUD репозитории
    \item Spring Data: обработка ошибок, @ControllerAdvice, @ExceptionHandler
    \item Spring Data: работа с базой данных H2, конфигурация
    \item Rest API: тестирование REST API
    \item Spring Boot: валидация данных, аннотации @Valid, @NotNull
\end{enumerate}

\section*{Практические задания}
\subsection*{Задание 1}
Напишите консольную программу, которая решает неравенство $(x - a)(x - b) > 0$, где $a$ и $b$ — вещественные 
числа, вводимые пользователем. Обеспечьте полную проверку ввода данных и 
обработку всех математических случаев. Создать Unit-тест для класса решения неравенства.

\subsection*{Задание 2}
Напишите консольную программу, которая по введённой дате (день, месяц, год) определяет дату 
следующего дня. Запрещено использовать стандартные классы для работы с датами. 
Обеспечьте проверку корректности ввода. Создать Unit-тест для класса, решающего задачу.

\subsection*{Задание 3}
Напишите JavaFX программу, которая решает неравенство $(x - a)(x - b) > 0$, где $a$ и $b$ — вещественные 
числа, вводимые пользователем. Обеспечьте полную проверку ввода данных и 
обработку всех математических случаев.

\subsection*{Задание 4}
Напишите JavaFX программу, которая по введённой дате (день, месяц, год) определяет дату 
следующего дня. Запрещено использовать стандартные классы для работы с датами. 
Обеспечьте проверку корректности ввода.

\subsection*{Задание 5}
Напишите программу, которая читает целое число $n$ ($1 \leq n \leq 100$), затем $n$ строк. 
Используя StringBuilder, обработайте каждую строку: инвертируйте её и добавьте 
в результат только если длина строки больше 5. Между добавленными строками 
вставьте разделитель "---".

\subsection*{Задание 6}
Напишите программу, которая читает целое число $m$ ($1 \leq m \leq 100$), затем $m$ пар: 
строка-ключ и целое значение. Используя HashMap<String, Integer>, сохраните данные 
(при дубликатах ключей суммируйте значения). Затем для каждого запроса 
выведите сумму значений, чьи ключи начинаются с запроса.

\subsection*{Задание 7}
Напишите программу, которая читает $p$ целых чисел ($1 \leq p \leq 200$). Используя 
HashSet<Integer>, сохраните уникальные значения. Затем для каждого запроса 
удалите число из множества (если оно есть) и добавьте его квадрат. 
В конце выведите элементы в порядке возрастания.

\subsection*{Задание 8}
Напишите программу, которая читает $r$ целых чисел ($1 \leq r \leq 100$). Используя 
ArrayList<Integer>, сохраните их. Затем обработайте операции: 
"add X Y" - добавить X в позицию Y mod size (только если X > 0). После всех 
операций выведите список в прямом порядке.

\subsection*{Задание 9}
Реализуйте иерархию классов для работников: базовый класс 
Сотрудник с полями ФИО и должность, производные классы 
Преподаватель (количество часов в год) и Лаборант (количество ставок). Реализуйте 
метод расчета годовой оплаты. Выведите список работников в порядке возрастания оплаты. 
Дополнительные необходимые параметры храните в отдельном класс Конфигурация.

\subsection*{Задание 10}
Создайте класс «Точка в 3D» с полями x, y, z (тип double). Реализуйте методы 
clone, equals, hashCode и toString. При некорректных данных для инициализации 
вызывайте исключение.

\subsection*{Задание 11}
Реализуйте класс MyLinkedList, содержащий внутренний класс Node и 
вложенный итератор. Метод add(int value) добавляет элемент в конец списка. 
Метод remove(int value) удаляет первое вхождение значения. 
Метод print() выводит все элементы. Итератор последовательно возвращает все элементы.

\subsection*{Задание 12}
Реализуйте класс LinearEquationSolver с методом static LinearResult 
solve(double a, double b), решающим уравнение $ax + b = 0$. LinearResult 
должен быть sealed-классом с подклассами NoSolution, InfiniteSolutions и 
UniqueSolution. Напишите unit-тесты для всех случаев.

\subsection*{Задание 13}
Реализуйте потокобезопасный класс Counter с методами increment(int value), 
decrement(int value) (только если результат не станет отрицательным) и getValue(). 
В main создайте экземпляр с начальным значением 500, запустите 10 потоков: 
5 потоков по 100 раз вызывают increment(5), 5 потоков по 100 раз вызывают decrement(5). 
После завершения выведите итоговое значение.

\subsection*{Задание 14}
Реализуйте программу с графическим интерфейсом JavaFX для решения 
квадратного уравнения $ax^2 + bx + c = 0$. Обеспечьте ввод коэффициентов, 
проверку корректности данных и вывод результатов в удобном виде.

\subsection*{Задание 15}
Реализуйте программу с графическим интерфейсом JavaFX, которая динамически 
создаёт таблицу (максимум $10\times10$) по заданным размерам. После заполнения 
таблицы числами программа рассчитывает сумму чисел в каждой строке, 
сумму чисел в каждом столбце и общую сумму всех чисел.

\subsection*{Задание 16}
Реализуйте программу с графическим интерфейсом JavaFX для имитации 
движения мяча на бильярдном столе. Учтите упругие отскоки от 
границ стола. Пользователь задаёт начальную скорость и направление движения, лунки не учитывайте.

\subsection*{Задание 17}
Создайте приложение на JavaFX для работы с 
базой данных заметок. Поля: заголовок, содержание. 
Используйте SQLite для хранения данных. Реализуйте возможность 
добавления, чтения, изменения записей.

\subsection*{Задание 18}
Создайте приложение на JavaFX для работы с 
базой данных заметок. Поля: заголовок, содержание. 
Используйте SQLite для хранения данных. Реализуйте возможность 
добавления, чтения, удаления записей.

\subsection*{Задание 19}
Разработайте REST API сервис на Spring Boot для управления заметками (поля -- заголовок и содержание). 
Предоставьте эндпоинты для CRUD-операций. Используйте реляционную базу данных 
для хранения данных.

\subsection*{Задание 20}
Разработайте REST API сервис на Spring Boot для управления заметками 
(поля -- заголовок и содержание). Реализуйте только создание и получение списка заметок.
Создайте клиент на JavaFX для этого сервиса

\subsection*{Задание 21}
Реализуйте функционал регистрации и аутентификации пользователей 
в REST API сервисе с использованием JWT. Реализуйте middleware 
для проверки JWT-токенов. Реализуйте JavaFX-клиент, который осуществляет регистрацию
и аутентификацию пользователей с помощью сервиса.

\subsection*{Задание 22}
Напишите консольное приложение для решения неравенства $\frac{x - a}{x - b} > 0$, где $a$ и $b$ — вещественные числа, 
вводимые пользователем. Обеспечьте проверку деления на ноль и вывод правильного решения.
Реализуйте UNIT-тестирование модели.

\subsection*{Задание 23}
Напишите JavaFX приложение для решения неравенства $\frac{x - a}{x - b} > 0$, где $a$ и $b$ — вещественные числа, 
вводимые пользователем. Обеспечьте проверку деления на ноль и вывод правильного решения.
Реализуйте UNIT-тестирование модели.

\subsection*{Задание 24}
Напишите консольную программу, которая по введённой дате 
определяет дату предыдущего дня. Запрещено использовать 
стандартные классы для работы с датами. Учтите високосные годы 
и корректное количество дней в месяцах. Проверяйте корректность входных данных.
Реализуйте UNIT-тестирование модели.

\subsection*{Задание 25}
Напишите JavaFX программу, которая по введённой дате 
определяет дату предыдущего дня. Запрещено использовать 
стандартные классы для работы с датами. Учтите високосные годы 
и корректное количество дней в месяцах. Проверяйте корректность входных данных.
Реализуйте UNIT-тестирование модели.

\subsection*{Задание 26}
Напишите JavaFX программу, которая находит наименьшее $k$, такое 
что $k! \geqslant n$ для заданного натурального $n$. Используйте цикл while.
Реализуйте UNIT-тестирование модели.

\subsection*{Задание 27}
Напишите программу, которая читает целое
 число $n$ ($1 \leq n \leq 100$), затем $n$ строк. 
 Используя StringBuilder, обработайте каждую строку: удалите 
 все гласные и добавьте в результат только если строка 
 начинается с согласной. Между добавленными строками вставьте разделитель "---".

\subsection*{Задание 28}
Напишите программу, которая читает целое число $m$ ($1 \leq m \leq 100$), 
затем $m$ пар: строка-ключ и целое значение. Используя HashMap<String, Integer>, 
сохраните данные. Затем для каждого запроса выведите максимальное значение 
среди тех, чьи ключи заканчиваются на запрос.

\subsection*{Задание 29}
Напишите программу, которая читает $p$ целых чисел ($1 \leq p \leq 200$). 
Используя HashSet<Integer>, сохраните уникальные значения. 
Затем для каждого запроса удалите число из множества (если оно есть) и 
добавьте его удвоенное значение. В конце выведите элементы в порядке возрастания.

\subsection*{Задание 30}
Напишите программу, которая читает $r$ целых чисел ($1 \leq r \leq 100$). 
Используя ArrayList<Integer>, сохраните их. Затем обработайте 
операции: "remove Z" - удалить число Z из списка, только если оно чётное. 
После всех операций выведите оставшиеся элементы.

\subsection*{Задание 31}
Реализуйте иерархию классов для геометрических фигур: базовый 
абстрактный класс Фигура с абстрактным методом площадь(), производные 
классы Круг (радиус), Прямоугольник (длина, ширина), Треугольник (стороны). 
Реализуйте корректный ввод данных о любом количестве фигур (по выбору пользователя) 
с клавиатуры (консоль). Выведите на экран информацию о всех фигурах и сумму площадей 
всех фигур.

\subsection*{Задание 32}
Создайте класс «Время суток» с полями hour (0–23), minute (0–59), second (0–59) (все — int). 
Реализуйте методы clone, equals, hashCode и toString. При некорректных данных 
для инициализации вызывайте исключение.

\subsection*{Задание 33}
Реализуйте класс MyLinkedList с методом add(int value), добавляющим 
элемент в начало списка. Метод remove(int index) удаляет элемент по индексу. 
Метод print() выводит все элементы. Итератор возвращает элементы в порядке хранения.

\subsection*{Задание 34}
Реализуйте класс SystemSolver с методом static SystemResult solve(double a1, double b1, double c1, double a2, double b2, double c2) для решения системы 
двух линейных уравнений. SystemResult должен быть sealed-классом с подклассами 
NoSolution, InfiniteSolutions и UniqueSolution. Напишите unit-тесты.

\subsection*{Задание 35}
Реализуйте потокобезопасный класс SafeWallet с 
методами addMoney(int amount), spendMoney(int amount) (возвращает true, если хватает средств) 
и getBalance(). В main инициализируйте кошелёк с 2000 единицами,
 запустите 10 потоков: 5 потоков по 80 раз добавляют по 15, 5 потоков по 80 раз тратят по 15.
  После завершения выведите баланс.

\subsection*{Задание 36}
Реализуйте программу с графическим интерфейсом JavaFX для решения 
неравенства $ax+b>0$. Обеспечьте ввод коэффициентов, проверку корректности 
данных и вывод результатов в удобном виде. Реализуйте UNIT-тестирование модели.

\subsection*{Задание 37}
Реализуйте программу с графическим интерфейсом JavaFX, которая позволяет 
пользователю указать размерность матрицы (максимум $5\times5$), ввести 
элементы матрицы, а затем найти минимальный элемент в каждой
 строке и максимальное значение среди этих минимальных элементов.

\subsection*{Задание 38}
Создайте приложение на JavaFX для работы с базой данных доходов. 
Поля: дата, количество рублей. Используйте SQLite для хранения
 данных. Реализуйте возможность добавления, чтения, изменения записей.

\subsection*{Задание 39}
Создайте приложение на JavaFX для работы с базой данных доходов. 
Поля: дата, количество рублей. Используйте SQLite для хранения
 данных. Реализуйте возможность добавления, чтения, удаления записей.

\subsection*{Задание 40}
Разработайте REST API сервис на Spring Boot для управления информацией о доходах (поля -- дата и количество рублей). 
Предоставьте эндпоинты для CRUD-операций. Используйте реляционную базу данных 
для хранения данных. Реализуйте валидацию входных данных.

\subsection*{Задание 41}
Разработайте REST API сервис на Spring Boot для управления информацией о доходах (поля -- дата и количество рублей). 
Предоставьте эндпоинты для добавления дохода и получения всех доходов. Используйте реляционную базу данных 
для хранения данных. Реализуйте валидацию входных данных. Создайте JavaFX клиент для этого сервиса.

\subsection*{Задание 42}
Разработайте консольную программу решения неравенства $|x - a| > b$, где $a$ и $b$ — 
вещественные числа, вводимые пользователем. Обеспечьте проверку входных данных и вывод правильного решения.
Разработайте UNIT-тесты для модели.

\subsection*{Задание 43}
Разработайте JavaFX программу решения неравенства $|x - a| > b$, где $a$ и $b$ — 
вещественные числа, вводимые пользователем. Обеспечьте проверку входных данных и вывод правильного решения.
Разработайте UNIT-тесты для модели.

\subsection*{Задание 44}
Напишите консольную программу, которая по введённой дате определяет дату, 
которая наступит через месяц. Если в следующем месяце 
нет дня с таким же числом, возьмите последний день следующего месяца. 
Запрещено использовать стандартные классы для работы с датами.
Проверяйте корректность входных данных. Разработайте UNIT-тесты для модели.

\subsection*{Задание 45}
Напишите JavaFX программу, которая по введённой дате определяет дату, 
которая наступит через месяц. Если в следующем месяце 
нет дня с таким же числом, возьмите последний день следующего месяца. 
Запрещено использовать стандартные классы для работы с датами.
Проверяйте корректность входных данных. Разработайте UNIT-тесты для модели.

\subsection*{Задание 46}
Напишите программу, которая читает целое число $n$ ($1 \leq n \leq 100$), затем $n$ строк. 
Используя StringBuilder, обработайте каждую строку: удвойте каждый символ 
и добавьте в результат только если строка содержит цифру. Между добавленными 
строками вставьте разделитель "---".

\subsection*{Задание 47}
Напишите программу, которая читает целое число $m$ ($1 \leq m \leq 100$), затем $m$ пар: 
строка-ключ и целое значение. Используя HashMap<String, Integer>, сохраните данные. 
Затем для каждого запроса выведите количество ключей, которые содержат 
запрос как подстроку.

\subsection*{Задание 48}
Напишите программу, которая читает $p$ целых чисел ($1 \leq p \leq 200$). 
Используя HashSet<Integer>, сохраните уникальные значения. 
Затем для каждого запроса удалите число из множества (если оно есть) 
и добавьте число +1. В конце выведите элементы в порядке возрастания.

\subsection*{Задание 49}
Напишите программу, которая читает $r$ целых чисел ($1 \leq r \leq 100$). Используя 
ArrayList<Integer>, сохраните их. Затем обработайте операции: "swap A B" —
 поменять 
местами элементы с индексами A и B, только если сумма значений 
этих элементов чётная. После всех операций выведите список.

\subsection*{Задание 50}
Реализуйте иерархию классов для транспортных средств: базовый класс 
Транспортное средство с полями марка, скорость, производные классы Автомобиль (количество дверей, тип кузова) и Мотоцикл (объем двигателя, наличие коляски). Реализуйте метод расчета 
времени в пути для заданного расстояния. Выведите информацию обо всех транспортных средствах.

\subsection*{Задание 51}
Создайте класс «Цвет в RGB» с полями red, green, blue (тип int, значения от 0 до 255). 
Реализуйте методы clone, equals, hashCode и toString. При некорректных данных для 
инициализации вызывайте исключение.

\subsection*{Задание 52}
Реализуйте класс MyLinkedList, в котором add(int value) вставляет элемент с 
сохранением неубывающего порядка. Метод remove(int value) удаляет все вхождения значения. 
Метод print() выводит все элементы. Итератор возвращает элементы по порядку.

\subsection*{Задание 53}
Реализуйте класс LineIntersectionAnalyzer с методом 
static IntersectionResult intersect(double k1, double b1, double k2, double b2) для анализа 
пересечения прямых. IntersectionResult должен быть sealed-классом с подклассами 
ParallelDistinct, Coincident и IntersectAtPoint. Напишите unit-тесты.

\subsection*{Задание 54}
Реализуйте потокобезопасный класс ScoreBoard с методами addPoints(int points), 
removePoints(int points) (возвращает false, если недостаточно очков) и getScore(). 
В main создайте объект с начальным счётом 0, запустите 10 потоков: 
5 потоков по 120 раз добавляют по 8 очков, 5 потоков по 120 раз убирают по 8. 
После завершения выведите итоговый счёт.

\subsection*{Задание 55}
Реализуйте программу с графическим интерфейсом JavaFX для перевода 
чисел из 10-ой системы счисления в 16-ую, 8-ую и 2-ую. Обеспечьте ввод числа, 
проверку корректности данных и вывод результатов в удобном виде.

\subsection*{Задание 56}
Реализуйте программу с графическим интерфейсом JavaFX для создания 
графического редактора. Пользователь может выбирать фигуры (прямоугольник, круг), вводить координаты 
и размещать их на холсте. Удаление фигур — клик по фигуре.

\subsection*{Задание 57}
Реализуйте программу с графическим интерфейсом JavaFX для имитации движения луча света. 
Пользователь задаёт координаты источника света, направление луча. 
Границы рисунка — идеальные зеркала. Программа отображает траекторию луча с 
сохранением следа.

\subsection*{Задание 58}
Создайте приложение на JavaFX для работы с базой данных самолетов авиакомпании. 
Поля: марка самолета, количество мест. Используйте SQLite для хранения данных. 
Реализуйте возможность добавления, чтения, изменения записей.

\subsection*{Задание 59}
Создайте приложение на JavaFX для работы с базой данных самолетов авиакомпании. 
Поля: марка самолета, количество мест. Используйте SQLite для хранения данных. 
Реализуйте возможность добавления, чтения, удаления записей.

\subsection*{Задание 60}
Создайте приложение на JavaFX для работы с базой данных самолетов авиакомпании. 
Поля: марка самолета, количество мест. Используйте SQLite для хранения данных. 
Реализуйте возможность добавления, чтения, фильтрации записей по первому полю.

\subsection*{Задание 61}
Разработайте REST API сервис на Spring Boot для хранения информации о самолетах 
авиакомпании (марка самолета и количество мест). Предоставьте эндпоинты для CRUD-операций. 
Используйте реляционную базу данных для хранения данных. 
Реализуйте валидацию входных данных.

\subsection*{Задание 62}
Разработайте REST API сервис на Spring Boot для хранения информации о самолетах 
авиакомпании (марка самолета и количество мест). Предоставьте эндпоинты для создания и получения
всех записей. 
Используйте реляционную базу данных для хранения данных. 
Реализуйте клиент сервиса на JavaFX.

\subsection*{Задание 63}
Напишите консольную программу, которая решает неравенство $(x - a)^2 \geqslant 0$, 
где $a$ — вещественное число, вводимое пользователем. 
Обеспечьте полную проверку ввода данных и корректную обработку всех математических случаев.
Разработайте UNIT-тесты.

\subsection*{Задание 64}
Напишите JavaFX программу, которая решает неравенство $(x - a)^2 \geqslant 0$, 
где $a$ — вещественное число, вводимое пользователем. 
Обеспечьте полную проверку ввода данных и корректную обработку всех математических случаев.
Разработайте UNIT-тесты.

\subsection*{Задание 65}
Напишите консольную программу, которая по введённой дате определяет дату, которая была месяц назад. 
Если в предыдущем месяце нет дня с таким же числом, возьмите последний день 
предыдущего месяца. Запрещено использовать стандартные классы для работы с датами.
Проверяйте корректность исходных данных. Разработайте UNIT-тесты

\subsection*{Задание 66}
Напишите JavaFX программу, которая по введённой дате определяет дату, которая была месяц назад. 
Если в предыдущем месяце нет дня с таким же числом, возьмите последний день 
предыдущего месяца. Запрещено использовать стандартные классы для работы с датами.
Проверяйте корректность исходных данных. Разработайте UNIT-тесты

\subsection*{Задание 67}
Напишите JavaFX программу, которая проверяет, является ли натуральное число 
$n$ факториалом какого-либо натурального числа. Если да — выведите это число, 
иначе — сообщение «Не является факториалом». Используйте цикл while. Покройте ее 
UNIT-тестами (в части модели)

\subsection*{Задание 68}
Напишите программу, которая читает целое число $n$ ($1 \leq n \leq 100$), 
затем $n$ строк. Используя StringBuilder, обработайте каждую строку: 
переведите в верхний регистр и добавьте в результат только если строка заканчивается на 'a'. 
Между добавленными строками вставьте разделитель "---".

\subsection*{Задание 69}
Напишите программу, которая читает целое число $m$ ($1 \leq m \leq 100$), затем $m$ пар: 
строка-ключ и целое значение. Используя HashMap<String, Integer>, сохраните данные. 
Затем для каждого запроса выведите минимальное значение среди тех,
 чьи ключи точно равны запросу.

\subsection*{Задание 70}
Напишите программу, которая читает $p$ целых чисел ($1 \leq p \leq 200$). 
Используя HashSet<Integer>, сохраните уникальные значения. 
Затем для каждого запроса удалите число из множества (если оно есть) и добавьте его 
факториал (для чисел до 10). В конце выведите элементы в порядке возрастания.

\subsection*{Задание 71}
Напишите программу, которая читает $r$ целых чисел ($1 \leq r \leq 100$). 
Используя ArrayList<Integer>, сохраните их. Затем обработайте операции: 
"add X to begin" - добавить число X в начало списка, только если X нечётное. 
После всех операций выведите список в обратном порядке.

\subsection*{Задание 72}
Реализуйте иерархию классов для документов: базовый класс Документ с полями 
номер, дата, производные классы Счёт (сумма, покупатель) и Приказ (текст приказа, исполнитель). 
Реализуйте метод вывода информации о документе. 
Выведите список документов в хронологическом порядке (с полной информацией о них).

\subsection*{Задание 73}
Создайте класс «Геокоординаты с высотой» с полями 
latitude, longitude (double), altitude (int, в метрах). 
Реализуйте методы clone, equals, hashCode и toString. При некорректных данных 
для инициализации вызывайте исключение. latitude может быть от -90 до +90, longitude 
от -180 до 180.

\subsection*{Задание 74}
Реализуйте класс MyLinkedList с методом add(int target, int value), добавляющим 
value после первого вхождения target (если target не найден — в конец). 
Метод remove(int value) удаляет первое вхождение. Метод print() выводит все элементы. 
Итератор возвращает все элементы.

\subsection*{Задание 75}
Реализуйте класс HeronTriangle с методом static AreaResult computeArea(double a, double b, double c) 
для вычисления площади треугольника по формуле Герона. 
AreaResult должен быть sealed-классом с подклассами InvalidTriangle, Degenerate и ValidTriangle. 
Напишите unit-тесты.

\subsection*{Задание 76}
Реализуйте потокобезопасный класс SafeAccount 
с методами topUp(int sum), pay(int sum) (возвращает true, если оплата возможна) 
и getAmount(). В main создайте счёт с 1500, запустите 10 потоков: 
5 потоков по 90 раз пополняют на 20, 5 потоков по 90 раз оплачивают по 20. 
После завершения выведите баланс.

\subsection*{Задание 77}
Реализуйте программу с графическим интерфейсом JavaFX, которая читает файл с 
описанием расположения TextField'ов (координаты). После выбора файла на 
форме динамически создаются TextField'ы согласно данным из файла. 
При изменении любого числа в TextField'ах, в Label автоматически обновляется 
сумма всех введённых чисел.

\subsection*{Задание 78}
Реализуйте программу с графическим интерфейсом JavaFX для имитации движения двух планет. 
Пользователь задаёт начальные координаты и векторы скоростей двух шаров. 
На каждый шар действует сила притяжения к другому, равная $\frac{k}{r^2}$. 
Движение прекращается при столкновении шаров.

\subsection*{Задание 79}
Создайте приложение на JavaFX для работы с базой данных автобусных маршрутов. 
Поля: номер маршрута, номер парка. Используйте SQLite для хранения данных. 
Реализуйте возможность добавления, чтения, изменения записей.

\subsection*{Задание 80}
Создайте приложение на JavaFX для работы с базой данных автобусных маршрутов. 
Поля: номер маршрута, номер парка. Используйте SQLite для хранения данных. 
Реализуйте возможность добавления, чтения, удаления записей.

\subsection*{Задание 81}
Создайте приложение на JavaFX для работы с базой данных автобусных маршрутов. 
Поля: номер маршрута, номер парка. Используйте SQLite для хранения данных. 
Реализуйте возможность добавления, чтения, фильтрации записей по второму полю.

\subsection*{Задание 82}
Разработайте REST API сервис на Spring Boot для управления автобусными маршрутами (номер маршрута и номер парка). 
Предоставьте эндпоинты для CRUD-операций. Используйте реляционную базу данных для хранения данных. 
Реализуйте валидацию входных данных.

\subsection*{Задание 83}
Разработайте REST API сервис на Spring Boot для управления автобусными маршрутами (номер маршрута и номер парка). 
Предоставьте эндпоинты для добавления и получения всех записей. Используйте реляционную базу данных для хранения данных. 
Разработайте клиент сервиса на JavaFX.

\subsection*{Задание 84}
Напишите консольную программу, которая решает неравенство 
$(a - x)(b - x) < 0$, где $a$ и $b$ — вещественные числа, вводимые пользователем. 
Обеспечьте полную проверку ввода данных и корректную обработку всех математических случаев.
Разработайте UNIT-тесты для модели.

\subsection*{Задание 85}
Напишите JavaFX программу, которая решает неравенство 
$(a - x)(b - x) < 0$, где $a$ и $b$ — вещественные числа, вводимые пользователем. 
Обеспечьте полную проверку ввода данных и корректную обработку всех математических случаев.
Разработайте UNIT-тесты для модели.

\subsection*{Задание 86}
Напишите консольную программу, которая по введённой дате определяет дату, 
которая наступит через 2 месяца. Учтите переход между годами 
и корректировку дня при отсутствии соответствующего дня в следующем месяце. 
Запрещено использовать стандартные классы для работы с датами.
Проверяйте корректность исходных данных. Разработайте UNIT-тесты для модели.

\subsection*{Задание 87}
Напишите JavaFX программу, которая по введённой дате определяет дату, 
которая наступит через 2 месяца. Учтите переход между годами 
и корректировку дня при отсутствии соответствующего дня в следующем месяце. 
Запрещено использовать стандартные классы для работы с датами.
Проверяйте корректность исходных данных. Разработайте UNIT-тесты для модели.

\subsection*{Задание 88}
Напишите программу, которая читает целое 
число $n$ ($1 \leq n \leq 100$), затем $n$ строк. Используя StringBuilder, 
обработайте каждую строку: удалите пробелы и добавьте в результат 
только если длина строки чётная. Между добавленными строками вставьте разделитель "---".

\subsection*{Задание 89}
Напишите программу, которая читает целое число $m$ ($1 \leq m \leq 100$), 
затем $m$ пар: строка-ключ и целое значение. Используя HashMap<String, Integer>, 
сохраните данные. Затем для каждого запроса выведите 
среднее значение (целое) среди тех, чьи ключи имеют ту же длину, что и запрос.

\subsection*{Задание 90}
Напишите программу, которая читает $p$ целых чисел ($1 \leq p \leq 200$). Используя 
HashSet<Integer>, сохраните уникальные значения. Затем для каждого 
запроса удалите число из множества (если оно есть) и добавьте 
его целую часть квадратного корня. В конце выведите элементы в порядке возрастания.

\subsection*{Задание 91}
Напишите программу, которая читает $r$ целых чисел ($1 \leq r \leq 100$). Используя 
ArrayList<Integer>, сохраните их. Затем обработайте операции: "remove последний" - 
удалить последний элемент списка, только если он положительный. 
После всех операций выведите список.

\subsection*{Задание 92}
Реализуйте иерархию классов для животных: базовый класс Животное с полями имя, 
возраст, производные классы Собака (порода, команды) и Кошка (окрас, характер). 
Реализуйте метод издавания звука (гав-гав для собаки, мяу для кошки). Выведите 
информацию обо всех животных.

\subsection*{Задание 93}
Создайте класс «Дробь с единицей измерения» с полями numerator, denominator 
(int, знаменатель $\neq$ 0), unit (String). Реализуйте методы 
clone, equals, hashCode и toString. При некорректных данных для 
инициализации вызывайте исключение.

\subsection*{Задание 94}
Реализуйте класс MyLinkedList с методом add(int index, int value), 
добавляющим элемент по индексу (с проверкой границ). Метод remove(int index) 
удаляет элемент по индексу. Метод print() выводит все элементы. 
Итератор возвращает элементы в прямом порядке.

\subsection*{Задание 95}
Реализуйте класс TriangleClassifier с методом static Classification 
classify(double a, double b, double c) для классификации треугольника. 
Classification должен быть sealed-классом с подклассами NotTriangle, 
Equilateral, IsoscelesRight, Isosceles, ScaleneRight, ScaleneAcuteOrObtuse. 
Напишите unit-тесты.

\subsection*{Задание 96}
Реализуйте потокобезопасный класс EnergyMeter с методами charge(int units), 
consume(int units) (возвращает true, если энергии хватает) и getLevel(). В main 
начальный уровень — 1000. Запустите 10 потоков: 5 потоков по 100 раз 
заряжают на 10, 5 потоков по 100 раз потребляют по 10. После завершения 
выведите итоговый уровень.

\subsection*{Задание 97}
Реализуйте программу с графическим интерфейсом JavaFX для поиска 
времени окончания интервала. Дано: часы и минуты начала интервала 
и количество минут, сколько он идет. Результат: часы 
и минуты окончания интервала. Обеспечьте ввод данных, 
проверку корректности и вывод результата.

\subsection*{Задание 98}
Реализуйте программу с графическим интерфейсом JavaFX, которая читает файл 
с описанием Label'ов и интервалов времени. После загрузки на форме 
появляются Label'ы с начальными значениями. Для каждого Label 
создаётся поток, который с указанным интервалом обновляет значение (увеличивает на 1).

\subsection*{Задание 99}
Реализуйте программу с графическим интерфейсом JavaFX для имитации 
движения спутника. Один шар неподвижен (центр масс), второй — спутник с
 заданной начальной скоростью. На спутник действует сила притяжения к 
 центру: $\frac{k}{r^2}$.

\end{document}
